\pseudochapter{Background I}

Before jumping into the main subject of this text, we will spend a few pages on some basic background material on rings, fields, and matrices. The results here are mostly a nuts-and-boltsy support structure to shore up the material to come later. Much of it will be familiar to the reader, and so many of the main ideas are outlined in the exercises.

The first kind of gadget discussed in this chapter is the class of \emph{rings} (and of the ring's close cousins, \emph{fields}). A ring is a kind of algebra which enjoys the basic behavior of arithmetic on the integers: that is, the interplay between addition and multiplication. A field is a ring in which multiplication is always invertible. Our prototypical examples of a ring and a field are the integers $\mathbb{Z}$ and the rational numbers $\mathbb{Q}$, respectively. But beware trying to generalize too much of the behavior of these familiar examples to all rings and fields - there are lots of much more exotic and interesting specimens to be found.

The second main idea we discuss here is that of a \emph{matrix}. For now, we will define matrices and their basic operations in a purely concrete way, thinking of them ``merely'' as a kind of abstract data structure. In later sections we will see that these concrete operations on matrices have very nice and useful abstract interpretations.