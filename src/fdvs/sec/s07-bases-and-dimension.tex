\section{Bases and Dimension}

\begin{prp}
Let $X \subseteq V$ be a subset. Then the following are equivalent.
\begin{enumerate*}
\item $X$ is an independent generating set of $V$.
\item $X$ is a $\subseteq$-minimal generating set of $V$.
\item $X$ is a $\subseteq$-maximal independent set in $V$.
\item $V$ is free on $X$.
\item Every $v \in V$ can be written uniquely as a linear combination of elements of $X$.
\end{enumerate*}
A subset having any (hence all) of these properties is called a \emph{basis}.
\end{prp}

\begin{proof}
(@@@)
\end{proof}

\begin{prp}
Every vector space has a basis.
\end{prp}

\begin{proof}
(@@@)
\end{proof}

\begin{prp}
Let $V$ be a vector space. Any two bases in a vector space have the same cardinality, which we call the \emph{dimension} of $V$ and denote $\Dim[F]{V}$.
\end{prp}

\begin{proof}
(@@@)
\end{proof}

\begin{exm}
\begin{enumerate*}
\item $\mathsf{dim}_F(F) = 1$.

\item Let $F_X$ be the free vector space on $X$ defined in (@@@). In (@@@), we showed that the set $\mathcal{B} = \{\alpha_x \mid x \in X\}$ is a generating set of $F_X$, and in (@@@) we showed that $\mathcal{B}$ is independent. Thus $\mathcal{B}$ is a basis of $F_X$ and $\Dim[F]{F_X} = |X|$.
\end{enumerate*}
\end{exm}

We say a vector space is \emph{finite dimensional} if (surprise!) its dimension is finite. Dimension gives us a way to speak of the ``size'' of a vector space that is more refined than cardinality. One consequence of this, which we will get quite a bit of use out of later, is that we can use induction to prove theorems about (finite dimensional) vector spaces. This is a good example of a common strategy in algebra. When studying gadgets of a particular type, it is frequently useful to attach a natural number to each gadget in a structurally respectful way and try to use inductive or (if we're lucky) number-theoretic arguments.

\begin{prp}
\begin{enumerate*}
\item Every generating set contains a basis.
\item Every independent set is contained in a basis.
\end{enumerate*}
\end{prp}

\begin{proof}
(@@@)
\end{proof}

\NowForSomeExercises

\begin{exercises}
\item{\label{exr:dim-zero}}
Show that if $V$ has dimension 0 then $V = 0$.

\item{\label{exr:dim-of-product}}
Show that if $V$ and $W$ are finite-dimensional vector spaces then so is $V \times W$, and $\Dim[F]{V \times W} = \Dim[F]{V} + \Dim[F]{W}$.

\item{\label{exr:dim-field-ext}}
Let $E$ be a field with $F \subseteq E$ a subfield such that $\Dim[F]{E}$ is finite. Suppose $V$ is a finite dimensional vector space over $E$. We saw in (@@@) that $V$ is naturally a vector space over $F$. Show that $V$ has finite dimension over $F$ and that $\Dim[F]{V} = \Dim[F]{E} \cdot \Dim[E]{V}$.

\item{\label{exr:oddtown}}
(Oddtown)
\end{exercises}