One of the goals of this class is to explore the classical theory of geometry as laid out in the famous \emph{Elements of Geometry} by the Greek mathematician Euclid. Before we begin, though, a few words about the distinction between a \emph{theory} and a \emph{model} are in order.

\begin{itemize}
\item A \textbf{theory} consists of one or more \emph{undefined terms}, which are used in one or more \emph{axioms} or \emph{definitions}, which are then the basis for a list of logical deductions called \emph{theorems}. This may be how you think about geometry as you learned it in high school.

\item A \textbf{model} is a concrete realization of a theory: a way to associate the undefined terms to ``real'' objects such that the axioms are satisfied. Models are where we perform calculations and draw pictures.
\end{itemize}

The correspondence between theories and models is not one-to-one (in either direction). There are theories which have many models, other theories with exactly one model, and yet other theories which have no models at all. Conversely, a given concrete object is likely a model of many different theories, though not all of them will be interesting.

The difference between theory and model turns out to have significant practical applications. In the real world we compute with concrete objects -- things like numbers and sets. This is useful, but concrete objects have a tendency to get very messy very quickly. However if some aspects of a concrete object are a model for some theory, we can ``throw away'' unimportant details and compute more easily at an abstract level. For instance many important theorems about matrices are difficult and tedious to prove if we think of matrices as arrays of numbers but become simple if we think of matrices as linear transformations.

An example of a theory which you may have seen before is Euclid's postualtes for geometry. These are a small number of statements which Euclid took to be obviously true (\emph{axioms} in modern lingo) such as ``two points determine a line'' and so on. Euclid developed this theory of geometry in a book, called \emph{The Elements}, which went on to become a standard mathematical textbook for many centuries.

The development and proliferation of Euclid's geometry predates the recognition of the need to carefully distinguish between a theory and its models, and early work did not make this distinction. Euclid seems to have written under the assumption that the universe comes equipped with exactly one geometry -- that his theory has only one model. Unfortunately for us, this early confusion led to two problems which are relevant for us. First, it turns out that there are many models of geometry, some of which are very strange. We will explore several models of geometry in this text.

The second problem we inherit from Euclid is more serious. Because he conflated his \emph{theory} of geometry with only one particular \emph{model}, and this confusion was not cleared up until much later, and because his book was so influential, Euclid left us with a language problem. The basic terms of geometry -- point, line, circle, segment, angle -- have multiple meanings. There is the meaning inside Euclid's \emph{model} of geometry, which corresponds mostly to the bits of geometry we use in college algebra and calculus. But these words have another, more abstract meaning inside Euclid's \emph{theory} of geometry, and when these terms are used inside other models we can easily get confused. We will see models of geometry where lines are circles, where points are lines, and where circles are ellipses. As we'll see, the key to keeping all of this straight is to think of some words as essentially meaningless.
