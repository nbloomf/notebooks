Traditional plane geometry involves many different concepts, including \emph{lines}, \emph{angles}, \emph{congruent}, and many others.
In order to manage the complexity this entails, we will build up our geometries one piece at a time starting here with the idea of \emph{collinearity}.

\begin{dfn}[Incidence Geometry] \label{dfn:incidence-geometry}
Let \(P\) be a set, whose elements we call \emph{points}.
A ternary relation \(\COLLINEAR{\ast}{\ast}{\ast}\) on \(P\) is called a \emph{collinearity relation}\index{collinearity} if the following properties are satisfied.
\begin{proplist}
\item[IG1.] If \(a\), \(b\), and \(c\) are points such that \(\COLLINEAR{a}{b}{c}\), then \(\COLLINEAR{b}{a}{c}\) and \(\COLLINEAR{a}{c}{b}\).
\item[IG2.] If \(a\) and \(b\) are points such that \(a \neq b\), then \(\COLLINEAR{a}{b}{b}\).
\item[IG3.] \(\COLLINEAR{a}{a}{a}\) does not hold for any point \(a\).
\item[IG4.] There exist points \(a\), \(b\), and \(c\) such that \(\COLLINEAR{a}{b}{c}\) does not hold.
\item[IG5.] If \(a\), \(b\), \(u\), and \(v\) are points such that \(a \neq b\), \(u \neq v\), \(\COLLINEAR{a}{b}{u}\), and \(\COLLINEAR{a}{b}{v}\), then \(\COLLINEAR{a}{u}{v}\).
\end{proplist}
If such a relation exists we say that \((P, \COLLINEAR{\ast}{\ast}{\ast})\) is an \emph{incidence geometry}\index{incidence geometry}.
In this case, when \(\COLLINEAR{a}{b}{c}\) we say that \(a\), \(b\), and \(c\) are \emph{collinear}.
\end{dfn}

It is important to remember that the word ``collinear'' here is an undefined term, and we have to try very hard not to think of ordinary lines and points when using it.
(This is part of the theory-model confusion we inherit from history.)
The meaning of the word ``collinear'' is determined precisely by how it is used in the incidence geometry axioms, and only becomes concrete when we specify a particular model.
In particular, it does not make sense to draw pictures of the points in an arbitrary incidence geometry!

Although ``collinear'' is an abstract, undefined term, we'd like for it to behave much like our intuition expects.
We want this undefined term to \emph{formalize} our intuition about collinearity.
To that end, note that collinearity satisfies some additional basic properties.

\begin{prop}
Let \(P\) be an incidence geometry.
Then we have the following for all points \(a\), \(b\), and \(c\).
\begin{proplist}
\item If \(\COLLINEAR{a}{b}{c}\), then we also have \(\COLLINEAR{b}{c}{a}\), \(\COLLINEAR{c}{a}{b}\), and \(\COLLINEAR{c}{b}{a}\).
\item If \(a \neq b\), then \(\COLLINEAR{a}{b}{a}\) and \(\COLLINEAR{b}{a}{a}\).
\end{proplist}
\end{prop}

It may seem strange to define ``collinear'' before we define ``line''; typically we think of points being collinear precisely when there is a unique line containing all of them.
But we can just as easily define lines in terms of collinearity as follows.

\begin{dfn}[Line] \label{dfn:line}
Let \(P\) be an incidence geometry with distinct points \(a\) and \(b\).
We define the \emph{line}\index{line} through \(a\) and \(b\) to be the set \[ \LINE{a}{b} = \{ c \in P \mid \COLLINEAR{a}{b}{c} \}. \]
If \(c \in \LINE{a}{b}\), we say that \(c\) \emph{lies on} \(\LINE{a}{b}\).
\end{dfn}

That is, the line through \(a\) and \(b\) is precisely the set of points which are collinear with \(a\) and \(b\).
Remember: it is vital that we not think about drawings of points and lines here.
In an arbitrary incidence geometry, ``point'' and ``line'' are just \emph{words} which we assume have a particular relationship with one another.
Thinking at this level of abstraction may seem unnecessarily difficult at first, but -- and it is difficult to overstate this -- the abstract way of thinking brings enormous power.
Here are some basic properties of lines which can be derived from the properties of collinearity alone.

\begin{prop}
Let \(P\) be an incidence geometry.
Then the following hold for all distinct points \(a\) and \(b\) in \(P\).
\begin{proplist}
\item \(a \in \LINE{a}{b}\) and \(b \in \LINE{a}{b}\).
\item \(\LINE{a}{b} = \LINE{b}{a}\).
\item If \(c \in \LINE{a}{b}\) and \(c \neq a\), then \(\LINE{a}{c} = \LINE{a}{b}\).
\item If \(u,v \in \LINE{a}{b}\) are distinct points, then \(a,b \in \LINE{u}{v}\).
\end{proplist}
\end{prop}

Though we have several examples of incidence geometry, it is crucial when proving theorems that we not rely on any specific model.
This is the power of abstraction: any theorem which depends only on properties common to \emph{all} incidence geometries immediately holds in \emph{any} incidence geometry.
For example, consider the following theorem.

\begin{prop}[Line Intersection]
Let \(P\) be an incidence geometry with lines \(\ell_1\) and \(\ell_2\).
Then exactly one of the following holds.
\begin{proplist}
\item \(\ell_1 = \ell_2\), and we say \(\ell_1\) and \(\ell_2\) are \emph{coincident}\index{coincident lines},
\item \(\ell_1 \cap \ell_2 = \varnothing\), and we say \(\ell_1\) and \(\ell_2\) are \emph{disjoint}\index{disjoint lines}, or
\item \(\ell_1 \cap \ell_2 = \{p\}\) for some point \(p\), and we say \(\ell_1\) and \(\ell_2\) are \emph{incident}\index{incident lines}.
\end{proplist}
In the first two cases (coincident or disjoint) we say that \(\ell_1\) and \(\ell_2\) are \emph{parallel}\index{parallel lines}.
\end{prop}

\begin{proof}
Suppose \(\ell_1 \cap \ell_2\) contains at least two points, say \(x\) and \(y\).
Then in fact \(\ell_1 = \LINE{x}{y} = \ell_2\).
So if \(\ell_1 \neq \ell_2\) then \(\ell_1 \cap \ell_2\) contains either exactly one or zero points.
\end{proof}

This theorem holds in any model of incidence geometry.
One problem: we don't have any models of incidence geometry yet!
We'll fix this in the next section.



%---------%
\Exercises%
%---------%

\begin{exercise}
Let \(A = (a_x, a_y, a_z)\), \(B = (b_x, b_y, b_z)\), and \(C = (c_x, c_y, c_z)\) be in \(\RR^3\) such that \(A\) and \(B\) are nonzero and \(B\) is not a multiple of \(A\).
Show that \[ \DET \begin{bmatrix} a_x & a_y & a_z \\ b_x & b_y & b_z \\ c_x & c_y & c_z \end{bmatrix} = 0 \] if and only if \(C = A + t(B - A)\) for some unique \(t \in \RR\).
\end{exercise}

