\subsection{The Unit Disc}

Once we have an incidence geometry \(P\) lying around, one way to try to build new ones is by restricting the collinearity on \(P\) to subsets of \(P\).

\begin{prop}[Unit Disc]
Let \(\mathbb{D} = \{ (x,y) \in \RR^2 \mid x^2 + y^2 < 1 \}\); these are points in the cartesian plane which are inside the unit circle.
Given points \(A\), \(B\), and \(C\) in \(\mathbb{D}\), we say they are collinear in \(\mathbb{D}\) if they are collinear in \(\RR^2\).
This relation makes \(\mathbb{D}\) an incidence geometry which we call the \emph{Unit Disc}.
\end{prop}

Lines in the unit disc are chords of the unit circle (not including their endpoints).
Already the word ``line'' is being twisted.



\subsection{The Fano Plane}

We will now see a very different and somewhat strange model of incidence geometry.

\begin{dfn}[The Fano Plane]
Let \(P = \{1,2,3,4,5,6,7\}\), and then let \(L = \{\{1,2,3\},\) \(\{2,4,6\},\) \(\{1,4,7\},\) \(\{1,5,6\},\) \(\{2,5,7\},\) \(\{3,4,5\},\) \(\{3,6,7\}\}\).
We then define a ternary relation on \(P\) by \(\COLLINEAR{a}{b}{c}\) if and only if \(\{a,b,c\}\) is not a singleton and is contained in some \(\ell \in L\).
The set \(P\) with this ternary relation is called the \emph{Fano Plane}.
\end{dfn}

It turns out that the Fano plane is an incidence geometry.
