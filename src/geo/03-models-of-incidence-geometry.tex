\subsection{The Fano Plane}

We will now see a very different and somewhat strange model of incidence geometry.

\begin{dfn}[The Fano Plane]
Let \(P = \{1,2,3,4,5,6,7\}\), and then let \(L = \{\{1,2,3\},\) \(\{2,4,6\},\) \(\{1,4,7\},\) \(\{1,5,6\},\) \(\{2,5,7\},\) \(\{3,4,5\},\) \(\{3,6,7\}\}\).
We then define a ternary relation on \(P\) by \(\COLLINEAR{a}{b}{c}\) if and only if \(\{a,b,c\}\) is not a singleton and is contained in some \(\ell \in L\).
The set \(P\) with this ternary relation is called the \emph{Fano Plane}.
\end{dfn}

It turns out that the Fano plane is an incidence geometry.
