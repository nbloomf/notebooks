We will now construct a strange model of incidence geometry.
For this example, our set of points is \(\mathbb{H} = \{ (x,y) \mid x,y \in \RR\ \mathrm{and}\ y > 0 \}\).
Suppose we have two such points \(A = (a_x, a_y)\) and \(B = (b_x, b_y)\).
If \(a_x \neq b_x\), then the \emph{ideal point} of \(A\) and \(B\) is the number \[ I_{A,B} = \frac{b_x^2 + b_y^2 - a_x^2 - a_y^2}{2(b_x - a_x)}. \]
Intuitively, the ideal point is the \(x\)-coordinate of the point on the cartesian \(x\)-axis which is ``equidistant'' from \(A\) and \(B\).
More important for us is the following property of the ideal point.

\begin{lem}
Let \(A = (a_x, a_y)\) and \(B = (b_x, b_y)\) be points in \(\mathbb{H}\) such that \(a_x \neq b_x\).
Then the ideal point \(I_{A,B}\) of \(A\) and \(B\) is the unique solution \(X\) of the following equation.
\[ (a_x - X)^2 + a_y^2 = (b_x - X)^2 + b_y^2. \]
\end{lem}

We use the ideal point to define a collinearity relation on \(\mathbb{H}\) as follows.

\begin{prop}\label{prop:hyp-half-plane}
Define a ternary relation \(\COLLINEAR{\ast}{\ast}{\ast}\) on \(\mathbb{H}\) as follows.
Given points \(A = (a_x, a_y)\), \(B = (b_x, b_y)\), and \(C = (c_x, c_y)\) in \(\mathbb{H}\), we say that \(\COLLINEAR{A}{B}{C}\) if \(A\), \(B\), and \(C\) are not all equal and one of the following holds.
\begin{proplist}
\item \(a_x = b_x\) and \(a_y = b_y\)
\item \(a_x = b_x\) and \(a_y \neq b_y\) and \(c_x = a_x\).
\item \(a_x \neq b_x\) and \[ (c_x - I_{A,B})^2 + c_y^2 = (a_x - I_{A,B})^2 + a_y^2. \]
\end{proplist}
This relation makes \(\mathbb{H}\) an incidence geometry, which we call the \emph{hyperbolic half-plane}\index{hyperbolic half-plane}.
\end{prop}

\begin{proof}
Note that IG3 holds by definition. (@@@)
\end{proof}

We can show that the lines in this incidence geometry come in two flavors, which we will call the Type I and Type II lines.

\begin{cor}[Lines in \(\mathbb{H}\)]
Let \(A, B \in \mathbb{H}\) be distinct points.
\begin{proplist}
\item If \(a_x = b_x\) we say \(\LINE{A}{B}\) is of \emph{Type I}, and we have \[ \LINE{A}{B} = \{ (x,y) \in \mathbb{H} \mid x = a_x \}. \]
\item If \(a_x \neq b_x\) we say \(\LINE{A}{B}\) is of \emph{Type II}, and we have \[ \LINE{A}{B} = \{ (x,y) \in \mathbb{H} \mid (x - I_{A,B})^2 + y^2 = (a_x - I_{A,B})^2 + a_y^2 \}. \]
\end{proplist}
\end{cor}

Viewed as sets in the cartesian plane, the Type I lines in \(\mathbb{H}\) are vertical half lines, and the Type II lines in \(\mathbb{H}\) are semicircles centered on the \(x\)-axis.
In fact the cartesian center of a Type II line is the point \((I_{A,B},0)\).

\begin{figure}[h]
\begin{center}
\begin{tikzpicture}
  \draw [dashed,<->] (-3,0) -- (3,0);
  \draw [->] (-2,0) -- (-2,3);
  \draw (2,0) arc (0:180:1.5);
  \node at (-3,1) {Type I};
  \node at (3,1) {Type II};
\end{tikzpicture}
\caption{\label{fig:lines-in-hyp-half-plane}Lines in \(\mathbb{H}\).}
\end{center}
\end{figure}



%---------%
\Exercises%
%---------%

\begin{exercise}
Complete the proof of \propref{prop:hyp-half-plane}.
\end{exercise}
