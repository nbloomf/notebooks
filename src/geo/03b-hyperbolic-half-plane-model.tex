We will now construct an odd model of incidence geometry.
For this example, our set of points is \(\HYPHP = \{ (x,y) \mid x,y \in \RR\ \mathrm{and}\ y > 0 \}\) -- the ``upper half plane''.
Suppose we have two such points \(A = (a_x, a_y)\) and \(B = (b_x, b_y)\).
If \(a_x \neq b_x\), then we define the \emph{hyperbolic ideal point}\index{hyperbolic ideal point} of \(A\) and \(B\) is the number \[ \HHPIDEALPOINT{A}{B} = \frac{b_x^2 + b_y^2 - a_x^2 - a_y^2}{2(b_x - a_x)}. \]
Intuitively, the ideal point is the \(x\)-coordinate of the point on the cartesian \(x\)-axis which is ``equidistant'' from \(A\) and \(B\).
More precisely, we have the following.

\begin{lem}\label{lem:hhp-ip-sol}
Let \(A = (a_x, a_y)\) and \(B = (b_x, b_y)\) be points in \(\HYPHP\) such that \(a_x \neq b_x\).
Then we have the following.
\begin{proplist}
\item \label{lem:hhp-ip-sol:uniq} The ideal point \(\HHPIDEALPOINT{A}{B}\) is the unique solution \(z\) of the following equation.
\[ (a_x - z)^2 + a_y^2 = (b_x - z)^2 + b_y^2. \]
\item \label{lem:hhp-ip-sol:sym} \(\HHPIDEALPOINT{A}{B} = \HHPIDEALPOINT{B}{A}\).
\end{proplist}
\end{lem}

Take care that \(\HHPIDEALPOINT{A}{B}\) only exists if \(A\) and \(B\) have distinct \(x\)-coordinates.
We use the ideal point to define a collinearity relation on \(\HYPHP\) as follows.

\begin{prop}\label{prop:hyp-half-plane}
Define a ternary relation \(\COLLINEAR{\ast}{\ast}{\ast}\) on \(\HYPHP\) as follows.
Given points \(A = (a_x, a_y)\), \(B = (b_x, b_y)\), and \(C = (c_x, c_y)\) in \(\HYPHP\), we say that \(\COLLINEAR{A}{B}{C}\) if one of the following holds.
\begin{proplist}
\item \(a_x = b_x\) and \(a_y = b_y\) and \(C \neq A\).
\item \(a_x = b_x\) and \(a_y \neq b_y\) and \(c_x = a_x\).
\item \(a_x \neq b_x\) and \( (c_x - \HHPIDEALPOINT{A}{B})^2 + c_y^2 = (a_x - \HHPIDEALPOINT{A}{B})^2 + a_y^2 \).
\end{proplist}
This is a collinearity relation on the set \(\HYPHP\), which we call the \emph{hyperbolic half plane}\index{hyperbolic half plane}.
\end{prop}

\begin{proof}\mbox{}
\begin{itemize}
\item[IG2.] Suppose \(A \neq B\).
If \(a_x = b_x\), then we must have \(a_y \neq b_y\).
Then \(\COLLINEAR{A}{B}{B}\) since \(b_x = a_x\).
If instead we have \(a_x \neq b_x\), then \(\COLLINEAR{A}{B}{B}\) using \sref{lem:hhp-ip-sol}{uniq}.

\item[IG3.] Note that \(\COLLINEAR{A}{A}{A}\) does not hold, because in this case we have \(a_x = a_x\) and \(a_y = a_y\) but \(A = A\).

\item[IG4.] Let \(A = (0,1)\), \(B = (0,2)\), and \(C = (1,1)\).
Now \(\COLLINEAR{A}{B}{C}\) does not hold, because we have \(a_x = b_x\) and \(a_y \neq b_y\) but \(c_x \neq a_x\).

\item[IG1.] Suppose \(\COLLINEAR{A}{B}{C}\).
First we will show that \(\COLLINEAR{B}{A}{C}\) by considering the three possibilities in the definition of \(\COLLINEAR{\ast}{\ast}{\ast}\).
If \(a_x = b_x\) and \(a_y = b_y\) and \(C \neq A\), then we have \(b_x = a_x\) and \(b_y = a_y\) and \(C \neq B\), so that \(\COLLINEAR{B}{A}{C}\).
If \(a_x = b_x\) and \(a_y \neq b_y\) and \(c_x = a_x\), then we have \(b_x = a_x\) and \(b_y \neq a_y\) and \(c_x = b_x\), so that \(\COLLINEAR{B}{A}{C}\).
Suppose now that \(a_x \neq b_x\) and \[ (c_x - \HHPIDEALPOINT{A}{B})^2 + c_y^2 = (a_x - \HHPIDEALPOINT{A}{B})^2 + a_y^2. \]
Now we have
\begin{eqnarray*}
(c_x - \HHPIDEALPOINT{B}{A})^2 + c_y^2
 & = & (c_x - \HHPIDEALPOINT{A}{B})^2 + c_y^2 \\
 & = & (a_x - \HHPIDEALPOINT{A}{B})^2 + a_y^2 \\
 & = & (b_x - \HHPIDEALPOINT{A}{B})^2 + b_y^2 \\
 & = & (b_x - \HHPIDEALPOINT{B}{A})^2 + b_y^2
\end{eqnarray*}
so that \(\COLLINEAR{B}{A}{C}\).

Next we show that \(\COLLINEAR{A}{C}{B}\), again by considering the three possibilities.
If \(a_x = b_x\) and \(a_y = b_y\) and \(C \neq A\) then we have \(\COLLINEAR{C}{A}{B}\) by IG2, and in the previous paragraph we saw that we can swap the first two slots in \(\COLLINEAR{\ast}{\ast}{\ast}\), so that \(\COLLINEAR{A}{C}{B}\).
Suppose \(a_x = b_x\) and \(a_y \neq b_y\) and \(c_x = a_x\).
Now if \(c_y = a_y\), then we have \(a_x = c_x\) and \(a_y = c_y\) and \(B \neq A\), so that \(\COLLINEAR{A}{C}{B}\).
If instead we have \(c_y \neq a_y\), then we have \(a_x = c_x\) and \(a_y \neq c_y\) and \(b_x = a_x\), so that \(\COLLINEAR{A}{C}{B}\).
Finally, suppose that \(a_x \neq b_x\) and \[ (c_x - \HHPIDEALPOINT{A}{B})^2 + c_y^2 = (a_x - \HHPIDEALPOINT{A}{B})^2 + a_y^2. \]
If \(a_x = c_x\), then we have \(a_y = c_y\) since \(a_y, c_y > 0\).
Since \(b_x \neq a_x\), we have \(\COLLINEAR{A}{C}{B}\).
Suppose instead that \(a_x \neq c_x\).
By \lemref{lem:hhp-ip-sol}, we have \(\HHPIDEALPOINT{A}{C} = \HHPIDEALPOINT{A}{B}\).
Thus \[ (b_x - \HHPIDEALPOINT{A}{C})^2 + b_y^2 = (a_x - \HHPIDEALPOINT{A}{C})^2 + a_y^2, \] and thus \(\COLLINEAR{A}{C}{B}\).

\item[IG5.] Suppose we have points \(A \neq B\) and \(U \neq V\) such that \(\COLLINEAR{A}{B}{U}\) and \(\COLLINEAR{A}{B}{V}\).
If \(a_x = b_x\), then we have \(u_x = a_x\) and \(v_x = a_x\).
Now \(u_x = v_x\) and so \(u_y \neq v_y\) and \(a_x = u_x\), so that \(\COLLINEAR{U}{V}{A}\), and thus \(\COLLINEAR{A}{U}{V}\) by IG1.
Suppose instead that \(a_x \neq b_x\).
Then we have \[ (u_x - \HHPIDEALPOINT{A}{B})^2 + u_y^2 = (a_x - \HHPIDEALPOINT{A}{B})^2 + a_y^2 \] and \[ (v_x - \HHPIDEALPOINT{A}{B})^2 + v_y^2 = (a_x - \HHPIDEALPOINT{A}{B})^2 + a_y^2. \]
If \(a_x = u_x\), then \(u_y = a_y\) and \(V \neq A\), so that \(\COLLINEAR{A}{U}{V}\).
If \(a_x \neq u_x\) we have \(\HHPIDEALPOINT{A}{U} = \HHPIDEALPOINT{A}{B}\), so that \[ (v_x - \HHPIDEALPOINT{A}{U})^2 + v_y^2 = (a_x - \HHPIDEALPOINT{A}{U})^2 + a_y^2 \] and thus \(\COLLINEAR{A}{U}{V}\).
\qedhere
\end{itemize}
\end{proof}

Showing that \(\HYPHP\) is an incidence geometry is a bit tedious, but it only has to be done once, and immediately all of our theorems about incidence geometries hold.
A good question to ask is this: what are the lines in \(\HYPHP\)?
It turns out that the lines in this model come in two flavors, which we will call Type I and Type II.

\begin{cor}[Lines in \(\HYPHP\)]
Let \(A, B \in \HYPHP\) be distinct points.
\begin{proplist}
\item If \(a_x = b_x\) we say \(\LINE{A}{B}\) is of \emph{Type I}.
In this case we have \[ \LINE{A}{B} = \{ (x,y) \in \HYPHP \mid x = a_x \}. \]
\item If \(a_x \neq b_x\) we say \(\LINE{A}{B}\) is of \emph{Type II}.
In this case we have \[ \LINE{A}{B} = \left\{ (x,y) \in \HYPHP \mid (x - \HHPIDEALPOINT{A}{B})^2 + y^2 = (a_x - \HHPIDEALPOINT{A}{B})^2 + a_y^2 \right\}. \]
\end{proplist}
\end{cor}

Viewed as sets in the cartesian plane, the Type I lines in \(\HYPHP\) are vertical half lines, and the Type II lines in \(\HYPHP\) are semicircles centered on the \(x\)-axis.
In fact the cartesian center of a Type II line is the point \((\HHPIDEALPOINT{A}{B},0)\).

\begin{figure}[h]
\begin{center}
\begin{tikzpicture}
  \draw [dashed,<->] (-3,0) -- (3,0);
  \draw [->] (-2,0) -- (-2,3);
  \draw (2,0) arc (0:180:1.5);
  \node at (-3,1) {Type I};
  \node at (3,1) {Type II};
\end{tikzpicture}
\caption{\label{fig:lines-in-hyp-half-plane}Lines in \(\HYPHP\).}
\end{center}
\end{figure}



%---------%
\Exercises%
%---------%

\begin{exercise}
Prove \lemref{lem:hhp-ip-sol}.
\end{exercise}
