We will now construct an alternative collinearity relation on the unit disc.
For this example, our set of points is \( \POINCAREDISC = \{ (x,y) \in \RR^2 \mid x^2 + y^2 < 1 \}\).
Given points \(A = (a_x, a_y)\) and \(B = (b_x, b_y)\) in \(\POINCAREDISC\), we define the abbreviation \[ \PDDET{A}{B} = \DET \begin{bmatrix} a_x & a_y \\ b_x & b_y \end{bmatrix}. \]
If \(\PDDET{A}{B} \neq 0\), we define the \emph{poincare ideal point}\index{poincare ideal point} of \(A\) and \(B\) to be \[ \PDIDEALPOINT{A}{B} = \frac{1}{2} \begin{bmatrix} a_x & a_y \\ b_x & b_y \end{bmatrix}^{-1} \begin{bmatrix} a_x^2 + a_y^2 + 1 \\[4pt] b_x^2 + b_y^2 + 1 \end{bmatrix}. \]

\begin{lem}\label{lem:pd-ideal-point}
Let \(A,B \in \POINCAREDISC\) such that \(\PDDET{A}{B} \neq 0\).
Then we have the following.
\begin{proplist}
\item \label{lem:pd-ideal-point:sym} \(\PDIDEALPOINT{B}{A} = \PDIDEALPOINT{A}{B}\).
\item \label{lem:pd-ideal-point:eq} The ideal point \(\PDIDEALPOINT{A}{B}\) is the unique solution \((x,y)\) to the following system of equations.
\[ \left\{ \begin{array}{l} x^2 + y^2 - 1 = (a_x - x)^2 + (a_y - y)^2 \\[4pt] x^2 + y^2 - 1 = (b_x - x)^2 + (b_y - y)^2 \end{array} \right. \]
\end{proplist}
\end{lem}

We can use the poincare ideal point to define a collinearity relation on \(\POINCAREDISC\) as follows.

\begin{prop}
Define a ternary relation on \(\POINCAREDISC\) as follows.
Given \(A = (a_x,a_y)\), \(B = (b_x, b_y)\), and \(C = (c_x, c_y)\) in \(\POINCAREDISC\), we say that \(\COLLINEAR{A}{B}{C}\) if one of the following holds.
\begin{proplist}
\item Exactly two of \(A\), \(B\), and \(C\) are equal.
\item \(A\), \(B\), and \(C\) are distinct and \(\PDDET{A}{B} = \PDDET{A}{C} = \PDDET{B}{C} = 0\).
\item \(A\), \(B\), and \(C\) are distinct and \(\PDDET{A}{B}\), \(\PDDET{A}{C}\), and \(\PDDET{B}{C}\) are nonzero and \(\PDIDEALPOINT{A}{B} = \PDIDEALPOINT{A}{C} = \PDIDEALPOINT{B}{C}\).
\end{proplist}
This is a collinearity relation on the set \(\POINCAREDISC\), which we call the \emph{poincare disc}\index{poincare disc}.
\end{prop}

\begin{proof}\mbox{}
\begin{itemize}
\item[IG2.] Let \(A,B \in \POINCAREDISC\) be distinct.
Then \(\COLLINEAR{A}{B}{B}\) holds by definition.

\item[IG3.] Let \(A \in \POINCAREDISC\).
Note that \(\COLLINEAR{A}{A}{A}\) does not hold by definition.

\item[IG4.] Let \(A = (0,0)\), \(B = (1,0)\), and \(C = (0,1)\).
Now \(A\), \(B\), and \(C\) are distinct, but we have \(\PDDET{A}{B} = 0\) \(\PDDET{B}{C} = 1 \neq 0\).
So \(\COLLINEAR{A}{B}{C}\) does not hold.

\item[IG1.] Suppose we have \(\COLLINEAR{A}{B}{C}\).
If exactly two of \(A\), \(B\), and \(C\) are equal, then we also have \(\COLLINEAR{B}{A}{C}\) since exactly two of \(B\), \(A\), and \(C\) are equal, and \(\COLLINEAR{A}{C}{B}\) since exactly two of \(A\), \(C\), and \(B\) are equal.
Suppose then that \(A\), \(B\), and \(C\) are distinct.
Certainly then \(B\), \(A\), and \(C\) are distinct, as are \(A\), \(C\), and \(B\).
If \(\PDDET{A}{B} = \PDDET{A}{C} = \PDDET{B}{C} = 0\), then \(\PDDET{B}{A} = \PDDET{B}{C} = \PDDET{A}{C} = 0\), so that \(\COLLINEAR{B}{A}{C}\), and \(\PDDET{A}{C} = \PDDET{C}{B} = \PDDET{C}{B} = 0\) so that \(\COLLINEAR{A}{C}{B}\).
Finally, suppose we have \(\PDDET{A}{B}\), \(\PDDET{A}{C}\), and \(\PDDET{B}{C}\) nonzero and \(\PDIDEALPOINT{A}{B} = \PDIDEALPOINT{A}{C} = \PDIDEALPOINT{B}{C}\).
Then \(\PDDET{B}{A}\), \(\PDDET{B}{C}\), and \(\PDDET{A}{C}\) are nonzero and \(\PDIDEALPOINT{B}{A} = \PDIDEALPOINT{B}{C} = \PDIDEALPOINT{A}{C}\), so that \(\COLLINEAR{B}{A}{C}\), and likewise \(\PDDET{A}{C}\), \(\PDDET{A}{B}\), and \(\PDDET{C}{B}\) are nonzero and \(\PDIDEALPOINT{A}{C} = \PDIDEALPOINT{A}{B} = \PDIDEALPOINT{C}{B}\), so that \(\COLLINEAR{A}{C}{B}\) as claimed.

\item[IG5.] Suppose we have \(A \neq B\) and \(U \neq V\) such that \(\COLLINEAR{A}{B}{U}\) and \(\COLLINEAR{A}{B}{V}\).
If \(U = B\), then we have \(\COLLINEAR{A}{U}{V}\); similarly, if \(V = B\) then \(\COLLINEAR{A}{V}{U}\), so that by IG1, \(\COLLINEAR{A}{U}{V}\).
Suppose then that \(U \neq B\) and \(V \neq B\).
If \(U = A\), then \(\COLLINEAR{A}{U}{V}\) by definition; similarly, if \(V = A\) then \(\COLLINEAR{A}{U}{V}\).
Suppose then that \(U \neq A\) and \(V \neq A\).
Now \(A\), \(B\), \(U\), and \(V\) are all distinct.
There are two possibilities for \(\PDDET{A}{B}\): either zero or nonzero.

If \(\PDDET{A}{B} = 0\), then in fact we have \(\PDDET{A}{U} = \PDDET{B}{U} = 0\) and \(\PDDET{A}{V} = \PDDET{B}{V} = 0\).
Now since \(A \neq B\), one of \(A\) or \(B\) must be nonzero; suppose without loss of generality it is \(A\).
Then either \(a_x\) or \(a_y\) must be nonzero; suppose it is \(a_x\) -- the argument for \(a_y\) is similar.
We have \(a_xu_y = a_yu_x\), so that \(v_xa_xu_y = v_xa_yu_x\), and thus \(v_xa_xu_y = v_ya_xu_x\).
Since \(a_x \neq 0\), we have \(v_xu_y = v_yu_x\), so that \(\PDDET{U}{V} = 0\).
Thus \(\COLLINEAR{A}{U}{V}\) as needed.

Suppose instead that \(\PDDET{A}{B} \neq 0\).
Then \(\PDDET{A}{U}\), \(\PDDET{B}{U}\), \(\PDDET{A}{V}\), and \(\PDDET{B}{V}\) are all nonzero, and we have \(\PDIDEALPOINT{A}{B} = \PDIDEALPOINT{A}{U} = \PDIDEALPOINT{B}{U} = \PDIDEALPOINT{A}{V} = \PDIDEALPOINT{B}{V}\).
Suppose first that \(\PDDET{U}{V} = 0\).
In this case, we have \(V = kU\) for some constant \(k\), since \(U \neq 0\).
Now we have \[ (u_x - x)^2 + (u_y - y)^2 = (ku_x - x)^2 + (ku_y - y)^2. \]
Expanding this equation and supposing \(k \neq 1\) we solve for \(k\) as \[ k = 2\frac{u_xx + u_yy}{u_x^2 + u_y^2} - 1. \]
But note that \(2(u_xx + u_yy) = u_x^2 + u_y^2 + 1\), so that in fact \(k = 1/(u_x^2 + u_y^2)\).
But now \[v_x^2 + v_y^2 = 1/(u_x^2 + u_y^2) > 1,\] a contradiction.
So in fact we have \(k = 1\), and thus \(V = U\), a contradiction.
So we must have \(\PDDET{U}{V}\).
Now by \sref{lem:pd-ideal-point}{eq}, we have \(\PDIDEALPOINT{U}{V} = \PDIDEALPOINT{A}{U} = \PDIDEALPOINT{A}{V}\), and thus \(\COLLINEAR{A}{U}{V}\) as needed.
\qedhere
\end{itemize}
\end{proof}

Much like the hyperbolic half plane, showing that the poincare disc is an incidence geometry is a bit tedious.
But this only has to be done once, and all our incidence geometry theorems immediately apply.

\begin{prop}[Lines in \(\POINCAREDISC\)]
Let \(A,B \in \POINCAREDISC\) be distinct points, and without loss of generality suppose \(A \neq 0\).
\begin{proplist}
\item If \(\PDDET{A}{B} = 0\), we say that \(\LINE{A}{B}\) is of \emph{Type I}.
In this case we have \[ \LINE{A}{B} = \left\{ (x,y) \in \POINCAREDISC \mid \DET \begin{bmatrix} a_x & a_y \\ x & y \end{bmatrix} = 0 \right\} \]
\item If \(\PDDET{A}{B} \neq 0\), we say that \(\LINE{A}{B}\) is of \emph{Type II}.
In this case we have \[ \LINE{A}{B} = \left\{ (x,y) \in \POINCAREDISC \mid (@@@) \right\} \]
\end{proplist}
\end{prop}
