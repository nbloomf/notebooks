In an incidence geometry we have the ability to detect whether three given points are collinear.
However an arbitrary incidence geometry has no notion of ``order'' for the points on a given line, as we might intuitively expect.
For instance, the points \((0,0)\), \((1,1)\), and \((2,2)\) are collinear in \(\RR^2\) and we think of \((1,1)\) as being ``between'' the other two.
But in the Fano plane, does it make sense to order the points on a line?
Presently we introduce another piece of technology to an incidence geometry which will allow us to formalize the concept of ``betweenness''.

\begin{dfn}[Betweenness]
Let \(P\) be an incidence geometry.
We say that a ternary relation \(\BETWEEN{\ast}{\ast}{\ast}\) on \(P\) is a \emph{betweenness relation}\index{betweenness} if the following properties hold.
\begin{proplist}
\item[B1.] If \(\BETWEEN{x}{y}{z}\), then \(x\), \(y\), and \(z\) are distinct and \(\COLLINEAR{x}{y}{z}\).
\item[B2.] If \(x\) and \(y\) are points such that \(\BETWEEN{x}{z}{y}\), then \(\BETWEEN{y}{z}{x}\).
\item[B3.] If \(x\), \(y\), and \(z\) are distinct points such that \(\COLLINEAR{x}{y}{z}\), then at least one of \(\BETWEEN{x}{y}{z}\), \(\BETWEEN{y}{z}{x}\), and \(\BETWEEN{z}{x}{y}\) is true.
\item[B4.] If \(\BETWEEN{x}{y}{z}\) and \(\BETWEEN{x}{z}{w}\), then \(\BETWEEN{x}{y}{w}\) and \(\BETWEEN{y}{z}{w}\).
\item[B5.] If \(\BETWEEN{x}{y}{z}\) and \(\BETWEEN{y}{z}{w}\), then \(\BETWEEN{x}{y}{w}\) and \(\BETWEEN{x}{z}{w}\).
\item[B6.] \textbf{Interpolation Property:} If \(x\) and \(y\) are distinct points, then there exist points \(z_1\), \(z_2\), and \(z_3\) such that \(\BETWEEN{z_1}{x}{y}\), \(\BETWEEN{x}{z_2}{y}\), and \(\BETWEEN{x}{y}{z_3}\).
\end{proplist}
\end{dfn}

If \(\BETWEEN{x}{y}{z}\), we say that \(z\) is \emph{between} \(x\) and \(y\).
As shorthand, if \(x\), \(y\), \(z\), and \(w\) are distinct points, we will say \([xyzw]\) precisely when \(\BETWEEN{x}{y}{z}\), \(\BETWEEN{x}{y}{w}\), \(\BETWEEN{x}{z}{w}\), and \(\BETWEEN{y}{z}{w}\).
More generally, if \(x_1, \ldots, x_n\) are distinct points, then \([x_1x_2 \ldots x_n]\) means that \(\BETWEEN{x_i}{x_j}{x_k}\) for all triples \((i,j,k)\) with \(1 \leq i < j < k \leq n\).
The next definition will probably look familiar.

\begin{dfn}[Segment, Ray]
Let \(x\) and \(y\) be distinct points in an ordered geometry \(P\).
\begin{proplist}
\item The set \[ \SEGMENT{x}{y} = \{ z \in P \mid z = x\ \mathrm{or}\ z = y\ \mathrm{or}\ \BETWEEN{x}{z}{y} \} \] is called the \emph{segment}\index{segment} with \emph{endpoints} \(x\) and \(y\).
If \(z \in \SEGMENT{x}{y}\) and \(z \neq x\) and \(z \neq y\), we say that \(z\) is \emph{interior to} \(\SEGMENT{x}{y}\).
\item The set \[ \RAY{x}{y} = \{ z \in P \mid z = x\ \mathrm{or}\ z = y\ \mathrm{or}\ \BETWEEN{x}{z}{y}\ \mathrm{or}\ \BETWEEN{x}{y}{z} \} \] is called the \emph{ray}\index{ray} with \emph{vertex} \(x\) \emph{toward} \(y\).
\end{proplist}
\end{dfn}

The following properties of segments and rays essentially fall out of the definition.

\begin{prop}
If \(P\) is an incidence geometry and \(\BETWEEN{\cdot}{\cdot}{\cdot}\) a betweenness relation on \(P\), then the following hold.
\begin{enumerate}
\item \(\SEGMENT{x}{y} = \SEGMENT{y}{x}\) for all distinct points \(x\) and \(y\).
\item \(\SEGMENT{x}{y} \subseteq \RAY{x}{y} \subseteq \LINE{x}{y}\) for all distinct points \(x\) and \(y\).
\item If \(\ell\) is a line and \(x\) and \(y\) distinct points, then \(\SEGMENT{x}{y} \cap \ell\) is either \(\SEGMENT{x}{y}\), \(\varnothing\), or \(\{p\}\) for some point \(p\).
\item \(\RAY{x}{y} \cap \RAY{y}{x} = \SEGMENT{x}{y}\) for all distinct points \(x\) and \(y\).
\end{enumerate}
\end{prop}



\begin{prop}[Line Decomposition]\label{prop:line-decomp}
Suppose \(P\) is an incidence geometry with a betweenness relation, and let \(x,y \in P\) be distinct points.
Then \[ \LINE{x}{y} = \{ z \mid z = x\ \mathrm{or}\ z = y\ \mathrm{or}\ \BETWEEN{z}{x}{y}\ \mathrm{or}\ \BETWEEN{x}{z}{y}\ \mathrm{or}\ \BETWEEN{x}{y}{z} \}. \]
\end{prop}

Proposition \ref{prop:line-decomp} is a useful technical result: if we know that a given point \(z\) lies on a line \(\ell_{x,y}\), then there are five possibilities.
The next result is useful for the same reason.

\begin{prop}
Let \(P\) be an incidence geometry with a betweenness relation and suppose \(x\), \(y\), \(z\), and \(w\) are points.
If \(\BETWEEN{x}{z}{y}\) and \(\BETWEEN{x}{w}{y}\), then either \(\BETWEEN{x}{z}{w}\) or \(\BETWEEN{x}{w}{z}\) or \(z = w\).
\end{prop}

We can now define \emph{convexity} in terms of betweenness as follows.

\begin{dfn}[Convexity]
Let \(P\) be an incidence geometry with a betweenness relation.
A non empty set \(S\) of points in \(P\) is called \emph{convex}\index{convex} if it is closed under betweenness in the following sense: if \(x,y \in S\) and \(\BETWEEN{x}{z}{y}\), then \(z \in S\).
\end{dfn}



%---------%
\Exercises%
%---------%

\begin{exercise}
Suppose \(P\) is an incidence geometry with a betweenness relation.
If \(x\), \(y\), and \(z\) are distinct points such that \(\BETWEEN{x}{y}{z}\), then the following hold.
\begin{proplist}
\item \(\SEGMENT{x}{y} \cup \SEGMENT{y}{z} = \SEGMENT{x}{z}\).
\item \(\SEGMENT{x}{y} \cap \SEGMENT{y}{z} = \{y\}\).
\item \(\RAY{y}{x} \cap \RAY{y}{z} = \{y\}\).
\item \(\RAY{x}{y} = \RAY{x}{z}\).
\item \(\RAY{x}{y} \cap \RAY{y}{x} = \SEGMENT{x}{y}\).
\item \(\RAY{y}{x} \cup \RAY{y}{z} = \LINE{x}{z}\).
\end{proplist}
\end{exercise}

\begin{exercise}
Let \(P\) be an incidence geometry with a betweenness relation, and let \(S \subseteq P\).
Show that \(S\) is convex if and only if \(\SEGMENT{x}{y} \subseteq S\) for all distinct points \(x,y \in S\).
\end{exercise}

\begin{exercise}
Let \(P\) be an incidence geometry with a betweenness relation and \(x,y \in P\) distinct points.
Show that the following sets are convex.
\begin{proplist}
\item \(\LINE{x}{y}\)
\item \(\SEGMENT{x}{y}\)
\item \(\RAY{x}{y}\)
\end{proplist}
\end{exercise}

\begin{exercise}
Let \(P\) be an incidence geometry with a betweenness relation.
Show that every line in \(P\) contains infinitely many points.
\end{exercise}
