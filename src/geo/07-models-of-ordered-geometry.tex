In this section we will establish some models of ordered geometry.
Remember that to show a given incidence geometry is also an ordered geometry, we need to specify (1) how to detect when one point is between two others and (2) how to determine when two points are on the same side of a line.

\subsection{Betweenness in \(\RR^2\)}

\begin{prop}\label{prop:rr2-between}
Given points \(A\), \(B\), and \(C\) in \(\RR^2\), we say \(\BETWEEN{A}{C}{B}\) if \(A \neq B\) and the equation \(C = A + t(B-A)\) has a solution \(t \in (0,1)\).
Then this \(\BETWEEN{\ast}{\ast}{\ast}\) is a betweenness relation on \(\RR^2\).
\end{prop}

\begin{proof}\mbox{}
\begin{itemize}
\item[B1.] Suppose \(\BETWEEN{A}{C}{B}\).
Now \(A \neq B\) by definition, and we have \(C = A + t(B-A)\) for some \(t \in (0,1)\).
If \(C = A\), then we have \((0,0) = t(B-A)\), so that \(B = A\) -- a contradiction.
So \(C \neq A\).
Similarly, if \(C = B\), then \((0,0) = (t-1)(B-A)\), so that \(B = A\) -- a contradiction.
So \(C \neq B\).
Finally we claim that \(A\), \(B\), and \(C\) are collinear.
To this end, note that \[ \DET \begin{bmatrix} a_x & a_y & 1 \\ b_x & b_y & 1 \\ c_x & c_y & 1 \end{bmatrix} = \DET \begin{bmatrix} a_x & a_y & 1 \\ b_x & b_y & 1 \\ a_x + t(b_x-a_x) & a_y + t(b_x-a_x) & 1 \end{bmatrix} = 0. \]

\item[B2.] If \(C = A + t(B-A)\) where \(t \in (0,1)\), then (rearranging) we also have \(C = B + (1-t)(A-B)\) with \(1-t \in (0,1)\) as needed.

\item[B3.] Suppose \(A\), \(B\), and \(C\) are distinct such that \(\COLLINEAR{A}{B}{C}\).
Using Exercise \ref{exerc:rr2-collinear-comb}, we have \(C = A + t(B-A)\) for some real number \(t\).
Note that \(t \neq 0\) and \(t \neq 1\) since in the first case we would have \(C = A\) and in the second, \(C = B\).
There are then three possibilities for \(t\).
If \(t \in (0,1)\), then \(\BETWEEN{A}{C}{B}\) by definition.
If \(t > 1\), then ``solving for \(B\)'' we have \(B = A + \frac{1}{t}(C-A)\), and since \(1/t \in (0,1)\), we have \(\BETWEEN{A}{B}{C}\).
If \(t < 0\), then we have \(A = C + \frac{-t}{1-t}(B-C)\), and since \(\frac{-t}{1-t} \in (0,1)\), we have \(\BETWEEN{C}{A}{B}\).

\item[B4.] Suppose \(\BETWEEN{A}{B}{C}\) and \(\BETWEEN{A}{C}{D}\); say we have \(B = A + t(C-A)\) and \(C = A + s(D-A)\) where \(s,t \in (0,1)\).
Now \(B = A + ts(D-A)\), and since \(st \in (0,1)\), we have \(\BETWEEN{A}{B}{D}\).
Similarly, we have \(C = B + \frac{s-ts}{1-ts}(D-B)\), and since \(\frac{s-ts}{1-ts} \in (0,1)\), we have \(\BETWEEN{B}{C}{D}\) as needed.

\item[B5.] Suppose \(\BETWEEN{A}{B}{C}\) and \(\BETWEEN{B}{C}{D}\); say we have \(B = t(C-A)\) and \(C = B + s(D-B)\) where \(s,t \in (0,1)\).
Now \(C = A + \frac{s}{1-t+st}(D-A)\), and we also have \(\frac{s}{1-t+st} \in (0,1)\).
(To see this, note that \(0 < (1-s)(1-t)\) and rearrange to get \(s < 1-t+st\).)
Thus \(\BETWEEN{B}{C}{D}\).
Next note that \(B = A + \frac{ts}{1-t+ts}(D-A)\), and since \(\frac{ts}{1-t+ts} \in (0,1)\), we have \(\BETWEEN{A}{B}{D}\) as needed.

\item[B6.] Let \(A\) and \(B\) be distinct points.
Given a real number \(t\), let \(C = A + t(B-A)\).
If \(t \in (0,1)\), then \(\BETWEEN{A}{C}{B}\) by definition.
If \(t > 1\), then \(B = A + \frac{1}{t}(C-A)\) with \(\frac{1}{t} \in (0,1)\) and we have \(\BETWEEN{A}{B}{C}\).
If \(t < 0\), then \(A = C + \frac{-t}{1-t}(B-C)\) with \(\frac{-t}{1-t} \in (0,1)\) and we have \(\BETWEEN{C}{A}{B}\).
\qedhere
\end{itemize}
\end{proof}

Next, we'd like to show that \(\RR^2\) is an ordered geometry by showing that it has the Line Separation property.
First, we need the following technical lemma about the intersection of a segment and a line in \(\RR^2\).

\begin{lem}
Let \(A,B \in \RR^2\) be distinct points and let \(X = (x_1, x_2), Y = (y_1, y_2) \in \RR^2\) be distinct points not in \(\ell_{A,B}\).
Then \(\SEGMENT{X}{Y} \cap \LINE{A}{B}\) consists of a single point if and only if \[ \DET \begin{bmatrix} a_1 & a_2 & 1 \\ b_1 & b_2 & 1 \\ x_1 & x_2 & 1 \end{bmatrix} \quad \mathrm{and} \quad \DET \begin{bmatrix} a_1 & a_2 & 1 \\ b_1 & b_2 & 1 \\ y_1 & y_2 & 1 \end{bmatrix} \] have opposite signs.
\end{lem}

\begin{proof}
Note that \(\LINE{X}{Y} \cap \LINE{A}{B}\) contains exactly one point if and only if the equation \(X + t(Y-X) = A + u(B-A)\) has a unique solution \((t,u)\).
In fact, by comparing coordinates we can rewrite this equation in matrix form as \[ \begin{bmatrix} t \\ -u \end{bmatrix} = \begin{bmatrix} y_1 - x_1 & b_1 - a_1 \\ y_2 - x_2 & b_2 - a_2 \end{bmatrix}^{-1} \begin{bmatrix} a_1 - x_1 \\ a_2 - x_2 \end{bmatrix}. \]
Comparing entries of this matrix, we see that \[ t = \frac{(b_2-a_2)(a_1-x_1) - (b_1-a_1)(a_2-x_2)}{(y_1-x_1)(b_2-a_2) - (y_2-x_2)(b_1-a_1)}; \] note that the denominator of this expression is not zero by Exercise \ref{exerc:parallels-in-rr2}.
Now by definition the unique point in \(\LINE{X}{Y} \cap \LINE{A}{B}\) is more specifically in the segment \(\SEGMENT{X}{Y}\) if and only if \(t \in (0,1)\).

There are now two possibilities, depending on whether the denominator of \(t\) is positive or negative.
If the denominator of \(t\) is positive, we can see that \(t \in (0,1)\) if and only if \[ \DET \begin{bmatrix} a_1 & a_2 & 1 \\ b_1 & b_2 & 1 \\ x_1 & x_2 & 1 \end{bmatrix} > 0 > \DET \begin{bmatrix} a_1 & a_2 & 1 \\ b_1 & b_2 & 1 \\ y_1 & y_2 & 1 \end{bmatrix}, \]
and if the denominator of \(t\) is negative, then \(t \in (0,1)\) if and only if \[ \DET \begin{bmatrix} a_1 & a_2 & 1 \\ b_1 & b_2 & 1 \\ x_1 & x_2 & 1 \end{bmatrix} < 0 < \DET \begin{bmatrix} a_1 & a_2 & 1 \\ b_1 & b_2 & 1 \\ y_1 & y_2 & 1 \end{bmatrix} \] as needed.
\end{proof}

We are now prepared to show the following.

\begin{prop}\label{prop:rr2-line-sep}
Given a line \(\ell = \LINE{A}{B}\) in \(\RR^2\), we define two half-planes as follows: \[ H_1 = \left\{ (x, y) \mid \DET \begin{bmatrix} a_x & a_y & 1 \\ b_x & b_y & 1 \\ x & y & 1 \end{bmatrix} > 0 \right\} \] and \[ H_2 = \left\{ (x, y) \mid \DET \begin{bmatrix} a_x & a_y & 1 \\ b_x & b_y & 1 \\ x & y & 1 \end{bmatrix} < 0 \right\}. \]
With half planes defined in this way for all lines, \(\RR^2\) is an ordered geometry.
\end{prop}

\begin{proof}
It suffices to show that the Line Separation property is satisfied.
Certainly both \(H_1\) and \(H_2\) are not empty, and they are disjoint by construction.
By the previous lemma, if \(X \in H_1\) and \(Y \in H_2\), then \(\SEGMENT{X}{Y} \cap \LINE{A}{B}\) consists of a unique point.
So it suffices to show that \(H_1\) and \(H_2\) are convex.

To see that \(H_1\) is convex, suppose BWOC that we have points \(X,Y \in H_1\) and a point \(Z = (z_1, z_2)\) such that \(\BETWEEN{X}{Z}{Y}\) and \(Z \notin H_1\).
Now \[ m = \DET \begin{bmatrix} a_1 & a_2 & 1 \\ b_1 & b_2 & 1 \\ z_1 & z_2 & 1 \end{bmatrix} \] is either 0 or negative.
If \(m = 0\), then in fact \(Z \in \LINE{A}{B}\).
Since \(X,Y \notin \LINE{A}{B}\), we have that \(\SEGMENT{X}{Y}\) and \(\LINE{A}{B}\) meet at a single point \(Z\); but we've seen this can only happen if \(X \in H_1\) and \(Y \in H_2\) (or vice versa).
Suppose instead that \(m < 0\); that is, \(Z \in H_2\).
Now we have that \(\SEGMENT{X}{Z}\) and \(\SEGMENT{Y}{Z}\) each intersect \(\LINE{A}{B}\) at unique points, say \(W\) and \(V\), respectively.
Note that \(\BETWEEN{X}{W}{Z}\) and \(\BETWEEN{Y}{V}{Z}\).
Since \(\BETWEEN{X}{Z}{Y}\), we have that \(X\), \(Y\), \(Z\), \(W\), and \(V\) are all collinear.
If \(W\) and \(V\) are distinct points, then in fact \(X, Y \in \LINE{W}{V} = \LINE{A}{B}\), a contradiction.
If \(W = V\), then we have \(\BETWEEN{X}{W}{Z}\) and \(\BETWEEN{Y}{W}{Z}\), so by the 4-point axiom, \(\BETWEEN{W}{Z}{Y}\), a contradiction.
So we must have \(Z \in H_1\), and thus \(H_1\) is convex.
A similar argument shows that \(H_2\) is convex.
\end{proof}

The proofs of Propositions \ref{prop:rr2-between} and \ref{prop:rr2-line-sep} remain valid if we replace \(\RR^2\) by \(\QQ^2\), so that the Rational Plane is also an ordered geometry.
However we cannot replace \(\RR^2\) by \(\CC^2\), because the order relation \(<\) does not make sense in the complex numbers.


\subsection{Betweenness in \(\mathbb{D}\)}

The Unit Disc inherits a natural betweenness relation from the Cartesian Plane.
Given points \(A\), \(B\), and \(C\) in the disc, say that \(\BETWEEN{A}{B}{C}\) if \(B\) is between \(A\) and \(C\) in \(\RR^2\).
Similarly, if \(A \neq B\) and \(\ell\) is the line generated by \(A\) and \(B\) in \(RR^2\), then the half-planes cut by \(\LINE{A}{B}\) in \(\mathbb{D}\) are \(H_1 \cap \mathbb{D}\) and \(H_2 \cap \mathbb{D}\), where \(H_1\) and \(H_2\) are the half-planes cut by \(\ell\) in \(\RR^2\).


\subsection{Betweenness in the Fano Plane}

Note that as a consequence of the Interpolation property, every line in an ordered geometry must contain infinitely many points.
As a consequence, the Fano Plane cannot possibly be an ordered geometry -- it has only 7 points.
