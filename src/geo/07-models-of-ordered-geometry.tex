In this section we will establish some models of ordered geometry.
Remember that to show a given incidence geometry is also an ordered geometry, we need to specify (1) how to detect when one point is between two others and (2) how to determine when two points are on the same side of a line.

\subsection{Betweenness in \(\mathbb{D}\)}

The Unit Disc inherits a natural betweenness relation from the Cartesian Plane.
Given points \(A\), \(B\), and \(C\) in the disc, say that \(\BETWEEN{A}{B}{C}\) if \(B\) is between \(A\) and \(C\) in \(\RR^2\).
Similarly, if \(A \neq B\) and \(\ell\) is the line generated by \(A\) and \(B\) in \(RR^2\), then the half-planes cut by \(\LINE{A}{B}\) in \(\mathbb{D}\) are \(H_1 \cap \mathbb{D}\) and \(H_2 \cap \mathbb{D}\), where \(H_1\) and \(H_2\) are the half-planes cut by \(\ell\) in \(\RR^2\).


\subsection{Betweenness in the Fano Plane}

Note that as a consequence of the Interpolation property, every line in an ordered geometry must contain infinitely many points.
As a consequence, the Fano Plane cannot possibly be an ordered geometry -- it has only 7 points.
