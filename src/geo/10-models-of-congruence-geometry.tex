Remember: to show that an ordered geometry is a congruence geometry, we need to specify (1) how to detect when two segments are congruent and (2) how to detect when two angles are congruent.

\subsection{Congruence in $\RR^2$}

\begin{prop}
Define a map $\delta : \RR^2 \times \RR^2 \rightarrow \RR$ by $\delta(A,B) = (B-A) \cdot (B-A)$, where $\cdot$ denotes the usual dot product. Now $\RR^2$ is a congruence geometry under the following relations.
\begin{proplist}
\item Given points $A$, $B$, $X$, and $Y$ in $\RR^2$, we say that $\SEGCONG{A}{B}{X}{Y}$ if \[ \delta(A,B) = \delta(X,Y). \]
\item Given points $A$, $O$, $B$, $X$, $P$, and $Y$ in $\RR^2$ such that $A \neq O$, $B \neq O$, $X \neq P$, and $Y \neq P$, we say that $\ANGCONG{A}{O}{B}{X}{P}{Y}$ if \[ \frac{((A-O) \cdot (B-O))^2}{\delta(A,O)\delta(B,O)} = \frac{((X-P) \cdot (Y-P))^2}{\delta(X,P)\delta(Y,P)}. \]
\end{proplist}
\end{prop}

\begin{proof}
(@@@)
\end{proof}



\subsection{Congruence in the Unit Disc}


