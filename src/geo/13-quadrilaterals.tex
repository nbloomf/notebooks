\documentclass{article}
\usepackage{neb-macros}
\usepackage{tikz}
  \usetikzlibrary{calc,intersections}

\begin{document}

\CheapTitle{Quadrilaterals}

\begin{dfn}[Quadrilateral]
Let $A$, $B$, $C$, and $D$ be points in a plane geometry. Then the set \[ \Quadrilateral{A}{B}{C}{D} = \Segment{A}{B} \cup \Segment{B}{C} \cup \Segment{C}{D} \cup \Segment{D}{A} \] is called the \emph{quadrilateral} with \emph{vertices} $A$, $B$, $C$, and $D$ (in that order).
\begin{itemize}
\item $\Segment{A}{B}$, $\Segment{B}{C}$, $\Segment{C}{D}$, and $\Segment{D}{A}$ are called the \emph{sides} of the quadrilateral.
\item $\Segment{A}{C}$ and $\Segment{B}{D}$ are called the \emph{diagonals}.
\item $\Angle{A}{B}{C}$, $\Angle{B}{C}{D}$, $\Angle{C}{D}{A}$, and $\Angle{D}{A}{B}$ are the \emph{interior angles}
\item Sides $\Segment{A}{B}$ and $\Segment{C}{D}$ are said to be \emph{opposite}, as are $\Segment{B}{C}$ and $\Segment{D}{A}$.
\end{itemize}
\end{dfn}

\begin{dfn}
Let $A$, $B$, $C$, and $D$ be points.
\begin{itemize}
\item If any two of $A$, $B$, $C$, and $D$ are equal, then $\Quadrilateral{A}{B}{C}{D}$ is said to be \emph{backtracking}.
\item $\Quadrilateral{A}{B}{C}{D}$ is called \emph{self-intersecting} if any point other than the vertices is on more than one side.
\item $\Quadrilateral{A}{B}{C}{D}$ is called \emph{degenerate} if any three of its vertices are collinear.
\item A quadrilateral which is not backtracking, self-intersecting, or degenerate is called \emph{simple}.
\end{itemize}
\end{dfn}

\begin{dfn}
We say that $\Quadrilateral{A}{B}{C}{D} \equiv \Quadrilateral{X}{Y}{Z}{W}$ if $\Segment{A}{B} \equiv \Segment{X}{Y}$, $\Segment{B}{C} \equiv \Segment{Y}{Z}$, $\Segment{C}{D} \equiv \Segment{Z}{W}$, $\Segment{D}{A} \equiv \Segment{W}{X}$, $\Angle{A}{B}{C} \equiv \Angle{X}{Y}{Z}$, $\Angle{B}{C}{D} \equiv \Angle{Y}{Z}{W}$, $\Angle{C}{D}{A} \equiv \Angle{Z}{W}{X}$, and $\Angle{D}{A}{B} \equiv \Angle{W}{X}{Y}$.
\end{dfn}

\begin{prop}
Quadrilateral congruence is an equivalence relation.
\end{prop}

\subsection*{Types of Quadrilaterals}

\begin{dfn}
A quadrilateral $\Quadrilateral{A}{B}{C}{D}$ is called
\begin{itemize}
\item \emph{equiangular} if all its interior angles are congruent;
\item a \emph{rectangle} if all its interior angles are right;
\item a \emph{kite} if it has two pairs of congruent adjacent sides;
\item \emph{equilateral} (a.k.a. a \emph{rhombus}) if all its sides are congruent;
\item a \emph{trapezoid} if one pair of opposite sides is parallel;
\item a \emph{parallelogram} if both pairs of opposite sides are parallel;
\item \emph{cyclic} if all its vertices lie on a common circle;
\item \emph{tangential} if all extended sides are tangent to a common circle;
\item \emph{regular} if it is both equilateral and equiangular.
\end{itemize}
\end{dfn}

Note: not all of these types of quadrilaterals are guaranteed to exist!

\end{document}
