The existence of a betweenness relation on an incidence geometry says something very strong.
Combined with an additional property -- the Line Separation property -- such geometries are called \emph{ordered}.

\begin{dfn}[Ordered Geometry]
Let \(P\) be an incidence geometry with a betweenness relation \(\BETWEEN{\ast}{\ast}{\ast}\).
We say that \(P\) is an \emph{ordered geometry} if it satisfies the following additional \emph{Line Separation Property}.
\begin{itemize}
\item[LS.] For every line \(\ell\) there are two sets, \(\HALFPLANEA{\ell}\) and \(\HALFPLANEB{\ell}\), which satisfy the following properties.
\begin{proplist}
\item \(\HALFPLANEA{\ell}\) and \(\HALFPLANEB{\ell}\) are not empty.
\item \(\ell\), \(\HALFPLANEA{\ell}\), and \(\HALFPLANEB{\ell}\) partition \(P\).
\item \(\HALFPLANEA{\ell}\) and \(\HALFPLANEB{\ell}\) are convex.
\item If \(x \in \HALFPLANEA{\ell}\) and \(y \in \HALFPLANEB{\ell}\), then \(\SEGMENT{x}{y} \cap \ell = \{p\}\) for some point \(p\).
\end{proplist}
The sets \(\HALFPLANEA{\ell}\) and \(\HALFPLANEB{\ell}\) are called \emph{halfplanes}\index{halfplane}.
\end{itemize}
\end{dfn}

The Line Separation property essentially states that every line divides the plane into two ``separate'' pieces -- we might call these pieces the \emph{sides} of the line.
Given a line \(\ell\) and points \(x,y \notin \ell\), we say that \(x\) and \(y\) are on the \emph{same side} of \(\ell\) if they are both in the same halfplane bounded by \(\ell\), and otherwise they are on \emph{opposite sides}.
See \autoref{fig:line-sep} for a picture.
(But remember that a given ordered geometry might look very different!)

\begin{figure}[h]
\begin{center}
\begin{tikzpicture}
  \draw [<->] (0,0) -- (2,4);
  \coordinate [label=below right:\(\ell\)] (x) at (1,2);
  \node at (-1,2) {\(\HALFPLANEA{\ell}\)};
  \node at (2.5,2) {\(\HALFPLANEB{\ell}\)};
\end{tikzpicture}
\caption{\label{fig:line-sep}The halfplanes bounded by a line.}
\end{center}
\end{figure}

\begin{dfn}[Triangle]
Let \(P\) be an incidence geometry, and let \(x\), \(y\), and \(z\) be points.
Then the set \[ \TRIANGLE{x}{y}{z} = \SEGMENT{x}{y} \cup \SEGMENT{y}{z} \cup \SEGMENT{z}{x} \] is called the \emph{triangle} with \emph{vertices} \(x\), \(y\), and \(z\).
The segments \(\SEGMENT{x}{y}\), \(\SEGMENT{y}{z}\), and \(\SEGMENT{z}{x}\) are called the \emph{sides}\index{sides (of a triangle)} of the triangle, and the lines \(\LINE{x}{y}\), \(\LINE{y}{z}\), and \(\LINE{z}{x}\) are called the \emph{extended sides}\index{extended sides (of a triangle)}.
If \(x\), \(y\), and \(z\) are not all distinct, we say that the triangle \(\TRIANGLE{x}{y}{z}\) is \emph{degenerate}\index{degenerate!triangle}.
\end{dfn}

The next result seems intuitive, but must be proven; it states that if a line ``enters'' a triangle through one side and does not contain any of the triangle's vertices, then it must ``exit'' the triangle through one of the other sides.
This is called \emph{Pasch's Axiom} for historical reasons.

\begin{prop}[Pasch's Axiom]
Let \(x\), \(y\), and \(z\) be distinct noncollinear points in an ordered geometry, and let \(\ell\) be a line such that \(x,y,z \notin \ell\).
Finally, suppose there is a point \(w \in \ell\) such that \(\BETWEEN{x}{w}{y}\); that is, \(\ell\) cuts the side \(\SEGMENT{x}{y}\).
Then precisely one of the following two things happens:
\begin{proplist}
\item \(\ell\) cuts \(\SEGMENT{y}{z}\) and does not cut \(\SEGMENT{z}{x}\), or
\item \(\ell\) cuts \(\SEGMENT{z}{x}\) and does not cut \(\SEGMENT{y}{z}\).
\end{proplist}
\end{prop}

\begin{proof}
Since \(P\) is an ordered geometry, it satisfies the Plane Separation property.
In particular, the points not on \(\ell\) are partitioned into two convex, nonempty half-planes, \(\HALFPLANEA{\ell}\) and \(\HALFPLANEB{\ell}\).
Since \(x\) and \(y\) are not on \(\ell\), without loss of generality we have \(x \in \HALFPLANEA{\ell}\) and \(y \in \HALFPLANEB{\ell}\).
Since \(z \notin \ell\), there are two possibilities: either \(z \in \HALFPLANEA{\ell}\) or \(z \in \HALFPLANEB{\ell}\).
In the first case, we see that \(\ell\) cuts \(\SEGMENT{y}{z}\) and does not cut \(\SEGMENT{z}{x}\), and in the second case, \(\ell\) cuts \(\SEGMENT{z}{x}\) but not \(\SEGMENT{y}{z}\).
\end{proof}

In other words, Pasch's Axiom states that if a line enters a triangle then it must also exit; see Figure \ref{fig:pasch}.

\begin{figure}[h]
\begin{center}
\begin{tikzpicture}[scale=1.1]
  \coordinate [label=below:\(x\)] (X) at (0,0);
  \coordinate [label=above left:\(y\)] (Y) at (-1,2);
  \coordinate [label=right:\(z\)] (Z) at (4,1);
  \coordinate (H) at (0,1);
  \coordinate (K) at (-2,2);
  \draw (X) -- (Y) -- (Z) -- cycle;
  \draw [->] (H) -- (K);
  \coordinate [label=below left:\(w\)] (W) at (intersection of X--Y and H--K);
  \draw [->,dashed] (H) -- (1,0.5);
  \draw [fill] (X) circle [radius=0.7pt];
  \draw [fill] (Y) circle [radius=0.7pt];
  \draw [fill] (Z) circle [radius=0.7pt];
  \draw [fill] (W) circle [radius=0.7pt];
\end{tikzpicture}
\caption{\label{fig:pasch}The setup of Pasch's Axiom}
\end{center}
\end{figure}

\begin{lem}\label{lem:betweenness-separation}
Let \(\ell\) be a line and \(C \in \ell\) a point in an ordered geometry.
Suppose \(A\) and \(B\) are points not on \(\ell\) such that \(\BETWEEN{A}{B}{C}\).
Then \(A\) and \(B\) are on the same side of \(\ell\).
\end{lem}

\begin{proof}
Suppose otherwise that \(A\) and \(B\) are on opposite sides of \(\ell\).
By the Line Separation property, and because \(A\) and \(B\) are not on \(\ell\), the segment \(\SEGMENT{A}{B}\) cuts \(\ell\) at a unique point \(D\).
That is, \(D \in \ell\) and \(\BETWEEN{A}{D}{B}\).
In particular, \(D\) and \(C\) must be distinct since we have \(\BETWEEN{D}{B}{C}\).
But note that \(C, D \in \ell\), so \(\LINE{C}{D} = \ell\), and also \(C, D \in \LINE{A}{B}\), so that \(\LINE{C}{D} = \LINE{A}{B}\).
But then \(\LINE{A}{B} = \ell\), a contradiction.
Thus \(A\) and \(B\) must be on the same side of \(\ell\).
\end{proof}



%---------%
\Exercises%
%---------%

\begin{exercise}
Let \(a\), \(b\), and \(c\) be distinct noncollinear points.
Suppose we have points \(d\) and \(e\) such that \(\BETWEEN{a}{b}{d}\) and \(\BETWEEN{a}{e}{c}\).
Then there exists a point \(f\) such that \(\BETWEEN{e}{f}{d}\) and \(\BETWEEN{b}{f}{c}\).
\end{exercise}

\begin{exercise}
Let \(a\), \(b\), and \(c\) be distinct noncollinear points.
Suppose we have points \(d\) and \(e\) such that \(\BETWEEN{a}{b}{d}\) and \(\BETWEEN{b}{e}{c}\).
Then there exists a point \(f\) such that \(\BETWEEN{f}{e}{d}\) and \(\BETWEEN{a}{f}{c}\).
\end{exercise}

\begin{exercise}[Sylvester-Gallai Theorem]
If \(S\) is a finite set of points, then there is a line \(\ell\) containing exactly two points from \(S\). (@@@ Should this be a separate section? See \cite{pambuccian2009})
\end{exercise}
