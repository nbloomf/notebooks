In this section we will establish that the cartesian plane is an ordered geometry.
To do this, we need to specify (1) how to detect when one point is between two others, and (2) the halfplanes for each line.

\begin{prop}\label{prop:rr2-between}
Given points \(A\), \(B\), and \(C\) in \(\RR^2\), we say \(\BETWEEN{A}{C}{B}\) if \(A \neq B\) and the equation \(C = A + t(B-A)\) has a solution \(t \in (0,1)\).
Then this \(\BETWEEN{\ast}{\ast}{\ast}\) is a betweenness relation on \(\RR^2\).
\end{prop}

\begin{proof}\mbox{}
\begin{itemize}
\item[B1.] Suppose \(\BETWEEN{A}{C}{B}\).
Now \(A \neq B\) by definition, and we have \(C = A + t(B-A)\) for some \(t \in (0,1)\).
If \(C = A\), then \((0,0) = t(B-A)\), so that \(B = A\); a contradiction.
If \(C = B\), then \((0,0) = (t-1)(B-A)\), so that \(B = A\); a contradiction.
So \(A\), \(B\), and \(C\) are distinct.
Now \(A\), \(B\), and \(C\) are collinear because \[ \DET \begin{bmatrix} a_x & a_y & 1 \\ b_x & b_y & 1 \\ c_x & c_y & 1 \end{bmatrix} = \DET \begin{bmatrix} a_x & a_y & 1 \\ b_x & b_y & 1 \\ a_x + t(b_x-a_x) & a_y + t(b_x-a_x) & 1 \end{bmatrix} = 0. \]

\item[B2.] If \(C = A + t(B-A)\) where \(t \in (0,1)\), then (rearranging) we also have \(C = B + (1-t)(A-B)\) with \(1-t \in (0,1)\) as needed.

\item[B3.] Suppose \(A\), \(B\), and \(C\) are distinct such that \(\COLLINEAR{A}{B}{C}\).
Using \eref{exerc:rr2-collinear-comb}, we have \(C = A + t(B-A)\) for some real number \(t\).
Note that \(t \neq 0\) and \(t \neq 1\) since in the first case we would have \(C = A\) and in the second, \(C = B\).
There are then three possibilities for \(t\).
If \(t \in (0,1)\), then \(\BETWEEN{A}{C}{B}\) by definition.
If \(t > 1\), then ``solving for \(B\)'' we have \(B = A + \frac{1}{t}(C-A)\), and since \(1/t \in (0,1)\), we have \(\BETWEEN{A}{B}{C}\).
If \(t < 0\), then we have \(A = C + \frac{-t}{1-t}(B-C)\), and since \(\frac{-t}{1-t} \in (0,1)\), we have \(\BETWEEN{C}{A}{B}\).

\item[B4.] Suppose \(\BETWEEN{A}{B}{C}\) and \(\BETWEEN{A}{C}{D}\); say we have \(B = A + t(C-A)\) and \(C = A + s(D-A)\) where \(s,t \in (0,1)\).
Now \(B = A + ts(D-A)\), and since \(st \in (0,1)\), we have \(\BETWEEN{A}{B}{D}\).
Similarly, we have \(C = B + \frac{s-ts}{1-ts}(D-B)\), and since \(\frac{s-ts}{1-ts} \in (0,1)\), we have \(\BETWEEN{B}{C}{D}\) as needed.

\item[B5.] Suppose \(\BETWEEN{A}{B}{C}\) and \(\BETWEEN{B}{C}{D}\); say we have \(B = t(C-A)\) and \(C = B + s(D-B)\) where \(s,t \in (0,1)\).
Now \(C = A + \frac{s}{1-t+st}(D-A)\), and we also have \(\frac{s}{1-t+st} \in (0,1)\).
(To see this, note that \(0 < (1-s)(1-t)\) and rearrange to get \(s < 1-t+st\).)
Thus \(\BETWEEN{B}{C}{D}\).
Next note that \(B = A + \frac{ts}{1-t+ts}(D-A)\), and since \(\frac{ts}{1-t+ts} \in (0,1)\), we have \(\BETWEEN{A}{B}{D}\) as needed.

\item[B6.] Let \(A\) and \(B\) be distinct points.
Given a real number \(t\), let \(C = A + t(B-A)\).
If \(t \in (0,1)\), then \(\BETWEEN{A}{C}{B}\) by definition.
If \(t > 1\), then \(B = A + \frac{1}{t}(C-A)\) with \(\frac{1}{t} \in (0,1)\) and we have \(\BETWEEN{A}{B}{C}\).
If \(t < 0\), then \(A = C + \frac{-t}{1-t}(B-C)\) with \(\frac{-t}{1-t} \in (0,1)\) and we have \(\BETWEEN{C}{A}{B}\).
\qedhere
\end{itemize}
\end{proof}

Next, we'd like to show that \(\RR^2\) is an ordered geometry by showing that it has the Line Separation property.
First, we need the following technical lemma about the intersection of a segment and a line in \(\RR^2\).

\begin{lem}\label{lem:rr2-opp-side-incidence}
Let \(A = (a_x,a_y)\) and \(B = (b_x,b_y)\) be distinct points in \(\RR^2\), and let \(U = (u_x, u_y)\) and \(V = (v_x, v_y)\) be distinct points not on \(\LINE{A}{B}\).
Then \(\SEGMENT{U}{V} \cap \LINE{A}{B}\) consists of a single point if and only if \[ \DET \begin{bmatrix} a_x & a_y & 1 \\ b_x & b_y & 1 \\ u_x & u_y & 1 \end{bmatrix} \quad \mathrm{and} \quad \DET \begin{bmatrix} a_x & a_y & 1 \\ b_x & b_y & 1 \\ v_x & v_y & 1 \end{bmatrix} \] have opposite signs.
\end{lem}

\begin{proof}
Note that \(\LINE{U}{V} \cap \LINE{A}{B}\) contains exactly one point if and only if the equation \(U + t(V-U) = A + u(B-A)\) has a unique solution \((t,u)\).
In fact, by comparing coordinates we can rewrite this equation in matrix form as \[ \begin{bmatrix} t \\ -u \end{bmatrix} = \begin{bmatrix} v_x - u_x & b_x - a_x \\ v_y - u_y & b_y - a_y \end{bmatrix}^{-1} \begin{bmatrix} a_x - u_x \\ a_y - u_y \end{bmatrix}. \]
(This matrix is invertible by \eref{exerc:parallels-in-rr2}.) Comparing entries, we have \[ t = \frac{(b_y-a_y)(a_x-u_x) - (b_x-a_x)(a_y-u_y)}{(v_x-u_x)(b_y-a_y) - (v_y-u_y)(b_x-a_x)}. \]
Now by definition the unique point in \(\LINE{U}{V} \cap \LINE{A}{B}\) is more specifically on the segment \(\SEGMENT{U}{V}\) if and only if \(t \in (0,1)\).

There are now two possibilities, depending on whether the denominator of \(t\) is positive or negative.
If the denominator of \(t\) is positive, we can see that \(t \in (0,1)\) if and only if \[ \DET \begin{bmatrix} a_x & a_y & 1 \\ b_x & b_y & 1 \\ u_x & u_y & 1 \end{bmatrix} > 0 > \DET \begin{bmatrix} a_x & a_y & 1 \\ b_x & b_y & 1 \\ v_x & v_y & 1 \end{bmatrix}, \]
and if the denominator of \(t\) is negative, then \(t \in (0,1)\) if and only if \[ \DET \begin{bmatrix} a_x & a_y & 1 \\ b_x & b_y & 1 \\ u_x & u_y & 1 \end{bmatrix} < 0 < \DET \begin{bmatrix} a_x & a_y & 1 \\ b_x & b_y & 1 \\ v_x & v_y & 1 \end{bmatrix} \] as needed.
\end{proof}

We are now prepared to show the following.

\begin{prop}\label{prop:rr2-line-sep}
For each line \(\ell = \LINE{A}{B}\) in \(\RR^2\), we define two halfplanes as follows.
\begin{eqnarray*}
\HALFPLANEA{\ell} & = & \left\{ (x, y) \mid \DET \begin{bmatrix} a_x & a_y & 1 \\ b_x & b_y & 1 \\ x & y & 1 \end{bmatrix} > 0 \right\} \\[4pt]
\HALFPLANEB{\ell} & = & \left\{ (x, y) \mid \DET \begin{bmatrix} a_x & a_y & 1 \\ b_x & b_y & 1 \\ x & y & 1 \end{bmatrix} < 0 \right\}.
\end{eqnarray*}
With halfplanes defined this way for all lines, \(\RR^2\) satisfies the Line Separation Property and thus is an ordered geometry.
\end{prop}

\begin{proof}
It is enough to show that the Line Separation property is satisfied.
But first, we need to verify that our halfplanes are well-defined; that is, they do not depend on the choice of line generators.
(@@@)

Certainly \(\HALFPLANEA{\ell}\), \(\HALFPLANEB{\ell}\), and \(\ell\) partition \(\RR^2\) by the trichotomy property of \(<\).
To see that both halfplanes are nonempty we consider three cases: if \(a_x = b_x\), then \((a_x+1,0)\) and \((a_x-1,0)\) are in opposite halfplanes; if \(a_y = b_y\), then \((0,a_y+1)\) and \((0,a_y-1)\) are in opposite halfplanes; and if \(a_x \neq b_x\) and \(a_y \neq b_y\) then \((a_x,b_y)\) and \((b_x,a_y)\) are in opposite halfplanes.
By Lemma \ref{lem:rr2-opp-side-incidence}, if \(U \in \HALFPLANEA{\ell}\) and \(V \in \HALFPLANEB{\ell}\), then \(\SEGMENT{U}{V} \cap \LINE{A}{B}\) consists of a unique point.
All that remains is to show that \(\HALFPLANEA{\ell}\) and \(\HALFPLANEB{\ell}\) are convex.

To see that \(\HALFPLANEA{\ell}\) is convex, let \(U,V \in \HALFPLANEA{\ell}\).
By definition, we have \[ \DET \begin{bmatrix} a_x & a_y & 1 \\ b_x & b_y & 1 \\ u_x & u_y & 1 \end{bmatrix} > 0\ \mathrm{and}\ \DET \begin{bmatrix} a_x & a_y & 1 \\ b_x & b_y & 1 \\ v_x & v_y & 1 \end{bmatrix} > 0. \]
Now suppose \(\BETWEEN{U}{W}{V}\); then again by definition we have \(t \in (0,1)\) such that \(W = U + t(V-U)\).
Then
\begin{eqnarray*}
 & & \DET \begin{bmatrix} a_x & a_y & 1 \\ b_x & b_y & 1 \\ w_x & w_y & 1 \end{bmatrix} \\
 & = & \DET \begin{bmatrix} a_x & a_y & 1 \\ b_x & b_y & 1 \\ u_x + t(v_x - u_x) & u_y + t(v_y - u_y) & 1 + t(1 - 1) \end{bmatrix} \\
 & = & \DET \begin{bmatrix} a_x & a_y & 1 \\ b_x & b_y & 1 \\ u_x & u_y & 1 \end{bmatrix} + t \left( \DET \begin{bmatrix} a_x & a_y & 1 \\ b_x & b_y & 1 \\ v_x & v_y & 1 \end{bmatrix} - \DET \begin{bmatrix} a_x & a_y & 1 \\ b_x & b_y & 1 \\ u_x & u_y & 1 \end{bmatrix} \right) \\
 & > & 0
\end{eqnarray*}
using \eref{exerc:rr-between-sign}, and because the determinant is multilinear.
So we have \(W \in \HALFPLANEA{\ell}\), and thus \(\HALFPLANEA{\ell}\) is convex.
A similar argument shows that \(\HALFPLANEB{\ell}\) is convex.
\end{proof}

The proofs of Propositions \ref{prop:rr2-between} and \ref{prop:rr2-line-sep} remain valid if we replace \(\RR^2\) by \(\QQ^2\), so that the Rational Plane is also an ordered geometry.
However we cannot replace \(\RR^2\) by \(\CC^2\), because the order relation \(<\) does not make sense in the complex numbers.



%---------%
\Exercises%
%---------%

\begin{exercise}\label{exerc:rr2-ray-positive}
Let \(A,B,C\) be points in the cartesian plane with \(A \neq B\).
Show that \(C \in \RAY{A}{B}\) if and only if \(C = A + t(B - A)\) for some \(t \in (0,\infty)\).
\end{exercise}
