Intuitively, we want to say that two sets of points in a geometry are ``congruent'' if they have the same size and shape.
But what exactly do ``size'' and ``shape'' mean?
Rather than defining congruence once and for all, we will instead define two primitive forms of congruence on segments and angles.
Then congruence for more complicated sets can be defined in terms of the primitives.

\begin{dfn}[Segment Congruence]
Let \(P\) be an ordered geometry, and suppose we have an equivalence relation on pairs of points, denoted \(\SEGCONG{\ast}{\ast}{\ast}{\ast}\).
We call \(\SEGCONG{\ast}{\ast}{\ast}{\ast}\) a \emph{segment congruence}\index{congruence!of segments} if the following properties are satisfied.
\begin{proplist}
\item[SC1.] \(\SEGCONG{x}{x}{y}{y}\) for all points \(x\) and \(y\).
\item[SC2.] \(\SEGCONG{x}{y}{y}{x}\) for all points \(x\) and \(y\).
\item[SC3.] If \(z \in \RAY{x}{y}\) such that \(\SEGCONG{x}{z}{x}{y}\), then \(z = y\).
\end{proplist}
In this case, \(\SEGCONG{\ast}{\ast}{\ast}{\ast}\) is well-defined on the set of \emph{segments} in \(P\), where we write \(\SEGMENT{x}{y} \equiv \SEGMENT{a}{b}\) to mean \(\SEGCONG{x}{y}{a}{b}\).
\end{dfn}

The first property handles the ``trivial'' case, the second makes the relation well-defined on segments, and the third ensures that it differentiates between segments on the same ray which share an endpoint.

\begin{dfn}[Angle Congruence]
Let \(P\) be an ordered geometry, and suppose we have an equivalence relation on triples of points, denoted \(\ANGCONG{\ast}{\ast}{\ast}{\ast}{\ast}{\ast}\).
We call \(\ANGCONG{\ast}{\ast}{\ast}{\ast}{\ast}{\ast}\) an \emph{angle congruence}\index{congruence!of angles} if the following properties are satisfied.
\begin{itemize}
\item[AC1.] If \(\BETWEEN{a_1}{b_1}{c_1}\) and \(\BETWEEN{a_2}{b_2}{c_2}\), then \(\ANGCONG{a_1}{b_1}{c_1}{a_2}{b_2}{c_2}\) and \(\ANGCONG{b_1}{a_1}{c_1}{b_2}{a_2}{c_2}\), and it is not the case that \(\ANGCONG{a_1}{b_1}{c_1}{b_1}{a_1}{c_1}\).
\item[AC2.] If \(a\), \(b\), \(o\), \(x\), and \(y\) are points such that \(o \notin \{a,b,x,y\}\), \(x \in \RAY{o}{a}\), and \(y \in \RAY{o}{b}\), then \(\ANGCONG{a}{o}{b}{x}{o}{y}\).
\item[AC3.] If \(a\), \(b\), and \(o\) are points such that \(o \notin \{a,b\}\), then \(\ANGCONG{a}{o}{b}{b}{o}{a}\).
\item[AC4.] If \(a\), \(b\), and \(o\) are distinct noncollinear points and \(x\) is on the \(b\)-side of \(\LINE{o}{a}\) such that \(\ANGCONG{a}{o}{b}{a}{o}{x}\), then \(x \in \RAY{o}{b}\).
\end{itemize}
In this case, \(\ANGCONG{\ast}{\ast}{\ast}{\ast}{\ast}{\ast}\) is an equivalence relation on the set of \emph{angles} in \(P\), and we write \(\ANGLE{a}{o}{b} \equiv \ANGLE{x}{p}{y}\) to mean \(\ANGCONG{a}{o}{b}{x}{p}{y}\).
\end{dfn}

Much like the properties of segment congruence, the first property handles the trivial cases, the second and third make the relation well-defined on angles, and the fourth ensures that it differentiates between angles on one half-plane which share a vertex.

\begin{dfn}[Congruence Geometry]
Let \(P\) be an ordered geometry.
If \(P\) has a segment congruence and an angle congruence, we say that \(P\) is an \emph{congruence geometry}\index{congruence geometry}.
\end{dfn}

We can define congruence of many different kinds of figures in terms of segment and angle congruence.
For instance...

\begin{dfn}[Triangle Congruence]
Let \(a\), \(b\), and \(c\) be distinct points, and let \(x\), \(y\), and \(z\) be distinct points.
We say that \(\TRIANGLE{a}{b}{c}\) is \emph{congruent}\index{congruence!of triangles} to \(\TRIANGLE{x}{y}{z}\), denoted \(\TRIANGLE{a}{b}{c} \equiv \TRIANGLE{x}{y}{z}\), if \[ \SEGMENT{a}{b} \equiv \SEGMENT{x}{y}, \quad \SEGMENT{b}{c} \equiv \SEGMENT{y}{z}, \quad \mathrm{and} \quad \SEGMENT{c}{a} \equiv \SEGMENT{z}{x} \] and \[ \ANGLE{a}{b}{c} \equiv \ANGLE{x}{y}{z}, \quad \ANGLE{b}{c}{a} \equiv \ANGLE{y}{z}{x}, \quad \mathrm{and} \quad \ANGLE{c}{a}{b} \equiv \ANGLE{z}{x}{y}. \]
\end{dfn}

\begin{dfn}
Let \(a\), \(b\), and \(c\) be distinct points.
\begin{proplist}
\item We say that the triangle \(\TRIANGLE{a}{b}{c}\) is \emph{equilateral}\index{equilateral} if \(\SEGMENT{a}{b} \equiv \SEGMENT{b}{c} \equiv \SEGMENT{c}{a}\).
\item We say that the triangle \(\TRIANGLE{a}{b}{c}\) is \emph{isoceles}\index{isoceles} if two of its sides are congruent to each other.
\end{proplist}
\end{dfn}


\begin{dfn}[Supplementary Angles]
We say that angles \(\ANGLE{a}{o}{b}\) and \(\ANGLE{x}{p}{y}\) are \emph{supplementary}\index{supplementary angles} if there is a linear pair, \(\ANGLE{u}{q}{v}\) and \(\ANGLE{v}{q}{w}\), such that \(\ANGLE{a}{o}{b} \equiv \ANGLE{u}{q}{v}\) and \(\ANGLE{x}{p}{y} \equiv \ANGLE{v}{q}{w}\).
In this case we say that \(\ANGLE{x}{p}{y}\) is a \emph{supplement} of \(\ANGLE{a}{o}{b}\).
\end{dfn}

\begin{prop}
Let \(P\) be a congruence geometry.
\begin{proplist}
\item If two angles form a linear pair, then they are supplementary.
\item Every angle has a supplement.
\end{proplist}
\end{prop}

\begin{dfn}
An angle is called \emph{right} if it is supplementary to itself.
\end{dfn}


%---------%
\Exercises%
%---------%

\begin{exercise}
Show that triangle congruence is an equivalence relation.
\end{exercise}
