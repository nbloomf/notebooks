\begin{dfn}[Circle]
Let \(P\) be a congruence geometry and let \(o,a \in P\) be points.
The set \[ \CIRCLE{o}{a} = \{ x \in P \mid \SEGMENT{o}{x} \equiv \SEGMENT{o}{a} \} \] is called the \emph{circle}\index{circle} with \emph{center} \(o\) and \emph{passing through} \(a\).
If \(o = a\), we say the circle \(\CIRCLE{o}{a}\) is \emph{degenerate}.
\end{dfn}


\begin{dfn}[Radius, Diameter, Chord]
Let \(o\) and \(a\) be distinct points in a congruence geometry, and let \(C = \CIRCLE{o}{a}\).
\begin{proplist}
\item If \(u \in C\), then the segment \(\SEGMENT{o}{u}\) is called a \emph{radius}\index{radius} of \(C\).
\item If \(u\) and \(v\) are distinct points on \(C\), then the segment \(\SEGMENT{u}{v}\) is called a \emph{chord}\index{chord} of \(C\).
\item A chord \(\SEGMENT{u}{v}\) of \(C\) is called a \emph{diameter}\index{diameter} if \(\BETWEEN{u}{o}{v}\).
\end{proplist}
\end{dfn}
