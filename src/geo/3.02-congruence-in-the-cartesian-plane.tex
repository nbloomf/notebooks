Remember: to show that an ordered geometry is a congruence geometry, we need to specify (1) how to detect when two segments are congruent and (2) how to detect when two angles are congruent.
In this section we will establish segment and angle congruences in the cartesian plane.

We define two helper functions, \(N\) and \(M\), as follows.
Given \(A,B \in \RR^2\), we define \[ N(A,B) = (b_x - a_x)^2 + (b_y - a_y)^2. \]
Given \(A,B,O \in \RR^2\), we define \[ M(A,O,B) = (a_x - o_x)(b_x - o_x) + (a_y - o_y)(b_y - o_y). \]

\begin{prop}\label{prop:rr2-cong-helper}
The following hold for all cartesian points \(A\) and \(B\).
\begin{proplist}
\item \label{prop:rr2-cong-helper:zero} \(N(A,B) \geq 0\), with equality if and only if \(A = B\).
\item \label{prop:rr2-cong-helper:sym} \(N(A,B) = N(B,A)\).
\item \(M(A,O,B) = M(B,O,A)\).
\item If \(C\) is a point such that \(A\), \(B\), and \(C\) are distinct and \(\COLLINEAR{A}{B}{C}\), then \[ \frac{M(A,B,C)^2}{N(A,B)N(C,B)} = 1. \]
\end{proplist}
\end{prop}

\begin{prop}
\(\RR^2\) is a congruence geometry under the following relations.
\begin{proplist}
\item Given points \(A\), \(B\), \(X\), and \(Y\) in \(\RR^2\), we say that \(\SEGCONG{A}{B}{X}{Y}\) if \[ N(A,B) = N(X,Y). \]
\item Given points \(A\), \(O\), \(B\), \(X\), \(P\), and \(Y\) in \(\RR^2\) such that \(A \neq O\), \(B \neq O\), \(X \neq P\), and \(Y \neq P\), we say that \(\ANGCONG{A}{O}{B}{X}{P}{Y}\) if \(M(A,O,B)\) and \(M(X,P,Y)\) have the same sign and \[ \frac{M(A,O,B)^2}{N(A,O)N(B,O)} = \frac{M(X,P,Y)^2}{N(X,P)N(Y,P)}. \]
\end{proplist}
\end{prop}

\begin{proof}
Certainly both \(\SEGCONG{\ast}{\ast}{\ast}{\ast}\) and \(\ANGCONG{\ast}{\ast}{\ast}{\ast}{\ast}{\ast}\) are equivalence relations.
\begin{itemize}
\item[SC1.] Note that \(N(A,A) = 0 = N(B,B)\) for all points \(A\) and \(B\) by \sref{prop:rr2-cong-helper}{zero}.

\item[SC2.] We have \(N(A,B) = N(B,A)\) for all points \(A\) and \(B\) by \sref{prop:rr2-cong-helper}{sym}

\item[SC3.] Suppose we have \(C \in \RAY{A}{B}\) such that \(\SEGCONG{A}{C}{A}{B}\).
By \eref{exerc:rr2-ray-positive}, we have \(C = A + t(B - A)\) for some real number \(t > 0\).
Comparing coordinates, then, we have \(c_x - a_x = t(b_x - a_x)\) and \(c_y - a_y = t(b_y - a_y)\).
Expanding the definition of segment congruence, and noting that \(A \neq B\), we see that \(t^2 = 1\).
Since \(t\) is positive, in fact \(t = 1\), and thus \(C = B\) as needed.

\item[AC1.] (@@@)

\item[AC2.] (@@@)

\item[AC3.] (@@@)

\item[AC4.] (@@@)
\end{itemize}
\end{proof}
