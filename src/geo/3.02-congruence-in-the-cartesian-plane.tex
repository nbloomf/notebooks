Remember: to show that an ordered geometry is a congruence geometry, we need to specify (1) how to detect when two segments are congruent and (2) how to detect when two angles are congruent.
In this section we will establish segment and angle congruences in the cartesian plane.

We define two helper functions, \(N\) and \(M\), as follows.
Given \(A,B \in \RR^2\), we define \[ N(A,B) = (b_x - a_x)^2 + (b_y - a_y)^2. \]
Given \(A,B,O \in \RR^2\), we define \[ M(A,O,B) = (a_x - o_x)(b_x - o_x) + (a_y - o_y)(b_y - o_y). \]

\begin{prop}
The following hold for all cartesian points \(A\) and \(B\).
\begin{proplist}
\item \(N(A,B) = 0\) if and only if \(A = B\).
\item \(N(A,B) = N(B,A)\).
\item \(M(A,O,B) = M(B,O,A)\).
\end{proplist}
\end{prop}

\begin{prop}
\(\RR^2\) is a congruence geometry under the following relations.
\begin{proplist}
\item Given points \(A\), \(B\), \(X\), and \(Y\) in \(\RR^2\), we say that \(\SEGCONG{A}{B}{X}{Y}\) if \[ N(A,B) = N(X,Y). \]
\item Given points \(A\), \(O\), \(B\), \(X\), \(P\), and \(Y\) in \(\RR^2\) such that \(A \neq O\), \(B \neq O\), \(X \neq P\), and \(Y \neq P\), we say that \(\ANGCONG{A}{O}{B}{X}{P}{Y}\) if \[ \frac{M(A,O,B)^2}{N(A,O)N(B,O)} = \frac{M(X,P,Y)^2}{N(X,P)N(Y,P)}. \]
\end{proplist}
\end{prop}

\begin{proof}
(@@@)
\end{proof}
