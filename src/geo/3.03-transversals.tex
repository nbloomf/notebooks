\begin{prop}[Supplements are unique] \mbox{}
\begin{proplist}
\item Suppose that \(\ANGLE{a_1}{o_1}{b_1}\) and \(\ANGLE{b_1}{o_1}{c_1}\) are a linear pair, and that \(\ANGLE{a_2}{o_2}{b_2}\) and \(\ANGLE{b_2}{o_2}{c_2}\) are a linear pair.
If \(\ANGLE{a_1}{o_1}{b_1} \equiv \ANGLE{a_2}{o_2}{b_2}\), then \(\ANGLE{b_1}{o_1}{c_1} \equiv \ANGLE{b_2}{o_2}{c_2}\).

\begin{center}
\begin{tikzpicture}[scale=0.8]
  \coordinate [label=below:\(o_1\)] (o1) at (-3,0);
  \coordinate [label=below:\(a_1\)] (a1) at ($ (o1)+(180:1) $);
  \coordinate [label=left:\(b_1\)] (b1) at ($ (o1)+(70:1) $);
  \coordinate [label=below:\(c_1\)] (c1) at ($ (o1)+(0:1) $);

  \draw [fill] (o1) circle [radius=1pt];
  \draw [fill] (a1) circle [radius=1pt];
  \draw [fill] (b1) circle [radius=1pt];
  \draw [fill] (c1) circle [radius=1pt];

  \draw [->] (o1) -- ($ (a1)+(180:0.5) $);
  \draw [->] (o1) -- ($ (b1)+(70:0.5) $);
  \draw [->] (o1) -- ($ (c1)+(0:0.5) $);

  \coordinate [label=below:\(o_2\)] (o2) at (2,0);
  \coordinate [label=below:\(a_2\)] (a2) at ($ (o2)+(180:1) $);
  \coordinate [label=left:\(b_2\)] (b2) at ($ (o2)+(80:1) $);
  \coordinate [label=below:\(c_2\)] (c2) at ($ (o2)+(0:1) $);

  \draw [fill] (o2) circle [radius=1pt];
  \draw [fill] (a2) circle [radius=1pt];
  \draw [fill] (b2) circle [radius=1pt];
  \draw [fill] (c2) circle [radius=1pt];

  \draw [->] (o2) -- ($ (a2)+(180:0.5) $);
  \draw [->] (o2) -- ($ (b2)+(80:0.5) $);
  \draw [->] (o2) -- ($ (c2)+(0:0.5) $);
\end{tikzpicture}
\end{center}

\item Suppose \(\alpha\), \(\beta\), and \(\gamma\) are angles such that \(\alpha\) and \(\beta\) are supplementary and \(\alpha\) and \(\gamma\) are supplementary.
Then \(\alpha \equiv \gamma\).
\end{proplist}
\end{prop}

\begin{proof}\mbox{}
\begin{proplist}
\item Suppose we have two such linear pairs.
Without loss of generality, we can suppose that \[ \SEGMENT{o_1}{a_1} \equiv \SEGMENT{o_1}{b_1} \equiv \SEGMENT{o_1}{c_1} \equiv \SEGMENT{o_2}{a_2} \equiv \SEGMENT{o_2}{b_2} \equiv \SEGMENT{o_2}{c_2}. \] (If they aren't, we can use Circle Cut and the Segment Copy construction to find such points.) Now \(\TRIANGLE{b_1}{o_1}{a_1} \equiv \TRIANGLE{b_2}{o_2}{a_2}\) by SAS, so that \(\ANGLE{b_1}{a_1}{o_1} \equiv \ANGLE{b_2}{a_2}{o_2}\).
Now \(\SEGMENT{a_1}{c_1} \equiv \SEGMENT{a_2}{c_2}\), so that \(\TRIANGLE{b_1}{a_1}{c_1} \equiv \TRIANGLE{b_2}{a_2}{c_2}\) by SAS.
So \(\SEGMENT{b_1}{c_1} \equiv \SEGMENT{b_2}{c_2}\), and thus \(\TRIANGLE{b_1}{o_1}{c_1} \equiv \TRIANGLE{b_2}{o_2}{c_2}\) by SSS.
Thus \(\ANGLE{b_1}{o_1}{c_1} \equiv \ANGLE{b_2}{o_2}{c_2}\) as needed.

\item Follows from the definition of supplementary.
\qedhere
\end{proplist}
\end{proof}

\begin{cor}
If angles \(\alpha\) and \(\beta\) are a vertical pair, then \(\alpha \equiv \beta\).
\end{cor}

\begin{proof}
If \(\ANGLE{a}{o}{b}\) and \(\ANGLE{c}{o}{d}\) are a vertical pair, with \(\BETWEEN{b}{o}{c}\) and \(\BETWEEN{a}{o}{d}\), then both \(\ANGLE{a}{o}{b}\) and \(\ANGLE{c}{o}{d}\) are supplementary to \(\ANGLE{a}{o}{c}\).
\end{proof}

\begin{dfn}
An angle is called \emph{right} if it is supplementary to itself.
\end{dfn}

\begin{cor}
Any two right angles are congruent.
\end{cor}

\begin{proof}
Suppose angles \(\alpha\) and \(\beta\) are right.
We can copy \(\beta\) to an angle \(\beta'\) in the configuration of \ref{lem:coinitial-right-angle-congruence}.
Since supplements are congruent, \ref{lem:coinitial-right-angle-congruence} applies.
\end{proof}


\begin{dfn}[Transversal]
Suppose we have three lines \(\ell_1\), \(\ell_2\), and \(t\) in a plane geometry.
We say that \(t\) is a \emph{transversal}\index{transversal} of \(\ell_1\) and \(\ell_2\) if \(t\) cuts both \(\ell_1\) and \(\ell_2\) at unique points, and these points are distinct.

Suppose \(t\) is a transversal of \(\ell_1\) and \(\ell_2\), cutting these lines at \(o_1\) and \(o_2\), respectively as shown.

\begin{center}
\begin{tikzpicture}[scale=0.5]
  \coordinate (x1) at (-5,0);
  \coordinate (y1) at (4,0);
  \draw [<->] (x1) -- (y1);
  \node at (4,-0.5) {\(\ell_1\)};
  \coordinate (x2) at (-4,5);
  \coordinate (y2) at (4,3);
  \draw [<->] (x2) -- (y2);
  \node  at (-3,5.5) {\(\ell_2\)};
  \coordinate (x3) at (-1,-1);
  \coordinate (y3) at (1,6);
  \draw [<->] (x3) -- (y3);
  \node at (0.7,6) {\(t\)};
  \coordinate [label=below right:\(o_1\)] (o1) at (intersection of x1--y1 and x3--y3);
  \draw [fill] (o1) circle [radius=1pt];
  \coordinate [label=below left:\(o_2\)] (o2) at (intersection of x2--y2 and x3--y3);
  \draw [fill] (o2) circle [radius=1pt];
  \coordinate (x4) at (-0.5,2);
  \coordinate (y4) at (1,3);
  \coordinate [label=below:\(a\)] (a) at (intersection of x1--y1 and x4--y4);
  \draw [fill] (a) circle [radius=1pt];
  \coordinate [label=below:\(b\)] (b) at (intersection of x2--y2 and x4--y4);
  \draw [fill] (b) circle [radius=1pt];
\end{tikzpicture}
\end{center}

If \(a\) is on \(\ell_1\) and \(b\) is on \(\ell_2\) such that \(a\) and \(b\) are on opposite sides of \(t\), then we say that \(\ANGLE{a}{o_1}{o_2}\) and \(\ANGLE{b}{o_2}{o_1}\) are \emph{alternate interior angles}\index{alternate interior angles} of this transversal.
\end{dfn}

Every transversal has two pairs of alternate interior angles.

\begin{prop}[AIA Theorem]\label{prop:aia-theorem}
If two lines \(\ell_1\) and \(\ell_2\) are cut by a transversal \(t\) so that a pair of alternate interior angles are congruent, then \(\ell_1\) and \(\ell_2\) are parallel.
\end{prop}

\begin{proof}
Suppose \(t\) meets \(\ell_1\) and \(\ell_2\) at points \(o_1\) and \(o_2\) respectively, and that \(a\) and \(b\) are on \(\ell_1\) and \(\ell_2\), respectively, and on opposite sides of \(t\), such that \(\ANGLE{a}{o_1}{o_2} \equiv \ANGLE{b}{o_2}{o_1}\).
Now suppose by way of contradiction that \(\ell_1\) and \(\ell_2\) are \emph{not} parallel; rather, they meet at a point \(x\) which (WLOG) is on the \(a\)-side of \(t\).
Let \(c\) be on \(\ell_1\) such that \(\BETWEEN{a}{o_1}{c}\).
Copy \(\SEGMENT{o_1}{x}\) onto \(\RAY{o_2}{b}\) at the point \(y\).
Note that \(\SEGMENT{o_1}{x} \equiv \SEGMENT{o_2}{y}\), \(\SEGMENT{o_1}{o_2} \equiv \SEGMENT{o_2}{o_1}\), and \(\ANGLE{x}{o_1}{o_2} \equiv \ANGLE{y}{o_2}{o_1}\); by \hyperref[prop:sas-theorem]{SAS} we thus have \(\TRIANGLE{x}{o_1}{o_2} \equiv \TRIANGLE{y}{o_2}{o_1}\).
In particular, \(\ANGLE{o_2}{o_1}{y} \equiv \ANGLE{o_1}{o_2}{x}\).

\begin{center}
\begin{tikzpicture}[scale=0.8]
  \coordinate [label=above right:\(o_1\)] (o1) at (0,2);
  \coordinate [label=below right:\(o_2\)] (o2) at ($ (o1)+(270:2) $);
  \coordinate [label=above:\(a\)] (a) at ($ (o1)+(160:-1) $);
  \coordinate [label=above:\(c\)] (c) at ($ (o1)+(160:1) $);
  \coordinate [label=above:\(b\)] (b) at ($ (o2)+(200:1) $);
  \coordinate [label=below:\(x\)] (x) at (intersection of o1--a and o2--b);
  \coordinate [label=above:\(y\)] (y) at ($ (o2)+(200:1.5) $);

  \draw [fill] (o1) circle [radius=1pt];
  \draw [fill] (o2) circle [radius=1pt];
  \draw [fill] (a) circle [radius=1pt];
  \draw [fill] (c) circle [radius=1pt];
  \draw [fill] (b) circle [radius=1pt];
  \draw [fill] (x) circle [radius=1pt];
  \draw [fill] (y) circle [radius=1pt];

  \draw [<->] ($ (o1)+(90:0.5) $) -- ($ (o2)+(270:0.5) $);
  \draw [<->] ($ (y)+(200:0.5) $) -- ($ (x)+(200:-0.5) $);
  \draw [<->] ($ (c)+(160:0.5) $) -- ($ (x)+(160:-0.5) $);
\end{tikzpicture}
\end{center}

Note that \(\ANGLE{x}{o_2}{o_1}\) and \(\ANGLE{o_1}{o_2}{y}\) are supplementary, and \(\ANGLE{o_1}{o_2}{y} \equiv \ANGLE{a}{o_1}{o_2}\) by hypothesis, so that \(\ANGLE{a}{o_1}{o_2}\) and \(\ANGLE{x}{o_2}{o_1}\) are supplementary.
Since \(\ANGLE{x}{o_2}{o_1} \equiv \ANGLE{y}{o_1}{o_2}\), we have that \(\ANGLE{a}{o_1}{o_2}\) and \(\ANGLE{y}{o_1}{o_2}\) are supplementary.
But also \(\ANGLE{a}{o_1}{o_2}\) and \(\ANGLE{o_2}{o_1}{c}\) are supplementary.
Thus \(\ANGLE{o_2}{o_1}{y} \equiv \ANGLE{o_2}{o_1}{c}\).
By AC4, we have that \(o_1\), \(c\), and \(y\) are collinear, so that \(y \in \ell_1\).
But now \(\ell_1\) and \(\ell_2\) have two points in common -- \(x\) and \(y\) -- which are necessarily distinct as they are on opposite halfplanes of \(t\).
So we have \(\ell_1 = \ell_2\), a contradiction.

Thus \(\ell_1\) and \(\ell_2\) must be parallel. 
\end{proof}

\begin{cor}\label{cor:a-triangle-has-at-most-one-right-angle}
A triangle formed by three distinct noncollinear points cannot have two interior angles which are both right.
\end{cor}

\begin{proof}
Such a triangle would violate the \hyperref[prop:aia-theorem]{AIA Theorem} since right angles are self-supplementary, and any two right angles are congruent.
\end{proof}

\begin{prop}[AAS Theorem]\label{prop:aas-theorem}
Suppose we have distinct noncollinear points \(a_1\), \(b_1\), and \(c_1\) and distinct noncollinear points \(a_2\), \(b_2\), and \(c_2\) such that \(\ANGLE{c_1}{a_1}{b_1} \equiv \ANGLE{c_2}{a_2}{b_2}\), \(\ANGLE{a_1}{b_1}{c_1} \equiv \ANGLE{a_2}{b_2}{c_2}\), and \(\SEGMENT{b_1}{c_1} \equiv \SEGMENT{b_2}{c_2}\).
Then \(\TRIANGLE{a_1}{b_1}{c_1} \equiv \TRIANGLE{a_2}{b_2}{c_2}\).
\end{prop}

\begin{proof}
Copy \(\SEGMENT{b_1}{a_1}\) onto \(\RAY{b_2}{a_2}\) at the point \(w\).
Note that \(w \neq b_2\), since \(a_1 \neq b_1\).
Note also that \(\TRIANGLE{w}{b_2}{c_2} \equiv \TRIANGLE{a_1}{b_2}{c_2}\) by \hyperref[prop:sas-theorem]{SAS}, so that \(\ANGLE{b_1}{a_1}{c_1} \equiv \ANGLE{b_2}{w}{c_2}\).
Suppose now that \(w\) and \(a_2\) are distinct points; say \(\BETWEEN{b_2}{w}{a_2}\).
(The case \(\BETWEEN{b_2}{a_2}{w}\) is very similar.)
In this case \(\LINE{a_2}{c_2}\) and \(\LINE{w}{c_2}\) are lines cut by a transversal \(\LINE{a_2}{b_2}\).

\begin{center}
\begin{tikzpicture}[scale=0.8]
  \coordinate [label=above right:\(w\)] (w) at (0,0);
  \coordinate [label=below left:\(a_2\)] (a2) at ($ (w)+(160:2) $);
  \coordinate [label=above:\(b_2\)] (b2) at ($ (w)+(160:-2.5) $);
  \coordinate [label=below:\(c_2\)] (c2) at ($ (w)+(100:-2) $);
  \coordinate [label=left:\(u\)] (u) at ($ (w)+(100:1) $);

  \draw [fill] (w) circle [radius=1pt];
  \draw [fill] (a2) circle [radius=1pt];
  \draw [fill] (b2) circle [radius=1pt];
  \draw [fill] (c2) circle [radius=1pt];
  \draw [fill] (u) circle [radius=1pt];

  \draw (a2) -- (b2) -- (c2) -- cycle;
  \draw [->] (c2) -- ($ (u)+(100:0.5) $);
\end{tikzpicture}
\end{center}

Moreover, if we let \(u\) be a point such that \(\BETWEEN{u}{w}{c_2}\), then \(\ANGLE{u}{w}{a_2}\) and \(\ANGLE{b_2}{w}{c_2}\) are vertical, hence congruent, and so we have \(\ANGLE{u}{w}{a_2} \equiv \ANGLE{b_2}{a_2}{c_2}\).
But now by the \hyperref[prop:aia-theorem]{AIA Theorem}, \(\LINE{a_2}{c_2}\) and \(\LINE{w}{c_2}\) must be parallel -- a contradiction since they meet at \(c_2\).
So in fact \(a_2\) and \(w\) are the same point, and thus \(\TRIANGLE{a_1}{b_1}{c_1} \equiv \TRIANGLE{a_2}{b_2}{c_2}\) by \hyperref[prop:sas-theorem]{SAS}.
\end{proof}

Note that the AAS Theorem is slightly less general than the triangle congruence theorems we've seen so far: it does not apply to triangles whose vertices are collinear, while the SSS and SAS Theorems do.

\begin{prop}[HL Theorem]
Suppose we have distinct noncollinear points \(a_1\), \(b_1\), and \(c_1\) and distinct noncollinear points \(a_2\), \(b_2\), and \(c_2\) such that \(\ANGLE{b_1}{c_1}{a_1}\) and \(\ANGLE{b_2}{c_2}{a_2}\) are right and \(\SEGMENT{a_1}{b_1} \equiv \SEGMENT{a_2}{b_2}\) and \(\SEGMENT{b_1}{c_1} \equiv \SEGMENT{b_2}{c_2}\).
Then \(\TRIANGLE{a_1}{b_1}{c_1} \equiv \TRIANGLE{a_2}{b_2}{c_2}\).
\end{prop}

\begin{proof}
Copy \(\SEGMENT{c_2}{a_2}\) onto the ray opposite \(\RAY{c_1}{a_1}\) at the point \(d\).
Now \(\ANGLE{b_1}{c_1}{d}\) is a right angle, since it is supplementary to \(\ANGLE{a_1}{c_1}{b_1}\).
By \hyperref[prop:sas-theorem]{SAS}, we have \(\TRIANGLE{a_2}{c_2}{b_2} \equiv \TRIANGLE{d}{c_1}{b_1}\), and thus \(\SEGMENT{b_1}{d} \equiv \SEGMENT{b_2}{a_2} \equiv \SEGMENT{b_1}{a_1}\).
Now \(\TRIANGLE{a_1}{b_1}{d}\) is isoceles with \(\SEGMENT{b_1}{a_1} \equiv \SEGMENT{b_1}{d}\), so that \(\ANGLE{b_1}{a_1}{c_1} = \ANGLE{b_1}{a_1}{d} \equiv \ANGLE{b_1}{d}{a_1} \equiv \ANGLE{b_2}{a_2}{c_2}\) using Pons Asinorum.
By \hyperref[prop:aas-theorem]{AAS}, we have \(\TRIANGLE{a_1}{b_1}{c_1} \equiv \TRIANGLE{a_2}{b_2}{c_2}\) as needed.
\end{proof}


\begin{construct}[Angle Bisector]
Let \(a\), \(o\), and \(b\) be noncollinear points.
There exists a unique line \(\ell\), containing \(o\), such that if \(u \in \ell\) is different from \(o\) then \(\ANGLE{a}{o}{u} \equiv \ANGLE{b}{o}{u}\).
This line is called the \emph{bisector}\index{bisector} of \(\ANGLE{a}{o}{b}\).
\end{construct}

\begin{proof}
WLOG, we can assume that \(\SEGMENT{o}{a} \equiv \SEGMENT{o}{b}\); if not, construct such a point on \(\RAY{o}{b}\) using Circle Cut.
Since the intersection of \(\CIRCLE{a}{o}\) and \(\CIRCLE{b}{o}\) contains a point not on \(\LINE{a}{b}\), by Interleaved Diameters there is a second point \(u\) on the opposite side of \(\LINE{a}{b}\) such that \(\SEGMENT{a}{u} \equiv \SEGMENT{b}{u}\).
Let \(\ell = \LINE{o}{u}\).
Note that \(\TRIANGLE{a}{o}{u} \equiv \TRIANGLE{b}{o}{u}\) by \hyperref[prop:sss-theorem]{SSS}, so that \(\ANGLE{a}{o}{u} \equiv \ANGLE{b}{o}{u}\).

Now let \(v\) be a point in \(\LINE{o}{u}\) distinct from \(o\).
Certainly if \(v \in \RAY{o}{u}\) then \(\ANGLE{v}{o}{a} \equiv \ANGLE{v}{o}{b}\).
If instead we have \(\BETWEEN{v}{o}{u}\), then \(\ANGLE{v}{o}{a} \equiv \ANGLE{v}{o}{b}\), since these are supplementary to congruent angles.

To see uniqueness, note that any such line must contain \(o\) and \(u\) and thus be equal to \(\LINE{o}{u}\).
\end{proof}

\begin{cor}
The points \(a\) and \(b\) are on opposite sides of the bisector of \(\ANGLE{a}{o}{b}\).
In particular, the bisector of \(\ANGLE{a}{o}{b}\) contains points which are interior to \(\ANGLE{a}{o}{b}\).
\end{cor}

\begin{proof}
Suppose otherwise, and let \(u \neq o\) be a point on the bisector.
Then \(\ANGLE{u}{o}{a}\) and \(\ANGLE{u}{o}{b}\) are congruent angles on the same halfplane of a ray, so that \(a\), \(b\), and \(o\) are collinear -- a contradiction.
Now by the Line Separation property there is a point \(w\) between \(a\) and \(b\) which is on the bisector; this point is interior to \(\ANGLE{a}{o}{b}\) by \lemref{lem:betweenness-separation} as needed.
\end{proof}

\begin{construct}[Segment Midpoint]
Let \(a\) and \(b\) be distinct points.
There is a unique point \(m\) such that \(\BETWEEN{a}{m}{b}\) and \(\SEGMENT{a}{m} \equiv \SEGMENT{b}{m}\).
This point is called the \emph{midpoint}\index{midpoint} of \(\SEGMENT{a}{b}\).
\end{construct}

\begin{proof}
Construct a point \(o\) such that \(\TRIANGLE{a}{o}{b}\) is equilateral, and construct the bisector of \(\ANGLE{a}{o}{b}\).
By the Crossbar Theorem, this bisector must cut \(\SEGMENT{a}{b}\) at an interior point, say \(m\).
Now \(\TRIANGLE{o}{a}{m} \equiv \TRIANGLE{o}{b}{m}\) by \hyperref[prop:sas-theorem]{SAS}, and thus \(\SEGMENT{a}{m} \equiv \SEGMENT{b}{m}\) as needed.
Note that \(m\) is unique by the uniqueness of congruent segments on a ray.
\end{proof}



%---------%
\Exercises%
%---------%

\begin{exercise}
Show that no right angle is also flat or straight.
\end{exercise}
