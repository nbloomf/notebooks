\begin{prop}\label{prop:bisector-foot}
Let \(a\), \(o\), and \(b\) be distinct points.
A point \(p\) not on \(\LINE{a}{o}\) or \(\LINE{b}{o}\) is on the bisector of \(\ANGLE{a}{o}{b}\) if and only if \(\SEGMENT{p}{x} \equiv \SEGMENT{p}{y}\), where \(x\) is the foot of \(p\) on \(\LINE{o}{a}\) and \(y\) is the foot of \(p\) on \(\LINE{o}{b}\).
\end{prop}

\begin{proof}
Suppose \(p\) has this property.
Now \(\TRIANGLE{o}{p}{x}\) and \(\TRIANGLE{o}{p}{y}\) are right, with \(\SEGMENT{p}{x} \equiv \SEGMENT{p}{y}\) and \(\SEGMENT{o}{p} \equiv \SEGMENT{o}{p}\).
By the HL Theorem, \(\TRIANGLE{o}{p}{x} \equiv \TRIANGLE{o}{p}{y}\), and thus \(\ANGLE{x}{o}{p} \equiv \ANGLE{y}{o}{p}\).
So \(p\) is on the bisector of \(\ANGLE{a}{o}{b}\).

Conversely, suppose \(p\) is on the bisector of \(\ANGLE{a}{o}{b}\), and let \(x\) be the foot of \(p\) on \(\LINE{o}{a}\) and \(y\) the foot of \(p\) on \(\LINE{o}{b}\).
Now \(\TRIANGLE{x}{o}{p} \equiv \TRIANGLE{y}{o}{p}\) by AAS, so that \(\SEGMENT{p}{x} \equiv \SEGMENT{p}{y}\).
\end{proof}

\begin{construct}[Incircle Theorem]
Let \(a\), \(b\), and \(c\) be distinct noncollinear points.
Then we have the following.
\begin{proplist}
\item The bisectors of the interior angles of \(\TRIANGLE{a}{b}{c}\) are concurrent at a point \(o\), called the \emph{incenter}\index{incenter} of the triangle, which is interior to \(\TRIANGLE{a}{b}{c}\).

\item The feet of \(o\) on the sides of \(\TRIANGLE{a}{b}{c}\) lie on a circle, called the \emph{incircle}\index{incircle} of \(\TRIANGLE{a}{b}{c}\), which is centered at \(o\) and tangent to the sides of \(\TRIANGLE{a}{b}{c}\).
\end{proplist}
\end{construct}

\begin{proof}
Construct the bisector of \(\ANGLE{b}{a}{c}\).
By the Crossbar Theorem this ray cuts \(\SEGMENT{b}{c}\) at a point, say \(h\).
Now construct the bisector of \(\ANGLE{a}{b}{c}\); again by the Crossbar Theorem this ray cuts \(\SEGMENT{a}{h}\) at a point, say \(o\).
Note that \(o\) is interior to \(\TRIANGLE{a}{b}{c}\).
Let \(x\), \(y\), and \(z\) be the feet of \(o\) on \(\LINE{a}{c}\), \(\LINE{a}{b}\), and \(\LINE{b}{c}\), respectively.
Since \(o\) is on the bisectors of \(\ANGLE{b}{a}{c}\) and \(\ANGLE{a}{b}{c}\), we have \(\SEGMENT{o}{x} \equiv \SEGMENT{o}{y}\) using the AAS Theorem on \(\TRIANGLE{o}{x}{a}\) and \(\TRIANGLE{o}{y}{a}\).
Similarly, \(\SEGMENT{o}{y} \equiv \SEGMENT{o}{z}\).
Thus \(\SEGMENT{o}{x} \equiv \SEGMENT{o}{z}\), and so \(o\) is also on the bisector of \(\ANGLE{B}{C}{A}\) by \ref{prop:bisector-foot}.
Thus the bisectors of the interior angles of \(\TRIANGLE{a}{b}{c}\) are concurrent at \(o\).

Now \(x\), \(y\), and \(z\) are the feet of \(o\) on the sides of \(\TRIANGLE{a}{b}{c}\), and we've seen that \(\SEGMENT{o}{x} \equiv \SEGMENT{o}{y} \equiv \SEGMENT{o}{z}\).
Thus the circle \(\CIRCLE{o}{x}\) contains \(x\), \(y\), and \(z\), and moreover is tangent to the sides of \(\TRIANGLE{a}{b}{c}\) at \(x\), \(y\), and \(z\).
\end{proof}

\begin{construct}[Excircle Theorem]
Let \(a\), \(b\), and \(c\) be distinct noncollinear points.
Then we have the following.
\begin{proplist}
\item The bisector of the interior angle at \(a\) and the exterior angles at \(b\) and \(c\) are concurrent at a point \(o\), called the \emph{excenter}\index{excenter} of \(\TRIANGLE{a}{b}{c}\) at \(a\).

\item The feet of \(o\) on the (extended) sides of \(\TRIANGLE{a}{b}{c}\) lie on a circle, called the \emph{excircle}\index{excircle} of \(\TRIANGLE{a}{b}{c}\) at \(a\), which is centered at \(o\) and tangent to the sides of \(\TRIANGLE{a}{b}{c}\).
\end{proplist}
\end{construct}

\begin{proof}
Essentially the same as the proof of the Incircle Theorem.
\end{proof}

To every triangle we can associate four special circles: the incircle, and one excircle for each vertex.
These circles are tangent to all three (extended) sides of the circle.

\begin{prop}
Any circle which is tangent to all three (extended) sides of a triangle is either the incircle or one of the excircles.
\end{prop}
