Recall that an incidence geometry is called \emph{euclidean} if, given any line \(\ell\) and any point \(p\) not on \(\ell\), there is exactly one line passing through \(p\) which is parallel to \(\ell\).
So far we have avoided using any assumptions about the uniqueness of parallel lines, and have been able to prove a good number of interesting results.
We will now specialize to the Euclidean case for a while.

\begin{prop}[Converse of the Alternate Interior Angles Theorem]
In a Euclidean plane geometry, if two parallel lines are cut by a transversal, then alternate interior angles formed by the cut are congruent.
\end{prop}

\begin{proof}
(copy angle, use AIA, use uniqueness.)
\end{proof}

\begin{prop}
If \(\ell\) and \(m\) are parallel and \(m\) and \(t\) are parallel, then \(\ell\) and \(t\) are parallel.
\end{prop}

\begin{proof}
We can assume that \(\ell\) and \(t\) are distinct (if equal, they are parallel).
Suppose BWOC that \(\ell\) and \(t\) meet at the unique point \(x\).
Since \(\ell\) and \(m\) are parallel, \(x\) is not on \(m\).
By the Euclidean property, there is a unique line \(s\) containing \(x\) which is parallel to \(m\).
But both \(\ell\) and \(t\) satisfy this condition, and they are distinct - a contradiction.
\end{proof}

\begin{cor}
If \(\ell_1\) and \(\ell_2\) are parallel and \(m\) is incident to \(\ell_1\), then \(m\) is incident to \(\ell_2\).
\end{cor}

\begin{prop}
If \(\ell_1\) and \(m\) are perpendicular, and if \(\ell_1\) and \(\ell_2\) are parallel, then \(\ell_2\) and \(m\) are perpendicular.
\end{prop}

\begin{proof}
If \(\ell_1 = \ell_2\) there's nothing to do.
Otherwise \(m\) is a transversal and the result follows from the converse of the AIA theorem.
\end{proof}

\begin{prop}
If \(\ell_1\) and \(\ell_2\) are parallel, \(m_1\) and \(\ell_1\) are perpendicular, and \(m_2\) and \(\ell_2\) are perpendicular, then \(m_1\) and \(m_2\) are parallel.
\end{prop}

\begin{proof}
\(\ell_2\) and \(m_1\) are perpendicular by the converse of AIA, and then \(m_1\) and \(m_2\) are parallel by AIA.
\end{proof}

\begin{construct}
Let \(a\), \(b\), and \(c\) be distinct noncollinear points.
Then the perpendicular bisectors of \(\SEGMENT{a}{b}\), \(\SEGMENT{b}{c}\), and \(\SEGMENT{c}{a}\) are concurrent at a point \(o\), called the \emph{circumcenter}\index{circumcenter} of \(a\), \(b\), and \(c\).
Moreover, \(\CIRCLE{o}{a}\) contains all of \(a\), \(b\), and \(c\) and is unique with this property; this circle is called the \emph{circumcircle}\index{circumcircle} of \(\TRIANGLE{a}{b}{c}\).
\end{construct}

\begin{proof}
Let \(\ell\) be the perpendicular bisector of \(\SEGMENT{a}{b}\) and let \(m\) be the perpendicular bisector of \(\SEGMENT{b}{c}\).
Now \(\ell\) and \(m\) must be incident at a unique point \(o\), since otherwise \(\LINE{a}{b}\) and \(\LINE{b}{c}\) are parallel (which they aren't, as they meet at the unique point \(b\) (since \(a\), \(b\), and \(c\) are not collinear)).
Recall that all points \(x\) on the perpendicular bisector of \(\SEGMENT{a}{b}\) have the property that \(\SEGMENT{a}{x} \equiv \SEGMENT{b}{x}\).
So we have \(\SEGMENT{a}{o} \equiv \SEGMENT{b}{o} \equiv \SEGMENT{c}{o}\).
Thus \(o\) is also on the perpendicular bisector of \(\SEGMENT{c}{a}\), and moreover \(\CIRCLE{o}{a}\) contains \(a\), \(b\), and \(c\).

To see uniqueness, note that the center of any circle containing all of \(a\), \(b\), and \(c\) must be on the perpendicular bisectors of \(\SEGMENT{a}{b}\), \(\SEGMENT{b}{c}\), and \(\SEGMENT{c}{a}\).
\end{proof}

\begin{prop} \mbox{}
\begin{proplist}
\item Opposite angles of a parallelogram are congruent.
\item Opposite sides of a parallelogram are congruent.
\item The diagonals of a parallelogram bisect each other.
\end{proplist}
\end{prop}

\begin{proof}
For the angles, use AIA and converse of AIA.
For the sides, construct a diagonal and use AAS.
For the diagonals, use converse of AIA and ASA.
\end{proof}

\begin{prop}[Thales' Theorem]
Suppose \(a\) and \(b\) are the opposite endpoints of a diameter of a circle centered at \(o\), and that \(c\) is a point on this circle distinct from \(a\) and \(b\).
Then \(\ANGLE{a}{c}{b}\) is right.
Moreover, \(\ANGLE{c}{a}{b}\) is congruent to the bisector of \(\ANGLE{c}{o}{b}\).
\end{prop}

\begin{proof}
Construct the second point \(d\) on the intersection of \(\CIRCLE{o}{a}\) and \(\LINE{o}{c}\) using Circle Cut.
Now \(\SEGMENT{a}{c} \equiv \SEGMENT{b}{d}\) using SAS, and similarly \(\SEGMENT{c}{b} \equiv \SEGMENT{a}{d}\).
Now \(\TRIANGLE{a}{b}{c} \equiv \TRIANGLE{b}{a}{d}\) by SSS, so that \(\ANGLE{c}{b}{a} \equiv \ANGLE{d}{a}{b}\).
Thus \(\LINE{b}{c}\) and \(\LINE{a}{d}\) are parallel by AIA.
Now \(\TRIANGLE{b}{a}{c} \equiv \TRIANGLE{d}{c}{a}\) by SSS, so that \(\ANGLE{b}{c}{a} \equiv \ANGLE{d}{a}{c}\).
Now \(\ANGLE{d}{a}{c}\) and \(\ANGLE{b}{c}{a}\) are supplementary by the converse of AIA.
So \(\ANGLE{b}{c}{a}\) is right.

Now let \(m\) be the point on \(\SEGMENT{b}{c}\) such that \(\RAY{o}{m}\) bisects \(\ANGLE{c}{o}{b}\).
(Use crossbar.)
We have \(\ANGLE{o}{c}{b} \equiv \ANGLE{o}{b}{c}\) by Pons Asinorum, so that \(\ANGLE{o}{m}{c} \equiv \ANGLE{o}{m}{b}\) by ASA.
Thus \(\ANGLE{c}{m}{o}\) is right.
By AIA, \(\LINE{o}{m}\) is parallel to \(\LINE{a}{c}\).
By the converse of AIA, \(\ANGLE{o}{c}{a} \equiv \ANGLE{c}{o}{m}\), and \(\ANGLE{c}{a}{o} \equiv \ANGLE{c}{o}{m}\) by Pons Asinorum.
\end{proof}

\begin{prop}[Converse of Thales' Theorem]
Let \(a\), \(b\), and \(c\) be distinct noncollinear points and let \(m\) be the midpoint of \(\SEGMENT{a}{b}\).
If \(\ANGLE{a}{c}{b}\) is right, then \(c\) lies on \(\CIRCLE{m}{a}\).
\end{prop}

\begin{proof}
Construct the antipode \(d\) of \(c\) through \(m\).
Now \(\SEGMENT{b}{c} \equiv \SEGMENT{a}{d}\) by SAS, and similarly \(\SEGMENT{a}{c} \equiv \SEGMENT{b}{d}\).
So \(\TRIANGLE{a}{b}{c} \equiv \TRIANGLE{b}{a}{d}\) by SSS.
Now \(\LINE{b}{c}\) is parallel to \(\LINE{a}{d}\) by AIA, and so \(\ANGLE{c}{a}{d} \equiv \ANGLE{a}{c}{b}\) using the converse of AIA.
Now \(\TRIANGLE{c}{a}{d} \equiv \TRIANGLE{a}{c}{b}\) by SAS, so that \(\SEGMENT{a}{b} \equiv \SEGMENT{c}{d}\).
Thus \(\SEGMENT{a}{m} \equiv \SEGMENT{c}{m}\).
\end{proof}

(Here we used a lemma that if two segments are congruent, then their midsegments are congruent.)

\begin{construct}[Tangent through an exterior point.]
Given a circle \(\CIRCLE{o}{a}\) and a point \(b\) exterior to this circle, there exist two lines which are tangent to \(\CIRCLE{o}{a}\) and which pass through \(b\).
\end{construct}

\begin{proof}
Construct the midpoint \(m\) of \(\SEGMENT{b}{o}\), and construct circle centered at \(m\) and passing through \(o\).
By the Interleaved Diameters axiom, \(\CIRCLE{m}{o} \cap \CIRCLE{o}{a}\) contains exactly two points, \(x\) and \(y\).
Note that \(\ANGLE{o}{x}{b}\) and \(\ANGLE{o}{y}{b}\) are inscribed on the diameter of a circle, and thus are right; so \(\LINE{b}{x}\) and \(\LINE{b}{y}\) are tangent to \(\CIRCLE{o}{a}\).

Suppose now that there is a third point \(z\) on \(\CIRCLE{o}{a}\), different from \(x\) and \(y\), such that \(\ANGLE{o}{z}{b}\) is right.
By the HL Theorem, we have \(\TRIANGLE{o}{x}{b} \equiv \TRIANGLE{o}{y}{b} \equiv \TRIANGLE{o}{z}{b}\).
In particular, \(\ANGLE{x}{o}{b} \equiv \ANGLE{y}{o}{b} \equiv \ANGLE{z}{o}{b}\).
But now \(z\) is on either the \(x\) side or the \(y\) side of \(\LINE{o}{b}\) (it cannot be on \(\LINE{o}{b}\) since that line is not tangent to \(\CIRCLE{o}{a}\)).
By the uniqueness of angles on a half-plane, either \(x = z\) or \(y = z\).
\end{proof}

\begin{prop}[Inscribed Angle Theorem]
Let \(a\), \(b\), and \(o\) be distinct noncollinear points with \(\SEGMENT{a}{o} \equiv \SEGMENT{b}{o}\).
If \(c\) is a point on \(\CIRCLE{o}{a}\) such that \(c\) and \(o\) are on the same side of \(\LINE{a}{b}\), then \(\ANGLE{a}{c}{b}\) is congruent to a bisector of \(\ANGLE{a}{o}{b}\).
In particular, any two such points form congruent angles.
\end{prop}

\begin{proof}
(@@@)
\end{proof}

\subsection*{Altitudes and the Orthocenter}

\begin{dfn}
Let \(a\), \(b\), and \(c\) be distinct noncollinear points, and let \(f\) be the foot of \(a\) on \(\LINE{b}{c}\).
Then \(\SEGMENT{a}{f}\) is called an \emph{altitude}\index{altitude!of a triangle} of \(\TRIANGLE{a}{b}{c}\).
\end{dfn}

\begin{prop}[Orthocenter Theorem]
Let \(a\), \(b\), and \(c\) be distinct noncollinear points.
Then the lines containing the three altitudes of \(\TRIANGLE{a}{b}{c}\) are concurrent at a point \(o\), called the \emph{orthocenter}\index{orthocenter} of \(\TRIANGLE{a}{b}{c}\).
\end{prop}

\begin{proof}
Construct the feet \(f_a\), \(f_b\), and \(f_c\) of \(a\), \(b\), and \(c\) on \(\SEGMENT{b}{c}\), \(\SEGMENT{a}{c}\), and \(\SEGMENT{a}{b}\), respectively.
Let \(\ell_a\) be the (unique) line passing through \(a\) and parallel to \(\LINE{b}{c}\); similarly, construct lines \(\ell_b\) and \(\ell_c\) passing through \(b\) and \(c\) and parallel to \(\LINE{a}{c}\) and \(\LINE{a}{b}\).
Note that \(\ell_a\) and \(\ell_b\) are incident at a point \(c^\prime\).
Similarly, \(\ell_a\) and \(\ell_c\) are incident at \(b^\prime\) and \(\ell_b\) and \(\ell_c\) are incident at \(a^\prime\).
Note that \(\SEGMENT{c^\prime}{a}\) and \(\SEGMENT{b}{c}\) are opposite sides of a parallelogram, and thus congruent; similarly we have \(\SEGMENT{a}{b^\prime} \equiv \SEGMENT{b}{c}\).
In particular, \(\LINE{a}{f_a}\) is the perpendicular bisector of \(\SEGMENT{b^\prime}{c^\prime}\).
Similarly, \(\LINE{b}{f_b}\) is the perpendicular bisector of \(\SEGMENT{a^\prime}{c^\prime}\) and \(\LINE{c}{f_c}\) the perpendicular bisector of \(\SEGMENT{a^\prime}{b^\prime}\).
That is, the lines containing the altitudes of \(\TRIANGLE{a}{b}{c}\) are the perpendicular bisectors of the sides of \(\TRIANGLE{a^\prime}{b^\prime}{c^\prime}\), and thus are concurrent.
\end{proof}

\subsection*{Medians and the Centroid}

\begin{prop}
Let \(a\), \(b\), and \(c\) be distinct noncollinear points.
Let \(m\) be the midpoint of \(\SEGMENT{a}{b}\), and let \(n\) be the midpoint of \(\SEGMENT{a}{c}\).
Finally, let \(\ell\) be the line parallel to \(\LINE{a}{b}\) through \(n\) and \(t\) the line parallel to \(\LINE{a}{c}\) through \(m\).
Then \(\ell\) and \(m\) meet at the midpoint of \(\SEGMENT{b}{c}\) and \(\LINE{m}{n}\) is parallel to \(\LINE{b}{c}\).
\end{prop}

\begin{proof}
Let \(p\) be the point of intersection of \(\ell\) and \(t\).
Now \(\ANGLE{m}{n}{a} \equiv \ANGLE{n}{m}{p}\) by the converse of AIA; similarly \(\ANGLE{n}{m}{a} \equiv \ANGLE{m}{n}{p}\).
Then \(\TRIANGLE{a}{m}{n} \equiv \TRIANGLE{p}{n}{m}\) by ASA.
In particular we have \(\SEGMENT{m}{p} \equiv \SEGMENT{a}{n}\) and \(\SEGMENT{n}{p} \equiv \SEGMENT{a}{m}\).
Next, using the converse of AIA and the Vertical Angles theorem, we have \(\ANGLE{m}{a}{n} \equiv \ANGLE{b}{m}{p}\) and \(\ANGLE{m}{a}{n} \equiv \ANGLE{p}{n}{c}\).
By SAS we have \(\TRIANGLE{b}{m}{p} \equiv \TRIANGLE{m}{a}{n} \equiv \TRIANGLE{p}{n}{c}\).
Using AIA, then, we have that \(\LINE{b}{p}\) and \(\LINE{p}{c}\) are both parallel to \(\LINE{m}{n}\); by the uniqueness of parallel lines through a point, in face \(b\), \(p\), and \(c\) are collinear, and we see that \(p\) is the midpoint of \(\SEGMENT{b}{c}\) as needed.
\end{proof}

\begin{lem}\label{lem:parallel-bisector}
Let \(a\), \(b\), and \(c\) be distinct collinear points, and \(a\), \(e\), and \(d\) distinct collinear points such that \(e\) is the midpoint of \(\SEGMENT{a}{d}\).
Then \(\LINE{b}{e}\) is parallel to \(\LINE{c}{d}\) if and only if \(\SEGMENT{a}{b} \equiv \SEGMENT{b}{c}\).
\end{lem}

\begin{proof}
Follows from the previous result.
\end{proof}

\begin{cor}
Let \(a\), \(b\), and \(c\) be distinct noncollinear points.
Then the medians of \(\TRIANGLE{a}{b}{c}\) are concurrent at a point \(o\), called the \emph{centroid}\index{centroid} of \(\TRIANGLE{a}{b}{c}\).
\end{cor}

\begin{proof}
Let \(m\) be the midpoint of \(\SEGMENT{a}{b}\) and \(n\) the midpoint of \(\SEGMENT{a}{c}\).
By the previous proposition, \(\LINE{m}{n}\) is parallel to \(\LINE{b}{c}\).
Now let \(o\) be the point of concurrency of \(\SEGMENT{b}{n}\) and \(\SEGMENT{c}{m}\), and using the Crossbar Theorem let \(t\) be the point on \(\SEGMENT{b}{c}\) which is also on \(\RAY{a}{o}\).
We need to show that \(t\) is the midpoint of \(\SEGMENT{b}{c}\).
To this end, construct the line parallel to \(\LINE{b}{m}\) through \(c\), and let \(h\) be the point where this line cuts \(\LINE{a}{o}\).
By the previous lemma, \(o\) is the midpoint of \(\SEGMENT{a}{h}\).
Again using the previous lemma, \(\LINE{b}{h}\) is parallel to \(\LINE{c}{n}\).
In particular, \(b\), \(h\), \(c\), \(o\) form a parallelogram whose diagonals meet at \(t\).
Thus \(t\) is the midpoint of \(\SEGMENT{b}{c}\).
\end{proof}

\begin{prop}[Centroid Trisection]

\end{prop}

\begin{prop}[Euler Line]
Let \(a\), \(b\), and \(c\) be distinct noncollinear points.
Then the orthocenter, centroid, and circumcenter of \(\TRIANGLE{a}{b}{c}\) are collinear.
The line containing these three points is called the \emph{euler line} of \(\TRIANGLE{a}{b}{c}\).
\end{prop}

\begin{proof}
Let \(o\) be the orthocenter and \(d\) the centroid of \(\TRIANGLE{a}{b}{c}\).
Let \(m\) be the midpoint of \(\SEGMENT{o}{d}\), and construct the antipode of \(m\) through \(d\) at \(p\).
We claim that \(p\) lies on the perpendicular bisector of \(\SEGMENT{b}{c}\).
To this end, let \(m_a\) be the midpoint of \(\SEGMENT{b}{c}\).
Note that \(\SEGMENT{m}{d} \equiv \SEGMENT{d}{p}\) by construction.
Now constuct the parallel to \(\LINE{a}{o}\) through \(m\), cutting \(\LINE{a}{d}\) at \(h\).
By the lemma, since \(\SEGMENT{o}{m} \equiv \SEGMENT{m}{d}\) we have \(\SEGMENT{a}{h} \equiv \SEGMENT{h}{d}\).
By the Centroid Trisection theorem we have \(\SEGMENT{h}{d} \equiv \SEGMENT{d}{m_a}\).
Note that \(\ANGLE{h}{d}{o} \equiv \ANGLE{m_a}{d}{p}\), and by SAS, \(\TRIANGLE{h}{d}{o} \equiv \TRIANGLE{m_a}{d}{p}\).
Thus \(\ANGLE{m}{h}{d} \equiv \ANGLE{m_a}{p}{d}\).
By AIA, we have \(\LINE{p}{m_a}\) parallel to \(\LINE{h}{m}\), so that \(\LINE{p}{m_a}\) is the perpendicular bisector of \(\SEGMENT{b}{c}\).
That is, \(p\) lies on the perpendicular bisector of \(\SEGMENT{b}{c}\).
Similar reasoning shows that \(p\) lies on the perpendicular bisectors of \(\SEGMENT{a}{b}\) and \(\SEGMENT{a}{c}\) as well; thus \(p\) is the circumcenter of \(\TRIANGLE{a}{b}{c}\).
In particular, the orthocenter, centroid, and circumcenter of \(\TRIANGLE{a}{b}{c}\) are collinear.
\end{proof}


\begin{prop}[9 Point Circle]
Let \(a\), \(b\), and \(c\) be distinct noncollinear points, and let \(o\) be the orthocenter of \(\TRIANGLE{a}{b}{c}\).
Then the following points all lie on a common circle:
\begin{proplist}
\item The midpoints \(m_a\), \(m_b\), and \(m_c\) of \(\SEGMENT{b}{c}\), \(\SEGMENT{a}{c}\), and \(\SEGMENT{a}{b}\),
\item The feet \(h_a\), \(h_b\), and \(h_c\) of \(a\) on \(\LINE{b}{c}\), \(b\) on \(\LINE{a}{c}\), and \(c\) on \(\LINE{a}{b}\), and
\item The midpoints \(n_a\), \(n_b\), and \(n_c\) of \(\SEGMENT{a}{o}\), \(\SEGMENT{b}{o}\), and \(\SEGMENT{c}{o}\).
\end{proplist}
\end{prop}

\begin{proof}
Note that \(\LINE{m_b}{m_c}\) is parallel to \(\LINE{b}{c}\), which is parallel to \(\LINE{n_b}{n_c}\) using \lemref{lem:parallel-bisector}.
Similarly, we have \(\LINE{n_b}{m_c}\) parallel to \(\LINE{a}{o}\) parallel to \(\LINE{n_c}{m_b}\).
Since \(\LINE{b}{c}\) is perpendicular to \(\LINE{a}{o}\), in fact \(m_c\), \(m_b\), \(n_c\), and \(n_b\) form a rectangle.
Thus \(m_c\), \(m_b\), \(n_c\), and \(n_b\) lie on a common circle having diameters \(\SEGMENT{m_b}{n_b}\) and \(\SEGMENT{m_c}{n_c}\).
By similar reasoning, \(m_c\), \(m_a\), \(n_c\), and \(n_a\) lie on a common circle having diameters \(\SEGMENT{m_a}{n_a}\) and \(\SEGMENT{m_c}{n_c}\).
Because every segment is the diameter of a unique circle, in fact all six points \(m_i\) and \(n_i\) lie on a common circle.
By the converse of Thales' Theorem, \(h_a\), \(h_b\), and \(h_c\) also lie on this circle as claimed.
\end{proof}
