\begin{prop}[Supplements are unique] \mbox{}
\begin{proplist}
\item Suppose that \(\ANGLE{a_1}{o_1}{b_1}\) and \(\ANGLE{b_1}{o_1}{c_1}\) are a linear pair, and that \(\ANGLE{a_2}{o_2}{b_2}\) and \(\ANGLE{b_2}{o_2}{c_2}\) are a linear pair.
If \(\ANGLE{a_1}{o_1}{b_1} \equiv \ANGLE{a_2}{o_2}{b_2}\), then \(\ANGLE{b_1}{o_1}{c_1} \equiv \ANGLE{b_2}{o_2}{c_2}\).

\begin{center}
\begin{tikzpicture}[scale=0.8]
  \coordinate [label=below:\(o_1\)] (o1) at (-3,0);
  \coordinate [label=below:\(a_1\)] (a1) at ($ (o1)+(180:1) $);
  \coordinate [label=left:\(b_1\)] (b1) at ($ (o1)+(70:1) $);
  \coordinate [label=below:\(c_1\)] (c1) at ($ (o1)+(0:1) $);

  \draw [fill] (o1) circle [radius=1pt];
  \draw [fill] (a1) circle [radius=1pt];
  \draw [fill] (b1) circle [radius=1pt];
  \draw [fill] (c1) circle [radius=1pt];

  \draw [->] (o1) -- ($ (a1)+(180:0.5) $);
  \draw [->] (o1) -- ($ (b1)+(70:0.5) $);
  \draw [->] (o1) -- ($ (c1)+(0:0.5) $);

  \coordinate [label=below:\(o_2\)] (o2) at (2,0);
  \coordinate [label=below:\(a_2\)] (a2) at ($ (o2)+(180:1) $);
  \coordinate [label=left:\(b_2\)] (b2) at ($ (o2)+(80:1) $);
  \coordinate [label=below:\(c_2\)] (c2) at ($ (o2)+(0:1) $);

  \draw [fill] (o2) circle [radius=1pt];
  \draw [fill] (a2) circle [radius=1pt];
  \draw [fill] (b2) circle [radius=1pt];
  \draw [fill] (c2) circle [radius=1pt];

  \draw [->] (o2) -- ($ (a2)+(180:0.5) $);
  \draw [->] (o2) -- ($ (b2)+(80:0.5) $);
  \draw [->] (o2) -- ($ (c2)+(0:0.5) $);
\end{tikzpicture}
\end{center}

\item Suppose \(\alpha\), \(\beta\), and \(\gamma\) are angles such that \(\alpha\) and \(\beta\) are supplementary and \(\alpha\) and \(\gamma\) are supplementary.
Then \(\alpha \equiv \gamma\).
\end{proplist}
\end{prop}

\begin{proof}\mbox{}
\begin{proplist}
\item Suppose we have two such linear pairs.
Without loss of generality, we can suppose that \[ \SEGMENT{o_1}{a_1} \equiv \SEGMENT{o_1}{b_1} \equiv \SEGMENT{o_1}{c_1} \equiv \SEGMENT{o_2}{a_2} \equiv \SEGMENT{o_2}{b_2} \equiv \SEGMENT{o_2}{c_2}. \] (If they aren't, we can use Circle Cut and the Segment Copy construction to find such points.) Now \(\TRIANGLE{b_1}{o_1}{a_1} \equiv \TRIANGLE{b_2}{o_2}{a_2}\) by SAS, so that \(\ANGLE{b_1}{a_1}{o_1} \equiv \ANGLE{b_2}{a_2}{o_2}\).
Now \(\SEGMENT{a_1}{c_1} \equiv \SEGMENT{a_2}{c_2}\), so that \(\TRIANGLE{b_1}{a_1}{c_1} \equiv \TRIANGLE{b_2}{a_2}{c_2}\) by SAS.
So \(\SEGMENT{b_1}{c_1} \equiv \SEGMENT{b_2}{c_2}\), and thus \(\TRIANGLE{b_1}{o_1}{c_1} \equiv \TRIANGLE{b_2}{o_2}{c_2}\) by SSS.
Thus \(\ANGLE{b_1}{o_1}{c_1} \equiv \ANGLE{b_2}{o_2}{c_2}\) as needed.

\item Follows from the definition of supplementary.
\qedhere
\end{proplist}
\end{proof}

\begin{cor}
If angles \(\alpha\) and \(\beta\) are a vertical pair, then \(\alpha \equiv \beta\).
\end{cor}

\begin{proof}
If \(\ANGLE{a}{o}{b}\) and \(\ANGLE{c}{o}{d}\) are a vertical pair, with \(\BETWEEN{b}{o}{c}\) and \(\BETWEEN{a}{o}{d}\), then both \(\ANGLE{a}{o}{b}\) and \(\ANGLE{c}{o}{d}\) are supplementary to \(\ANGLE{a}{o}{c}\).
\end{proof}


\begin{dfn}[Transversal]
Suppose we have three lines \(\ell_1\), \(\ell_2\), and \(t\) in a plane geometry.
We say that \(t\) is a \emph{transversal}\index{transversal} of \(\ell_1\) and \(\ell_2\) if \(t\) cuts both \(\ell_1\) and \(\ell_2\) at unique points, and these points are distinct.

Suppose \(t\) is a transversal of \(\ell_1\) and \(\ell_2\), cutting these lines at \(o_1\) and \(o_2\), respectively as shown.

\begin{center}
\begin{tikzpicture}[scale=0.5]
  \coordinate (x1) at (-5,0);
  \coordinate (y1) at (4,0);
  \draw [<->] (x1) -- (y1);
  \node at (4,-0.5) {\(\ell_1\)};
  \coordinate (x2) at (-4,5);
  \coordinate (y2) at (4,3);
  \draw [<->] (x2) -- (y2);
  \node  at (-3,5.5) {\(\ell_2\)};
  \coordinate (x3) at (-1,-1);
  \coordinate (y3) at (1,6);
  \draw [<->] (x3) -- (y3);
  \node at (0.7,6) {\(t\)};
  \coordinate [label=below right:\(o_1\)] (o1) at (intersection of x1--y1 and x3--y3);
  \draw [fill] (o1) circle [radius=1pt];
  \coordinate [label=below left:\(o_2\)] (o2) at (intersection of x2--y2 and x3--y3);
  \draw [fill] (o2) circle [radius=1pt];
  \coordinate (x4) at (-0.5,2);
  \coordinate (y4) at (1,3);
  \coordinate [label=below:\(a\)] (a) at (intersection of x1--y1 and x4--y4);
  \draw [fill] (a) circle [radius=1pt];
  \coordinate [label=below:\(b\)] (b) at (intersection of x2--y2 and x4--y4);
  \draw [fill] (b) circle [radius=1pt];
\end{tikzpicture}
\end{center}

If \(a\) is on \(\ell_1\) and \(b\) is on \(\ell_2\) such that \(a\) and \(b\) are on opposite sides of \(t\), then we say that \(\ANGLE{a}{o_1}{o_2}\) and \(\ANGLE{b}{o_2}{o_1}\) are \emph{alternate interior angles}\index{alternate interior angles} of this transversal.
\end{dfn}

Every transversal has two pairs of alternate interior angles.

\begin{prop}[Alternate Interior Angles]\label{prop:aia-theorem}
If two lines \(\ell_1\) and \(\ell_2\) are cut by a transversal \(t\) so that a pair of alternate interior angles are congruent, then \(\ell_1\) and \(\ell_2\) are parallel.
\end{prop}

\begin{proof}
Suppose \(t\) meets \(\ell_1\) and \(\ell_2\) at points \(o_1\) and \(o_2\) respectively, and that \(a\) and \(b\) are on \(\ell_1\) and \(\ell_2\), respectively, and on opposite sides of \(t\), such that \(\ANGLE{a}{o_1}{o_2} \equiv \ANGLE{b}{o_2}{o_1}\).
Now suppose by way of contradiction that \(\ell_1\) and \(\ell_2\) are \emph{not} parallel; rather, they meet at a point \(x\) which (WLOG) is on the \(a\)-side of \(t\).
Let \(c\) be on \(\ell_1\) such that \(\BETWEEN{a}{o_1}{c}\).
Copy \(\SEGMENT{o_1}{x}\) onto \(\RAY{o_2}{b}\) at the point \(y\).
Note that \(\SEGMENT{o_1}{x} \equiv \SEGMENT{o_2}{y}\), \(\SEGMENT{o_1}{o_2} \equiv \SEGMENT{o_2}{o_1}\), and \(\ANGLE{x}{o_1}{o_2} \equiv \ANGLE{y}{o_2}{o_1}\); by \hyperref[prop:sas-theorem]{SAS} we thus have \(\TRIANGLE{x}{o_1}{o_2} \equiv \TRIANGLE{y}{o_2}{o_1}\).
In particular, \(\ANGLE{o_2}{o_1}{y} \equiv \ANGLE{o_1}{o_2}{x}\).

\begin{center}
\begin{tikzpicture}[scale=0.8]
  \coordinate [label=above right:\(o_1\)] (o1) at (0,2);
  \coordinate [label=below right:\(o_2\)] (o2) at ($ (o1)+(270:2) $);
  \coordinate [label=above:\(a\)] (a) at ($ (o1)+(160:-1) $);
  \coordinate [label=above:\(c\)] (c) at ($ (o1)+(160:1) $);
  \coordinate [label=above:\(b\)] (b) at ($ (o2)+(200:1) $);
  \coordinate [label=below:\(x\)] (x) at (intersection of o1--a and o2--b);
  \coordinate [label=above:\(y\)] (y) at ($ (o2)+(200:1.5) $);

  \draw [fill] (o1) circle [radius=1pt];
  \draw [fill] (o2) circle [radius=1pt];
  \draw [fill] (a) circle [radius=1pt];
  \draw [fill] (c) circle [radius=1pt];
  \draw [fill] (b) circle [radius=1pt];
  \draw [fill] (x) circle [radius=1pt];
  \draw [fill] (y) circle [radius=1pt];

  \draw [<->] ($ (o1)+(90:0.5) $) -- ($ (o2)+(270:0.5) $);
  \draw [<->] ($ (y)+(200:0.5) $) -- ($ (x)+(200:-0.5) $);
  \draw [<->] ($ (c)+(160:0.5) $) -- ($ (x)+(160:-0.5) $);
\end{tikzpicture}
\end{center}

Note that \(\ANGLE{x}{o_2}{o_1}\) and \(\ANGLE{o_1}{o_2}{y}\) are supplementary, and \(\ANGLE{o_1}{o_2}{y} \equiv \ANGLE{a}{o_1}{o_2}\) by hypothesis, so that \(\ANGLE{a}{o_1}{o_2}\) and \(\ANGLE{x}{o_2}{o_1}\) are supplementary.
Since \(\ANGLE{x}{o_2}{o_1} \equiv \ANGLE{y}{o_1}{o_2}\), we have that \(\ANGLE{a}{o_1}{o_2}\) and \(\ANGLE{y}{o_1}{o_2}\) are supplementary.
But also \(\ANGLE{a}{o_1}{o_2}\) and \(\ANGLE{o_2}{o_1}{c}\) are supplementary.
Thus \(\ANGLE{o_2}{o_1}{y} \equiv \ANGLE{o_2}{o_1}{c}\).
By AC4, we have that \(o_1\), \(c\), and \(y\) are collinear, so that \(y \in \ell_1\).
But now \(\ell_1\) and \(\ell_2\) have two points in common -- \(x\) and \(y\) -- which are necessarily distinct as they are on opposite halfplanes of \(t\).
So we have \(\ell_1 = \ell_2\), a contradiction.

Thus \(\ell_1\) and \(\ell_2\) must be parallel. 
\end{proof}

\begin{prop}[AAS Theorem]\label{prop:aas-theorem}
Suppose we have triangles \(\TRIANGLE{A}{B}{C}\) and \(\TRIANGLE{X}{Y}{Z}\) such that \(\ANGLE{C}{A}{B} \equiv \ANGLE{Z}{X}{Y}\), \(\ANGLE{A}{B}{C} \equiv \ANGLE{X}{Y}{Z}\), and \(\SEGMENT{B}{C} \equiv \SEGMENT{Y}{Z}\).
Then \(\TRIANGLE{A}{B}{C} \equiv \TRIANGLE{X}{Y}{Z}\).
\end{prop}

\begin{proof}
Copy \(\SEGMENT{B}{A}\) onto \(\RAY{Y}{X}\) at the point \(W\).
Note that \(\TRIANGLE{W}{Y}{Z} \equiv \TRIANGLE{A}{B}{C}\) by SAS, so that \(\ANGLE{B}{A}{C} \equiv \TRIANGLE{Y}{W}{Z}\).
Suppose now that \(W\) and \(X\) are distinct points.
In this case \(\LINE{X}{Z}\) and \(\LINE{W}{Z}\) are lines cut by a transversal \(\LINE{X}{Y}\).
Moreover, if we let \(U\) be a point such that \(\BETWEEN{U}{X}{Z}\), then \(\ANGLE{U}{X}{W}\) and \(\ANGLE{Y}{X}{Z}\) are vertical, hence congruent, and so \(\ANGLE{U}{X}{W} \equiv \ANGLE{Y}{X}{Z}\).
But now by the Alternate Interior Angles theorem \(\LINE{X}{Z}\) and \(\LINE{W}{Z}\) must be parallel, a contradiction since they meet at \(Z\).
So in fact \(X\) and \(W\) are the same point, and thus \(\TRIANGLE{A}{B}{C} \equiv \TRIANGLE{X}{Y}{Z}\) by SAS.
\end{proof}

\begin{prop}[HL Theorem]
Let \(\TRIANGLE{A}{B}{C}\) and \(\TRIANGLE{X}{Y}{Z}\) be triangles such that \(\ANGLE{B}{C}{A}\) and \(\ANGLE{Y}{Z}{X}\) are right and \(\SEGMENT{A}{B} \equiv \SEGMENT{X}{Y}\) and \(\SEGMENT{B}{C} \equiv \SEGMENT{Y}{Z}\).
Then \(\TRIANGLE{A}{B}{C} \equiv \TRIANGLE{X}{Y}{Z}\).
\end{prop}

\begin{proof}
Copy \(\SEGMENT{Z}{X}\) onto the ray opposite \(\RAY{C}{A}\) at the point \(D\).
Now \(\ANGLE{B}{C}{D}\) is a right angle, since it is supplementary to \(\ANGLE{A}{C}{B}\).
By SAS, we have \(\TRIANGLE{X}{Y}{Z} \equiv \TRIANGLE{D}{C}{B}\), and thus \(\SEGMENT{B}{D} \equiv \SEGMENT{Y}{X} \equiv \SEGMENT{B}{A}\).
Now \(\TRIANGLE{A}{B}{D}\) is isoceles with \(\SEGMENT{B}{A} \equiv \SEGMENT{B}{D}\), so that \(\ANGLE{B}{A}{C} \equiv \ANGLE{B}{A}{D} \equiv \ANGLE{B}{D}{A} \equiv \ANGLE{Y}{X}{Z}\).
By AAS, we have \(\TRIANGLE{A}{B}{C} \equiv \TRIANGLE{X}{Y}{Z}\).
\end{proof}

\begin{prop}
A triangle formed by three noncollinear points cannot have two interior angles which are both right.
\end{prop}

\begin{proof}
Such a triangle would violate the Alternate Interior Angles theorem since right angles are self-supplementary, and any two right angles are congruent.
\end{proof}


\begin{construct}[Angle Bisector]
Let \(A\), \(O\), and \(B\) be noncollinear points.
There exists a unique line \(\ell\), containing \(O\), such that if \(U \in \ell\) is different from \(O\) then \(\ANGLE{A}{O}{U} \equiv \ANGLE{B}{O}{U}\).
This line is called the \emph{bisector} of \(\ANGLE{A}{O}{B}\).
\end{construct}

\begin{proof}
Note that we can assume WLOG that \(\SEGMENT{O}{A} \equiv \SEGMENT{O}{B}\); if not, construct such a point on \(\RAY{O}{B}\) using the Circle Separation property.
Since the intersection of \(\CIRCLE{A}{O}\) and \(\CIRCLE{B}{O}\) contains a point not on \(\LINE{A}{B}\), by Circle Cut Transfer there is a second point \(U\) on the opposite side of \(\LINE{A}{B}\) such that \(\SEGMENT{A}{U} \equiv \SEGMENT{B}{U}\).
Let \(\ell = \LINE{O}{U}\).
Note that \(\TRIANGLE{A}{O}{U} \equiv \TRIANGLE{B}{O}{U}\) by SSS, so that \(\ANGLE{A}{O}{U} \equiv \ANGLE{B}{O}{U}\).
Then if \(V\) is a point such that \(\BETWEEN{V}{O}{U}\), we have \(\ANGLE{V}{O}{A} \equiv \ANGLE{V}{O}{B}\), since these are supplementary to congruent angles.

To see uniqueness, note that any such line must contain \(O\) and \(U\).
\end{proof}

\begin{cor}
\(A\) and \(B\) are on opposite sides of the bisector of \(\ANGLE{A}{O}{B}\).
In particular, the bisector of \(\ANGLE{A}{O}{B}\) contains points which are interior to \(\ANGLE{A}{O}{B}\).
\end{cor}

\begin{proof}
Suppose otherwise, and let \(U \neq O\) be a point on the bisector.
Then \(\ANGLE{U}{O}{A}\) and \(\ANGLE{U}{O}{B}\) are congruent angles on the same half-plane of a ray, so that \(A\), \(B\), and \(O\) are collinear -- a contradiction.
By the plane separation property there is a point \(W\) between \(A\) and \(B\) which is on the bisector; this point is interior to \(\ANGLE{A}{O}{B}\) as needed.
\end{proof}

\begin{construct}[Segment Midpoint]
Let \(A\) and \(B\) be distinct points.
There is a unique point \(M\) such that \(\BETWEEN{A}{M}{B}\) and \(\SEGMENT{A}{M} \equiv \SEGMENT{B}{M}\).
This point is called the \emph{midpoint} of \(\SEGMENT{A}{B}\).
\end{construct}

\begin{proof}
Construct a point \(O\) such that \(\TRIANGLE{A}{O}{B}\) is equilateral, and construct the bisector of \(\ANGLE{A}{O}{B}\).
By the Crossbar theorem, this bisector must cut \(\SEGMENT{A}{B}\) at an interior point, say \(M\).
Now \(\TRIANGLE{O}{A}{M} \equiv \TRIANGLE{O}{B}{M}\) by SAS, and thus \(\SEGMENT{A}{M} \equiv \SEGMENT{B}{M}\) as needed.
Note that \(M\) is unique by the uniqueness of congruent segments on a ray.
\end{proof}
