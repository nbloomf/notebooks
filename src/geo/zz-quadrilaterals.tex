\begin{dfn}[Quadrilateral]
Let \(A\), \(B\), \(C\), and \(D\) be points in a plane geometry.
Then the set \[ \QUADRILATERAL{A}{B}{C}{D} = \SEGMENT{A}{B} \cup \SEGMENT{B}{C} \cup \SEGMENT{C}{D} \cup \SEGMENT{D}{A} \] is called the \emph{quadrilateral} with \emph{vertices} \(A\), \(B\), \(C\), and \(D\) (in that order).
\begin{itemize}
\item \(\SEGMENT{A}{B}\), \(\SEGMENT{B}{C}\), \(\SEGMENT{C}{D}\), and \(\SEGMENT{D}{A}\) are called the \emph{sides} of the quadrilateral.
\item \(\SEGMENT{A}{C}\) and \(\SEGMENT{B}{D}\) are called the \emph{diagonals}.
\item \(\ANGLE{A}{B}{C}\), \(\ANGLE{B}{C}{D}\), \(\ANGLE{C}{D}{A}\), and \(\ANGLE{D}{A}{B}\) are the \emph{interior angles}
\item Sides \(\SEGMENT{A}{B}\) and \(\SEGMENT{C}{D}\) are said to be \emph{opposite}, as are \(\SEGMENT{B}{C}\) and \(\SEGMENT{D}{A}\).
\end{itemize}
\end{dfn}

\begin{dfn}
Let \(A\), \(B\), \(C\), and \(D\) be points.
\begin{itemize}
\item If any two of \(A\), \(B\), \(C\), and \(D\) are equal, then \(\QUADRILATERAL{A}{B}{C}{D}\) is said to be \emph{backtracking}.
\item \(\QUADRILATERAL{A}{B}{C}{D}\) is called \emph{self-intersecting} if any point other than the vertices is on more than one side.
\item \(\QUADRILATERAL{A}{B}{C}{D}\) is called \emph{degenerate} if any three of its vertices are collinear.
\item A quadrilateral which is not backtracking, self-intersecting, or degenerate is called \emph{simple}.
\end{itemize}
\end{dfn}

\begin{dfn}
We say that \(\QUADRILATERAL{A}{B}{C}{D} \equiv \QUADRILATERAL{X}{Y}{Z}{W}\) if \(\SEGMENT{A}{B} \equiv \SEGMENT{X}{Y}\), \(\SEGMENT{B}{C} \equiv \SEGMENT{Y}{Z}\), \(\SEGMENT{C}{D} \equiv \SEGMENT{Z}{W}\), \(\SEGMENT{D}{A} \equiv \SEGMENT{W}{X}\), \(\ANGLE{A}{B}{C} \equiv \ANGLE{X}{Y}{Z}\), \(\ANGLE{B}{C}{D} \equiv \ANGLE{Y}{Z}{W}\), \(\ANGLE{C}{D}{A} \equiv \ANGLE{Z}{W}{X}\), and \(\ANGLE{D}{A}{B} \equiv \ANGLE{W}{X}{Y}\).
\end{dfn}

\begin{prop}
Quadrilateral congruence is an equivalence relation.
\end{prop}

\subsection*{Types of Quadrilaterals}

\begin{dfn}
A quadrilateral \(\QUADRILATERAL{A}{B}{C}{D}\) is called
\begin{itemize}
\item \emph{equiangular} if all its interior angles are congruent;
\item a \emph{rectangle} if all its interior angles are right;
\item a \emph{kite} if it has two pairs of congruent adjacent sides;
\item \emph{equilateral} (aka a \emph{rhombus}) if all its sides are congruent;
\item a \emph{trapezoid} if one pair of opposite sides is parallel;
\item a \emph{parallelogram} if both pairs of opposite sides are parallel;
\item \emph{cyclic} if all its vertices lie on a common circle;
\item \emph{tangential} if all extended sides are tangent to a common circle;
\item \emph{regular} if it is both equilateral and equiangular.
\end{itemize}
\end{dfn}

Note: not all of these types of quadrilaterals are guaranteed to exist!

