\documentclass{article}
\usepackage{neb-macros}
\usepackage{tikz}
  \usetikzlibrary{calc,intersections}

\begin{document}

\CheapTitle{Transversals}

\begin{prop}[Supplements are unique] \mbox{}
\begin{itemize}
\item Suppose that $\Angle{A}{O}{B}$ and $\Angle{B}{O}{C}$ are a linear pair, and that $\Angle{X}{P}{Y}$ and $\Angle{Y}{P}{Z}$ are a linear pair. If $\Angle{A}{O}{B} \equiv \Angle{X}{P}{Y}$, then $\Angle{B}{O}{C} \equiv \Angle{Y}{P}{Z}$.
\item Suppose $\Angle{A}{B}{C}$ and $\Angle{X}{Y}{Z}$ are supplementary, and that $\Angle{A}{B}{C}$ and $\Angle{H}{K}{L}$ are supplementary. Then $\Angle{X}{Y}{Z} \equiv \Angle{H}{K}{L}$.
\end{itemize}
\end{prop}

\begin{proof}
Suppose we have two such linear pairs. Without loss of generality, we can suppose that \[ \Segment{O}{A} \equiv \Segment{O}{B} \equiv \Segment{O}{C} \equiv \Segment{P}{X} \equiv \Segment{P}{Y} \equiv \Segment{P}{Z}. \] (If they aren't, we can use Circle Separation and the Segment Copy construction to find such points.) Now $\Triangle{B}{O}{A} \equiv \Triangle{Y}{P}{X}$ by SAS, so that $\Angle{B}{A}{O} \equiv \Angle{Y}{X}{P}$. Now $\Segment{A}{C} \equiv \Segment{X}{Z}$, so that $\Triangle{B}{A}{C} \equiv \Triangle{Y}{X}{Z}$ by SAS. So $\Segment{B}{C} \equiv \Segment{Y}{Z}$, and thus $\Triangle{B}{O}{C} \equiv \Triangle{Y}{P}{Z}$ by SSS. Thus $\Angle{B}{O}{C} \equiv \Angle{Y}{P}{Z}$.

The second statement follows easily.
\end{proof}

\begin{cor}
Vertical pairs of angles are congruent.
\end{cor}

\section*{Transversals}

\begin{dfn}[Transversal]
Suppose we have three lines $\ell_1$, $\ell_2$, and $t$ in a plane geometry. We say that $t$ is a \emph{transversal} of $\ell_1$ and $\ell_2$ if $t$ cuts both $\ell_1$ and $\ell_2$ at unique points, and these points are distinct.
\end{dfn}

Suppose $t$ is a transversal of $\ell_1$ and $\ell_2$, cutting these lines at $O_1$ and $O_2$, respectively as shown.

\begin{center}
\begin{tikzpicture}[scale=0.6]
  \coordinate (x1) at (-5,0);
  \coordinate (y1) at (4,0);
  \draw [<->] (x1) -- (y1);
  \node at (4,-0.5) {$\ell_1$};

  \coordinate (x2) at (-4,5);
  \coordinate (y2) at (4,3);
  \draw [<->] (x2) -- (y2);
  \node  at (-3,5.5) {$\ell_2$};

  \coordinate (x3) at (-1,-1);
  \coordinate (y3) at (1,6);
  \draw [<->] (x3) -- (y3);
  \node at (0.7,6) {$t$};

  \coordinate [label=below right:$O_1$] (o1) at (intersection of x1--y1 and x3--y3);
  \draw [fill] (o1) circle [radius=1pt];

  \coordinate [label=below left:$O_2$] (o2) at (intersection of x2--y2 and x3--y3);
  \draw [fill] (o2) circle [radius=1pt];

  \coordinate (x4) at (-0.5,2);
  \coordinate (y4) at (1,3);

  \coordinate [label=below:$A$] (a) at (intersection of x1--y1 and x4--y4);
  \draw [fill] (a) circle [radius=1pt];
  \coordinate [label=below:$B$] (b) at (intersection of x2--y2 and x4--y4);
  \draw [fill] (b) circle [radius=1pt];
\end{tikzpicture}
\end{center}

If $A$ is on $\ell_1$ and $B$ is on $\ell_2$ such that $A$ and $B$ are on opposite sides of $t$, then we say that $\Angle{A}{O_1}{O_2}$ and $\Angle{B}{O_2}{O_1}$ are \emph{alternate interior angles} of this transversal.

\begin{prop}[Alternate Interior Angles]
If two lines $\ell_1$ and $\ell_2$ are cut by a transversal $t$ so that a pair of alternate interior angles are congruent, then $\ell_1$ and $\ell_2$ are parallel.
\end{prop}

\begin{proof}
Suppose $t$ meets $\ell_1$ and $\ell_2$ at points $O_1$ and $O_2$ respectively, and that $A$ and $B$ are on $\ell_1$ and $\ell_2$, respectively, and on opposite sides of $t$. Let $C$ be on $\ell_1$ such that $\Between{A}{O_1}{C}$. Suppose by way of contradiction that $\ell_1$ and $\ell_2$ are \emph{not} parallel; rather, they meet at a point $X$ which (WLOG) is on the $A$-side of $t$. Copy $\Segment{O_1}{X}$ onto $\Ray{O_2}{B}$ at the point $Y$. Now $\Segment{O_1}{X} \equiv \Segment{O_2}{Y}$, $\Segment{O_1}{O_2} \equiv \Segment{O_2}{O_1}$, and $\Angle{X}{O_1}{O_2} \equiv \Angle{Y}{O_2}{O_1}$, so by SAS we have $\Triangle{X}{O_1}{O_2} \equiv \Triangle{Y}{O_2}{O_1}$. In particular, $\Angle{O_2}{O_1}{Y} \equiv \Angle{O_1}{O_2}{X}$.

Now $\Angle{X}{O_2}{O_1}$ and $\Angle{O_1}{O_2}{Y}$ are supplementary, and $\Angle{O_1}{O_2}{Y} \equiv \Angle{A}{O_1}{O_2}$, so that $\Angle{A}{O_1}{O_2}$ and $\Angle{X}{O_2}{O_1}$ are supplementary. Since $\Angle{X}{O_2}{O_1} \equiv \Angle{Y}{O_1}{O_2}$, we have that $\Angle{A}{O_1}{O_2}$ and $\Angle{Y}{O_1}{O_2}$ are supplementary. But also $\Angle{A}{O_1}{O_2}$ and $\Angle{O_2}{O_1}{C}$ are supplementary. Now $\Angle{O_2}{O_1}{Y} \equiv \Angle{O_2}{O_1}{C}$. By the uniqueness of congruent angles on a half-plane, we have that $O_1$, $C$, and $Y$ are collinear, so that $Y \in \ell_1$. But now $\ell_1$ and $ell_2$ have two points in common -- $X$ and $Y$ -- and thus must be equal, a contradiction.

So in fact $\ell_1$ and $\ell_2$ must be parallel.  
\end{proof}

\begin{prop}[AAS]
Suppose we have triangles $\Triangle{A}{B}{C}$ and $\Triangle{X}{Y}{Z}$ such that $\Angle{C}{A}{B} \equiv \Angle{Z}{X}{Y}$, $\Angle{A}{B}{C} \equiv \Angle{X}{Y}{Z}$, and $\Segment{B}{C} \equiv \Segment{Y}{Z}$. Then $\Triangle{A}{B}{C} \equiv \Triangle{X}{Y}{Z}$.
\end{prop}

\begin{proof}
Copy $\Segment{B}{A}$ onto $\Ray{Y}{X}$ at the point $W$. Note that $\Triangle{W}{Y}{Z} \equiv \Triangle{A}{B}{C}$ by SAS, so that $\Angle{B}{A}{C} \equiv \Triangle{Y}{W}{Z}$. Suppose now that $W$ and $X$ are distinct points. In this case $\Line{X}{Z}$ and $\Line{W}{Z}$ are lines cut by a transversal $\Line{X}{Y}$. Moreover, if we let $U$ be a point such that $\Between{U}{X}{Z}$, then $\Angle{U}{X}{W}$ and $\Angle{Y}{X}{Z}$ are vertical, hence congruent, and so $\Angle{U}{X}{W} \equiv \Angle{Y}{X}{Z}$. But now by the Alternate Interior Angles theorem $\Line{X}{Z}$ and $\Line{W}{Z}$ must be parallel, a contradiction since they meet at $Z$.

So in fact $X$ and $W$ are the same point, and thus $\Triangle{A}{B}{C} \equiv \Triangle{X}{Y}{Z}$ by SAS.
\end{proof}

\begin{prop}[HL]
Let $\Triangle{A}{B}{C}$ and $\Triangle{X}{Y}{Z}$ be triangles such that $\Angle{B}{C}{A}$ and $\Angle{Y}{Z}{X}$ are right and $\Segment{A}{B} \equiv \Segment{X}{Y}$ and $\Segment{B}{C} \equiv \Segment{Y}{Z}$. Then $\Triangle{A}{B}{C} \equiv \Triangle{X}{Y}{Z}$.
\end{prop}

\begin{proof}
Copy $\Segment{Z}{X}$ onto the ray opposite $\Ray{C}{A}$ at the point $D$. Now $\Angle{B}{C}{D}$ is a right angle, since it is supplementary to $\Angle{A}{C}{B}$. By SAS, we have $\Triangle{X}{Y}{Z} \equiv \Triangle{D}{C}{B}$, and thus $\Segment{B}{D} \equiv \Segment{Y}{X} \equiv \Segment{B}{A}$. Now $\Triangle{A}{B}{D}$ is isoceles with $\Segment{B}{A} \equiv \Segment{B}{D}$, so that $\Angle{B}{A}{C} \equiv \Angle{B}{A}{D} \equiv \Angle{B}{D}{A} \equiv \Angle{Y}{X}{Z}$. By AAS, we have $\Triangle{A}{B}{C} \equiv \Triangle{X}{Y}{Z}$.
\end{proof}

\begin{prop}
A triangle formed by three noncollinear points cannot have two interior angles which are both right.
\end{prop}

\begin{proof}
Such a triangle would violate the Alternate Interior Angles theorem since right angles are self-supplementary, and any two right angles are congruent.
\end{proof}

\subsection*{Bisection}

\begin{construct}[Angle Bisector]
Let $A$, $O$, and $B$ be noncollinear points. There exists a unique line $\ell$, containing $O$, such that if $U \in \ell$ is different from $O$ then $\Angle{A}{O}{U} \equiv \Angle{B}{O}{U}$. This line is called the \emph{bisector} of $\Angle{A}{O}{B}$.
\end{construct}

\begin{proof}
Note that we can assume WLOG that $\Segment{O}{A} \equiv \Segment{O}{B}$; if not, construct such a point on $\Ray{O}{B}$ using the Circle Separation property. Since the intersection of $\Circle{A}{O}$ and $\Circle{B}{O}$ contains a point not on $\Line{A}{B}$, by Circle Cut Transfer there is a second point $U$ on the opposite side of $\Line{A}{B}$ such that $\Segment{A}{U} \equiv \Segment{B}{U}$. Let $\ell = \Line{O}{U}$. Note that $\Triangle{A}{O}{U} \equiv \Triangle{B}{O}{U}$ by SSS, so that $\Angle{A}{O}{U} \equiv \Angle{B}{O}{U}$. Then if $V$ is a point such that $\Between{V}{O}{U}$, we have $\Angle{V}{O}{A} \equiv \Angle{V}{O}{B}$, since these are supplementary to congruent angles.

To see uniqueness, note that any such line must contain $O$ and $U$.
\end{proof}

\begin{cor}
$A$ and $B$ are on opposite sides of the bisector of $\Angle{A}{O}{B}$. In particular, the bisector of $\Angle{A}{O}{B}$ contains points which are interior to $\Angle{A}{O}{B}$.
\end{cor}

\begin{proof}
Suppose otherwise, and let $U \neq O$ be a point on the bisector. Then $\Angle{U}{O}{A}$ and $\Angle{U}{O}{B}$ are congruent angles on the same half-plane of a ray, so that $A$, $B$, and $O$ are collinear -- a contradiction. By the plane separation property there is a point $W$ between $A$ and $B$ which is on the bisector; this point is interior to $\Angle{A}{O}{B}$ as needed.
\end{proof}

\begin{construct}[Segment Midpoint]
Let $A$ and $B$ be distinct points. There is a unique point $M$ such that $\Between{A}{M}{B}$ and $\Segment{A}{M} \equiv \Segment{B}{M}$. This point is called the \emph{midpoint} of $\Segment{A}{B}$.
\end{construct}

\begin{proof}
Construct a point $O$ such that $\Triangle{A}{O}{B}$ is equilateral, and construct the bisector of $\Angle{A}{O}{B}$. By the Crossbar theorem, this bisector must cut $\Segment{A}{B}$ at an interior point, say $M$. Now $\Triangle{O}{A}{M} \equiv \Triangle{O}{B}{M}$ by SAS, and thus $\Segment{A}{M} \equiv \Segment{B}{M}$ as needed. Note that $M$ is unique by the uniqueness of congruent segments on a ray.
\end{proof}

\end{document}
