Here we \textbf{briefly} review some basic ideas from the naive theory of sets.

\begin{proplist}
\item \emph{Set} is an undefined term.

\item There is an undefined predicate \(\in\) among sets, pronounced ``is an element of'', such that given two sets \(A\) and \(B\) the statement \(A \in B\) is either true or false.

\item There is a set \(\varnothing\), called the \emph{empty set}, such that \(x \in \varnothing\) is false for all \(x\).

\item Given two sets \(A\) and \(B\), we say \(A\) is a \emph{subset} of \(B\), denoted \(A \subseteq B\), if for all sets \(x\), \(x \in A\) implies \(x \in B\).

\item If \(A\) is a set, then there is a set \(\POW{A}\), called the \emph{powerset}\index{powerset} of \(A\), such that \(x \in \POW{A}\) if and only if \(x \subseteq A\).

\item Given two sets \(A\) and \(B\), we say \(A\) is \emph{equal to} \(B\) if \(A \subseteq B\) and \(B \subseteq A\).

\item If \(a\) and \(b\) are sets, there is a set \((a,b)\) called the \emph{ordered pair} with \emph{first coordinate} \(a\) and \emph{second coordinate} \(b\).

\item If \(A\) and \(B\) are sets, there is a set \(A \times B\), called the \emph{cartesian product} of \(A\) and \(B\), such that \(x \in A \times B\) if and only if \(x = (a,b)\) for some \(a \in A\) and \(b \in B\).

\item Let \(A\) and \(B\) be sets, with \(B \neq \varnothing\).
A subset \(f \subseteq A \times B\) is called a \emph{function}\index{function} or \emph{mapping} if
\begin{proplist}
\item \(f\) is \emph{total}: for every \(a \in A\), there exists a \(b \in B\) such that \((a,b) \in f\)
\item \(f\) is \emph{well-defined}: if \((a_1,b_1)\) and \((a_2,b_2)\) are in \(f\) such that \(a_1 = a_2\), then \(b_1 = b_2\).
\end{proplist}
In this case we say \(f\) is a function \emph{from} \(A\) \emph{to} \(B\), denoted \(f : A \rightarrow B\).
We say that \(A\) is the \emph{domain} of \(f\) and that \(B\) is the \emph{codomain} of \(f\).
Note that if \(f\) is a function from \(A\) to \(B\), then for every \(a \in A\) there exists a unique \(b \in B\) such that \((a,b) \in f\).
We call this \(b\) the \emph{image} of \(a\) under \(f\), denoted \(f(a)\).

\item Let \(f : A \rightarrow B\) be a function.
\begin{proplist}
\item We say \(f\) is \emph{injective}\index{injective} (also known as ``one-to-one'') if whenever \(a_1, a_2 \in A\) such that \(f(a_1) = f(a_2)\), in fact \(a_1 = a_2\).
\item We say \(f\) is \emph{surjective}\index{surjective} (also known as ``onto'') if for every \(b \in B\), there exists an \(a \in A\) such that \(f(a) = b\).
\item We say \(f\) is \emph{bijective}\index{bijective} if it is both injective and surjective.
\end{proplist}

\item If \(A\) is a set, then subsets of \(A \times A\) are called \emph{binary relations} on \(A\).
If \(\sigma\) is a binary relation on \(A\), we will typically write \(x \,\sigma\, y\) rather than \((x,y) \in \sigma\).

\item Let \(A\) be a set and \(\sigma\) a binary relation on \(A\).
We say that \(\sigma\) is a \emph{partial order}\index{partial order} on \(A\) if
\begin{proplist}
\item \(\sigma\) is \emph{reflexive}: \(a \,\sigma\, a\) for all \(a \in A\).
\item \(\sigma\) is \emph{antisymmetric}: If \(a \,\sigma\, b\) and \(b \,\sigma\, a\), then \(a = b\).
\item \(\sigma\) is \emph{transitive}: If \(a \,\sigma\, b\) and \(b \,\sigma\, c\), then \(a \,\sigma\, c\).
\end{proplist}

\item Let \(A\) be a set and \(\sigma\) a binary relation on \(A\).
We say that \(\sigma\) is an \emph{equivalence}\index{equivalence} if
\begin{proplist}
\item \(\sigma\) is \emph{reflexive}: \(a \,\sigma\, a\) for all \(a \in A\).
\item \(\sigma\) is \emph{symmetric}: If \(a \,\sigma\, b\) then \(b \,\sigma\, a\).
\item \(\sigma\) is \emph{transitive}: If \(a \,\sigma\, b\) and \(b \,\sigma\, c\), then \(a \,\sigma\, c\).
\end{proplist}

\item If \(A\) is a set, then the functions from \(A \times A\) to \(A\) are called \emph{binary operations} on \(A\).
If \(f\) is a binary operation on \(A\), we will typically write \(x \,f\, y\) rather than \(f(x,y)\).
\end{proplist}
