The integers have the nice property that, for any constants \(a\) and \(b\), the equation \[ a + x = b \] always has a solution in \(\ZZ\) -- namely, \(b - a\).
This is no longer true if we replace the plus by times; the equation \(ax = b\) may or may not have any solutions in \(\ZZ\).
If so, \(a\) and \(b\) enjoy a special relationship.

\begin{dfn}[Divides]
Given integers \(a\) and \(b\), we say that \(a\) \emph{divides} \(b\), written \(a|b\), if there is an integer \(c\) such that \(ac = b\).
In this case we say that \(a\) is a \emph{divisor} of \(b\).
\end{dfn}

Divisibility satisfies several basic properties.

\begin{prop}\label{prop:int-div}
The following hold for all integers \(a\), \(b\), and \(c\).
\begin{proplist*}
\item \label{prop:int-div:zero} \(a|0\).
\item \label{prop:int-div:one} \(1|a\).
\item \label{prop:int-div:self} \(a|a\).
\item If \(a|b\) and \(b|c\) then \(a|c\).
\item \label{prop:int-div:bound} If \(a|b\) and \(b \neq 0\) then \(|a| < |b|\).
\end{proplist*}
\end{prop}

Note that \sref{prop:int-div}{one} and \sref{prop:int-div}{self} say that every integer has at least two divisors, though typical integers have many more.
Likewise \sref{prop:int-div}{zero} says that every integer is a divisor of at least one other.
If one integer is simultaneously a divisor of two others, we say it is a \emph{common divisor}, and this is very special.

\begin{dfn}
Let \(a\), \(b\), \(c\), and \(d\) be integers.
\begin{proplist}
\item We say that \(c\) is a \emph{common divisor} of \(a\) and \(b\) if \(c|a\) and \(c|b\).
\item We say that \(d\) is a \emph{greatest common divisor} of \(a\) and \(b\) if \(d\) is a common divisor, and if \(c\) is another common divisor, then \(c \leq d\).
\end{proplist}
\end{dfn}

In fact any two integers (not both zero) \emph{have} a greatest common divisor, which is unique.

\begin{prop}\label{prop:zz-gcds-exist}
Any two integers (not both zero) have a unique greatest common divisor, which we denote \(\GCD{a}{b}\).
We also define \(\GCD{0}{0} = 0\) as a special case.
\end{prop}

\begin{proof}
First we show existence.
Let \(M = \max(|a|,|b|)\) and let \[ S = \{ M - c \mid c\ \mathrm{is\ a\ common\ divisor\ of}\ a\ \mathrm{and}\ b \}. \]
Now \(S\) is not empty since \(M - 1 \in S\), and in fact \(S \subseteq \NN\) since \(c \leq M\) for all common divisors \(c\) of \(a\) and \(b\).
By WOP, then, \(S\) has a least element; say \(t = M - d\).
Now \(d\) is a common divisor of \(a\) and \(b\) by definition.
If \(c\) is a common divisor of \(a\) and \(b\), then \(M - d \leq M - c\), which rearranges as \(c \leq d\).
So \(d\) is a greatest common divisor of \(a\) and \(b\).
To see uniqueness, note that if \(d_1\) and \(d_2\) are both greatest common divisors of \(a\) and \(b\) then we have \(d_1 \leq d_2\) and \(d_2 \leq d_1\) so that \(d_1 = d_2\).
\end{proof}

Because it is unique, we can think of the greatest common divisor as a function of two integers.

\begin{prop}
The following hold for all integers \(a\) and \(b\).
\begin{proplist*}
\item \(\GCD{a}{b} = \GCD{b}{a}\).
\item \(\GCD{a}{a} = |a|\).
\item If \(b|a\), then \(\GCD{a}{b} = |b|\).
\item \(\GCD{a}{1} = 1\).
\item \(\GCD{a}{0} = |a|\).
\end{proplist*}
\end{prop}

Our proof that GCDs always exist is not constructive since it depends on the Well-Ordering Property.
We know that GCDs exist, but actually finding them is difficult!
For example, suppose we want to find the greatest common divisor of \(2\) and \(3\).
From the definition, we can restrict our search to the common divisors of \(2\) and \(3\).
Using \sref{prop:int-div}{bound}, we can show that the divisors of \(2\) are \(\pm 1\) and \(\pm 2\) and the divisors of \(3\) are \(\pm 1\) and \(\pm 3\).
The largest number which appears in both lists is 1, so that \(\GCD{2}{3} = 1\).
This works well enough, but what if we want to find \(\GCD{1337}{1007}\)?
The same brute force strategy works in principle, but would take a long time.

It is possible to give a constructive, induction based proof of \propref{prop:zz-gcds-exist} (try it!) but we will avoid doing so here.
Such a proof will be longer and more tedious while providing no more insight than this one does -- giving more heat than light, so to speak -- and the resulting algorithm slow.
Instead we give the following result which connects greatest common divisors to the division algorithm.

\begin{prop}[Euclidean Algorithm for \(\ZZ\)]
If \(a\) and \(b\) are integers with \(b > 0\), and if \(a = qb + r\) where \(0 \leq r < b\), then \(\GCD{a}{b} = \GCD{b}{r}\).
\end{prop}

\begin{proof}
Let \(d = \GCD{a}{b}\) and \(e = \GCD{b}{r}\).
We need to show that \(d = e\); to do this, we will show that \(d \leq e\) and \(e \leq d\).
\begin{proplist}
\item By definition we have \(d|a\) and \(d|b\); that is, \(a = da'\) and \(b = db'\) for some integers \(a'\) and \(b'\).
Now \[ r = a - qb = da' - qdb' = d(a' - qb'), \] so that \(d|r\).
In particular, \(d\) is a common divisor of \(b\) and \(r\), and so \(d \leq e\).
\item Similarly, we have \(e|b\) and \(e|r\), so that \(e|a\), and thus \(e \leq d\).
\qedhere
\end{proplist}
\end{proof}

The Euclidean Algorithm gives us a way to explicitly compute the GCD of two integers \emph{as long as} we can compute quotients and remainders as in the Division Algorithm; in fact, it is quite fast.
Note that since \(r\) is strictly less than \(b\), this recursion must eventually terminate with a statement of the form \(\GCD{a}{0}\).
This result is named after \emph{the} Euclid, because as far as we know he was the first to write it down -- in his book \emph{The Elements of Geometry}.
Euclid himself thought of the algorithm in geometric terms, where \(a\) and \(b\) represent the lengths of line segments and \(\GCD{a}{b}\) the length of the largest line segment that can ``tile'' both simultaneously.

\begin{thm}[Bezout's Identity]
If \(a\) and \(b\) are integers, then there exist integers \(u\) and \(v\) such that \(\GCD{a}{b} = ua + vb\).
\end{thm}

\begin{proof}
We start with the case \(b \geq 0\), proceeding by strong induction.
\begin{itemize}
\item \textbf{Base Case} (\(b = 0\)): Note that \(\GCD{a}{0} = a = a \cdot 1 + 0 \cdot 0\) as needed.
That is, the result holds with \(u = 1\) and \(v = 0\).
\item \textbf{Base Case} (\(b = 1\)): Note that \(\GCD{a}{1} = 1 = a \cdot 0 + 1 \cdot 1\) as needed.
That is, the result holds with \(u = 0\) and \(v = 1\).
\item \textbf{Inductive Step}: Suppose the result holds for all integers \(b'\) with \(0 \leq b' < b\), where \(b > 1\).
That is, for all such \(b'\) and all integers \(a\) there exist integers \(u\) and \(v\) such that \(\GCD{a}{b'} = au + b'v\).
Now consider \(b\).
By the division algorithm we have integers \(q\) and \(r\) such that \(a = qb + r\) and \(0 \leq r < b\).
We have two possibilities to consider.
\begin{itemize}
\item If \(r = 0\), then in fact \(b|a\), since \(a = qb\).
So \(\GCD{a}{b} = b = a \cdot 0 + q \cdot b\).
That is, the result holds with \(u = 0\) and \(v = 1\).
\item If \(r > 0\), then by the induction hypothesis there exist integers \(u'\) and \(v'\) such that \(\GCD{b}{r} = bu' + rv'\).
By the euclidean algorithm, we have
\begin{eqnarray*}
\GCD{a}{b} & = & \GCD{b}{r} \\
 & = & bu' + rv' \\
 & = & bu' + (a - qb)v' \\
 & = & av' + b(u' - qv').
\end{eqnarray*}
That is, the result holds with \(u = v'\) and \(v = u' - qv'\).
\end{itemize}
\end{itemize}
By Strong Induction, for all \(b \geq 0\) and all integers \(a\) there exist integers \(u\) and \(v\) such that \(\GCD{a}{b} = au + bv\).

Now suppose \(b < 0\), so that \(-b > 0\).
By the previous discussion, there exist integers \(u'\) and \(v'\) such that \(\GCD{a}{-b} = au' + (-b)v'\).
Now \[ \GCD{a}{b} = \GCD{a}{-b} = au' + (-b)v' = au' + b(-v'). \]
That is, the result holds with \(u = u'\) and \(v = -v'\). 
\end{proof}

Similar to the Euclidean Algorithm, this proof of Bezout's Identity provides us with a strategy for actually finding the coefficients \(u\) and \(v\) recursively.

\begin{dfn}[Relatively Prime]
We say that integers \(a\) and \(b\) are \emph{relatively prime} if \(\GCD{a}{b} = 1\).
\end{dfn}

\begin{thm}[Euclid's Lemma]
If \(a\) and \(b\) are relatively prime integers and \(c\) an integer such that \(a | bc\), then \(a|c\).
\end{thm}

\begin{proof}
By Bezout's Identity, we have \(1 = au + bv\) for some integers \(u\) and \(v\); so \(c = auc + bvc\).
Since \(a|bc\), we have \(bc = at\) for some integer \(t\).
Thus \[ c = auc + bvc = auc + atv = a(uc + tv), \] and so \(a|c\) as claimed.
\end{proof}



%---------%
\Exercises%
%---------%

\begin{exercise}
Let \(a\) and \(b\) be integers.
Show that if \(ac = b\) and \(ad = b\) then \(c = d\).
\end{exercise}

\begin{exercise}
Suppose \(a\) and \(b\) are integers such that \(a|b\) and \(b|a\). Show that \(b = \pm a\).
\end{exercise}

\begin{exercise}
Suppose \(a\) and \(b\) are integers such that \(a|b\).
Show that \((-a)|b\), \(a|(-b)\), and \((-a)|(-b)\).
\end{exercise}

\begin{exercise}
Use the Euclidean Algorithm to compute the following.
\begin{proplist*}
\item \(\GCD{12}{5}\)
\item \(\GCD{57}{17}\)
\item \(\GCD{210}{115}\)
\item \(\GCD{1337}{1007}\)
\end{proplist*}
\end{exercise}
