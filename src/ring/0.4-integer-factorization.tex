Recall that every integer except 1 and -1 has at least two divisors (four if we count negatives): itself and 1.
The integers which have \emph{only} these divisors are somehow simpler than the rest, so we give them a name.

\begin{dfn}[Prime] \label{dfn:zz-prime}
We say an integer \(p \notin \{1,0,-1\}\) is \emph{prime} if whenever \(p = ab\), either \(a = \pm 1\) or \(b = \pm 1\).
\end{dfn}

Equivalently, \(p\) is prime if it is not 0, 1, or -1, and the only divisors of \(p\) are \(\pm 1\) and \(\pm p\).
You might wonder why we have excluded 0, 1, and -1 from our definition of ``prime''.
After all, 1 has no positive integer divisors other than itself and 1, right?
This is true, but it turns out that 1 and -1 are special for another reason: we can ``divide'' by them in the integers in the sense that \(1 \cdot x = a\) and \(-1 \cdot x = a\) have solutions in \(\ZZ\) for all integers \(a\).
For reasons we'll explain in a bit, this makes it inconvenient to think of 1 and -1 as prime.
On the other hand there is a very good argument to be made that we should think of 0 as prime, even though it is divisible by every integer!
The point (for now) is that the whole business of prime numbers becomes much simpler if we just exclude 1, 0, and -1 from consideration, and so we do.

\begin{prop}\label{prop:zz-prime}
Let \(p\) and \(a\) be integers.
\begin{proplist}
\item 2 and 3 are prime.
\item If \(p\) is prime, then \(-p\) is prime.
\item If \(p\) and \(q\) are prime such that \(p|q\), then \(q = \pm p\).
\item \label{prop:zz-prime:gcd} If \(p\) is prime then \(\GCD{a}{p}\) is \(|p|\) if \(p|a\) and \(1\) otherwise.
\end{proplist}
\end{prop}

The following result is an alternative way to characterize the prime integers which will turn out to be interesting.

\begin{prop} \label{prop:zz-prime-2}
An integer \(p\) is prime if and only if whenever \(p|ab\), either \(p|a\) or \(p|b\).
\end{prop}

\begin{proof}
This is an ``if and only if'' statement; we proceed by proving the ``only if'' part and then the ``if'' part.
\begin{proplist}
\item[\((\Rightarrow)\)] Suppose \(p\) is prime and that \(p|ab\).
Consider \(\GCD{a}{p}\).
Since \(p\) is prime, by \sref{prop:zz-prime}{gcd} there are two possibilities.
\begin{proplist}
\item If \(\GCD{a}{p} = |p|\), then \(|p|\) divides \(a\), so that \(p|a\).
\item If \(\GCD{a}{p} = 1\), then by Euclid's Lemma, \(p|b\).
\end{proplist}

\item[\((\Leftarrow)\)] Suppose \(p\) has the property that if \(p|ab\) then either \(p|a\) or \(p|b\).
Suppose further that \(p = ab\).
In particular \(p|ab\) (since \(1 \cdot p = ab\)) and so, without loss of generality, \(p|a\).
Say \(a = pa'\).
Now \(p = ab = pa'b\), and thus \(1 = a'b\).
Now \(|b| = 1\), and thus \(|p| = |a|\), so that \(p = \pm a\) as needed.
\qedhere
\end{proplist}
\end{proof}

(By the way, this is where the argument that 0 should be prime comes in: it satisfies \propref{prop:zz-prime-2}.)

\begin{cor}
If \(p\) is a prime and \(a_i\) integers such that \(p|a_1a_2 \cdots p_n\), then \(p|a_k\) for some \(k\).
\end{cor}

The next two results are the reason why primes are interesting, and ultimately the reason why we don't consider 1 and -1 to be primes.
The primes are the building blocks of all other integers in a very precise sense: every integer can be written as a product of primes in essentially one way.
This 

\begin{thm}[Fundamental Theorem of Arithmetic: Existence]
Every integer other than \(0\), \(1\), and \(-1\) can be written as a product of primes.
That is, every such \(n\) can be expressed as \(n = p_1p_2 \cdots p_k\), where the \(p_i\) are prime.
This is called a \emph{prime factorization} of \(n\).
\end{thm}

\begin{proof}
Suppose first that \(b > 0\).
We proceed by strong induction.
\begin{proplist}
\item \textbf{Base Case} (\(b = 2\)): If \(d|2\), then \(0 < |d| \leq |2|\).
Thus the only possible divisors of 2 are \(\pm 1\) and \(\pm 2\), and so 2 is prime by definition.
\item \textbf{Inductive Step}: Suppose that for some \(n\), every integer \(2 \leq n' < n\) can be written as a product of primes, and consider \(n\).
If \(n\) is itself prime, then \(n = n\) is its own prime factorization.
If \(n\) is not prime, then by definition there exist integers \(a\) and \(b\) such that \(n = ab\) and \(n \neq \pm a\) and \(n \neq \pm b\).
Since \(n > 0\), we can assume that \(a > 0\) and \(b > 0\).
In fact we have \(a > 1\) and \(b > 1\), so that \(a,b < n\).
By the inductive hypothesis, \(a\) and \(b\) have prime factorizations; say \(a = p_1p_2 \cdots p_h\) and \(b = q_1q_2 \cdots q_k\).
Now \[ n = ab = p_1p_2 \cdots p_h q_1q_2 \cdots q_k \] has a prime factorization.
\end{proplist}
Thus by strong induction every integer \(n \geq 2\) has a prime factorization.
If \(n < 0\), then \(-n > 0\), so that \(-n = p_1p_2 \cdots p_k\) has a prime factorization; then \(n = (-p_1)p_2 \cdots p_k\) also has a prime factorization.
\end{proof}

\begin{thm}[Fundamental Theorem of Arithmetic: Uniqueness]
The prime factorization of an integer is unique in the following sense.
If \(n\) has two prime factorizations \[ n = p_1p_2 \cdots p_k = q_1q_2 \cdots q_\ell, \] then \(k = \ell\) and, after relabeling the \(q_i\), we have \(p_i = \pm q_i\) for each \(1 \leq i \leq k\).
\end{thm}

\begin{proof}
We saw in FTA Part 1 that every such \(n\) has at least one prime factorization, which consists of at least one prime factor.
To show uniqueness we will proceed by strong induction on the \emph{length} of the shortest prime factorization of \(n\).
\begin{proplist}
\item \textbf{Base Case:} Suppose \(n = p\) has a prime factorization of length 1; that is, \(n\) itself is prime.
Suppose \(p = q_1q_2 \ldots q_\ell\) is another prime factorization of \(n\).
Since \(p\) is prime, we have (rearranging the \(q_i\) if necessary) \(p|q_1\).
Since \(p\) and \(q_1\) are prime, we have \(q_1 = \pm p\).
Now if \(\ell > 1\) we have \[ 1 = |q_2 \cdots q_\ell|, \] so that \(|q_i| = 1\) for each \(q_i\), a contradiction.
So in fact \(\ell = 1\) and \(q_1 = \pm p\), as claimed.
\item \textbf{Inductive Step:} Suppose that every integer having a shortest prime factorization of length at most \(k\) has a unique prime factorization, and suppose \(n = p_1p_2 \cdots p_{k+1}\) is an integer with a shortest prime factorization of length \(k+1\).
Suppose further that \(n = q_1q_2 \cdots q_\ell\) is another prime factorization of \(n\).
In particular, \(p_1 | q_1q_2 \cdots q_\ell\), so that, rearranging the \(q_i\) if necessary, \(p|q_1\).
Now \(q_1 = \pm p\), and we have \[p_2 \cdots p_{k+1} = q_2 \cdots q_\ell. \]
Note that these are two prime factorizations of an integer having a shortest prime factorization of length at most \(k\).
By the Inductive Hypothesis, we have \(\ell = k+1\) and, relabeling the \(q_i\) if necessary, \(q_i = \pm p_i\) for each \(2 \leq i \leq k+1\).
So the prime factorization of \(n\) is unique.
\qedhere
\end{proplist}
\end{proof}

We've established that every integer has an essentially unique prime factorization.
But \textbf{how do we prove that a given integer is prime}?
The definition suggests one way to do it, known as \emph{trial division}: an integer \(n\) is prime if and only if \(n\) is not divisible by any integer \(t\) with \(1 < t < n\).
This works, but is extremely time consuming.
A better method is suggested by the following.

\begin{prop}
Let \(n > 1\) be an integer, and let \(t\) be the largest integer such that \(t^2 < n\).
(Such an integer exists, since the set of all \(t\) with \(t^2 < n\) is bounded above by \(n\).)
(Also, this \(t\) is \(\lfloor \sqrt{n} \rfloor\), but we don't know what \(\sqrt{\cdot}\) means.)
Then \(n\) is prime if and only if \(n\) is not divisible by any prime \(p\) with \(2 \leq p \leq t\).
\end{prop}

\begin{proof}
Certainly if \(n\) is prime it is not divisible by any such \(p\).
We prove the ``only if'' part by contraposition.
Suppose that \(n\) is \emph{not} prime; say \(n = ab\), where \(n \neq a\) and \(n \neq b\).
(We can assume positive signs here since \(n > 0\).)
If \(a\) and \(b\) are both strictly larger than \(t\), then we have \(n > t^2 > ab > n\), a contradiction.
Without loss of generality, then, \(a \leq t\).
In particular, all prime factors of \(a\) are less than \(t\) in absolute value, and so \(n\) has a prime factor \(p\) such that \(2 \leq p \leq t\).
\end{proof}

For example, consider \(n = 5\).
The largest \(t\) such that \(t^2 < 5\) is \(t = 2\), and the only prime \(p\) such that \(2 \leq p \leq 2\) is \(p = 2\).
Since (by the Division Algorithm) we have \(5 = 2 \cdot 2 + 1\), with remainder \(1 \neq 0\), 2 does not divide 5.
So \textbf{5 is prime}.

This result gives us a strategy for finding prime numbers, but it only works if we have a complete list of primes up to \(\sqrt{n}\) to begin with.
In other words, we can find a prime if we start with a list of primes.
This can be made into a reasonably efficient algorithm for finding all the primes up to some bound, which we will explore in the exercises.
There are some interesting questions left unanswered, though.
Suppose we have an integer \(n\).
\begin{itemize}
\item Is \(n\) prime?
\item What is the smallest prime factor of \(n\)?
\item What is the prime factorization of \(n\)?
\item How many prime factors does \(n\) have?
\item How many \emph{distinct} prime factors does \(n\) have?
\item How difficult is it to answer these questions?
\item How difficult is it to \emph{verify} the answers to these questions?
\end{itemize}



%---------%
\Exercises%
%---------%

\begin{dfn}[Euler Totient]
Let \(n\) be a positive integer.
We define the \emph{totient}\index{totient} of \(n\) to be the cardinality of the set \[ \{ a \mid 0 \leq a < n, \GCD{a}{n} = 1 \}. \]
We denote this number by \(\TOT{n}\).
\end{dfn}

\begin{exercise}[Sieve of Eratosthenes.]
(@@@)
\end{exercise}
