These are the lecture notes of a first abstract algebra course the author teaches at the Northeastern State University of Oklahoma. Each section is intended to be one-half to one class meeting's worth of material.

As these are functionally lecture notes, used in a specific class, the main expository sections develop only the basic theory of rings as dictated by the needs of my students. Plenty of supplementary material, including some important definitions and theorems, is relegated to the ``exercises''. The exercises come in a few flavors: some are straight computation, some ask for examples or counterexamples, and some require a proof. Some exercises -- my favorites -- will guide you through the proofs of reasonably high-powered theorems. Motivated readers are encouraged to work as many of these as they can stand.

Some additional supplementary material, appropriate to the main text but cut from a typical class due to time constraints, is given in the appendices. These include constructions of the real, complex, and \(p\)-adic numbers, Zorn's lemma, and a brief discussion of categories.
