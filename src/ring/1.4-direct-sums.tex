How do we find concrete examples of rings? Most sets don't have obvious choices for the plus and times operations lying around. But if we have some \emph{existing} rings, equipped with plus and times, we might hope to use that existing arithmetic to construct a new one. So a major motivating question throughout this text is what we will call
\begin{titlebox}{The Construction Problem}
\begin{center}
How can we build new rings out of the ``parts'' of old ones?
\end{center}
\end{titlebox}
We will see that, while this question is nebulous enough to avoid a complete answer, there are many useful partial answers. The first, and perhaps the simplest, is to construct \emph{direct sums}: two (or more) rings glued together as a cartesian product.

\begin{dfn}[Direct Sum of Two Rings] \label{dfn:direct-sum}
Let \(R\) and \(S\) be rings. Then the \emph{direct sum} \index{direct sum} of \(R\) and \(S\), denoted \(R \oplus S\), is the cartesian product \[ R \oplus S = \{ (r,s) \mid r \in R, s \in S \} \] equipped with the ``componentwise'' operations defined as follows.
\begin{eqnarray*}
(r_1, s_1) + (r_2, s_2) & = & (r_1 + r_2, s_1 + s_2) \\
(r_1, s_1) \cdot (r_2, s_2) & = & (r_1 \cdot r_2, s_1 \cdot s_2)
\end{eqnarray*}
\end{dfn}

The cartesian product is one of the least sophisticated ways to bolt two sets together, and the least sophisticated way to attempt a ring structure on a cartesian product is to perform arithmetic coordinatewise. Of course we have to verify that these operations make \(R \oplus S\) into a ring; the proof of this is left to the exercises.

\begin{prop} \label{prop:direct-sum}
Given rings \(R\) and \(S\) we have the following.
\begin{proplist}
\item The direct sum \(R \oplus S\) with the componentwise operations is a ring.
\item The \emph{coordinate projections} \(\pi_1 : R \oplus S \rightarrow R\) and \(\pi_2 : R \oplus S \rightarrow S\), given by \(\pi_1(r,s) = r\) and \(\pi_2(r,s) = s\), are surjective ring homomorphisms.
\item The \emph{coordinate injections} \(\iota_1 : R \rightarrow R \oplus S\) and \(\iota_2 : S \rightarrow R \oplus S\), given by \(\iota_1(r) = (r,0_S)\) and \(\iota_2(s) = (0_R,s)\), are injective ring homomorphisms.
\end{proplist}
\end{prop}

Although (or perhaps because) the direct sum is such a simple way to build new rings out of old ones, it is a very important tool in our quest to understand the structure of rings. We also get some mappings for free, the coordinate projections \(\pi_1\) and \(\pi_2\) and injections \(\iota_1\) and \(\iota_2\). These mappings are simple, but they play a very important structural role for the direct sum.

\begin{examples}
\item The ring \(\ZZ \oplus \ZZ\) is, as a set, \[ \ZZ \oplus \ZZ = \{ (a,b) \mid a,b \in \ZZ \}. \] In this ring we have, for example, \[ (1,2) + (4,-3) = (1+4, 2-3) = (5,-1) \] and \[ (2,-1) \cdot (5,0) = (2 \cdot 5, -1 \cdot 0) = (10,0). \]
\item The ring \(\ZZ/(2) \oplus \ZZ/(2)\) has 4 elements. The cayley table of this ring is shown in \autoref{fig:cayley-zz2pzz2}.
\begin{figure}[h!]
\begin{center}
\small
\addtolength{\tabcolsep}{-2pt}
\begin{tabular}{c|cccc}
\(+\) & (0,0) & (0,1) & (1,0) & (1,1) \\ \hline
(0,0) & (0,0) & (0,1) & (1,0) & (1,1) \\
(0,1) & (0,1) & (0,0) & (1,1) & (1,0) \\
(1,0) & (1,0) & (1,1) & (0,0) & (0,1) \\
(1,1) & (1,1) & (1,0) & (0,1) & (0,0)
\end{tabular}
\quad
\begin{tabular}{c|cccc} \setlength{\tabcolsep}{5pt}
\(\cdot\)
      & (0,0) & (0,1) & (1,0) & (1,1) \\ \hline
(0,0) & (0,0) & (0,0) & (0,0) & (0,0) \\
(0,1) & (0,0) & (0,1) & (0,0) & (0,1) \\
(1,0) & (0,0) & (0,0) & (1,0) & (1,0) \\
(1,1) & (0,0) & (0,1) & (1,0) & (1,1)
\end{tabular}
\addtolength{\tabcolsep}{2pt}
\end{center}
\caption{Cayley tables of \(\ZZ/(2) \oplus \ZZ/(2)\).\label{fig:cayley-zz2pzz2}}
\end{figure}
\end{examples}

We can think of \(\oplus\) like an operation on \emph{rings}, which takes two rings and builds a new one. It isn't an operation quite like plus on \(\ZZ\), though, because operations can only be defined on \emph{sets}, but there is no set of all rings. But it is still reasonable to hope that as an operation-like thing, \(\oplus\) behaves nicely. For instance, we might hope that for two rings \(R\) and \(S\), we have \(R \oplus S = S \oplus R\). Of course this cannot possibly be true; as sets, \(R \oplus S\) and \(S \oplus R\) will generally be disjoint. In the next section, we will see how to get around this restriction.

There are other ways to put a ring structure on a cartesian product of rings; some of these are explored in the exercises, and another is in \autoref{sec:quad-ext}.



%---------%
\Exercises%
%---------%

\begin{exercise}
Construct the cayley tables of the following rings.
\begin{proplist*}
\item \(\ZZ/(2) \oplus \ZZ/(2)\)
\item \(\ZZ/(2) \oplus \ZZ/(3)\)
\item \(2R \oplus \ZZ/(3)\), where \(R = \ZZ/(4)\)
\end{proplist*}
\end{exercise}

\begin{exercise}
Let \(R\) and \(S\) be rings.
\begin{proplist*}
\item Show that \(R \oplus S\) is commutative if and only if \(R\) and \(S\) are commutative.
\item Show that \(R \oplus S\) is boolean if and only if \(R\) and \(S\) are boolean.
\item Show that \(R \oplus S\) is unital if and only if \(R\) and \(S\) are unital, and in this case \(1_{R \oplus S} = (1_R, 1_S)\).
\item Show that \(R \oplus S\) is null if and only if \(R\) and \(S\) are null.
\item Show that \(R \oplus S\) is finite if and only if \(R\) and \(S\) are finite.
\end{proplist*}
\end{exercise}

\begin{exercise}
Show that if \(R\) is unital, then the coordinate projection \(\pi_1 : R \oplus S \rightarrow R\) is a unital homomorphism.
\end{exercise}

\begin{exercise}
Prove \propref{prop:direct-sum}.
\end{exercise}

\begin{exercise}
Let \(R\) and \(S\) be rings and let \((r,s) \in R \oplus S\).
\begin{proplist}
\item Show that \((r,s)^n = (r^n, s^n)\) for all \(n \in \NN \setminus \{0\}\). (cf. \eref{exerc:ring-powers})
\item If \(R\) and \(S\) are both unital, show that this identity also holds for \(n = 0\).
\end{proplist}
\end{exercise}

\begin{exercise}
Let \(R\) and \(S\) be rings and let \((r,s) \in R \oplus S\). Show that \(n(r,s) = (nr, ns)\) for all \(n \in \ZZ\). (cf. \eref{exerc:elt-mult})
\end{exercise}

\begin{exercise}
Show that \(\CHAR{R \oplus S} = \LCM{\CHAR{R}}{\CHAR{S}}\).
\end{exercise}

\begin{exercise}[Universal Property of Binary Direct Sums.] \label{exerc:up-direct-sum} \index{universal property!of direct sums}
Let \(R\), \(S\), and \(T\) be rings. Suppose we have ring homomorphisms \(\varphi : T \rightarrow R\) and \(\psi : T \rightarrow S\). Then there is a unique ring homomorphism \(\Theta : T \rightarrow R \oplus S\) such that the following diagram commutes.
\begin{center}
\begin{tikzcd}
 & T \arrow[ld,"\varphi"'] \arrow[rd, "\psi"] \arrow[d, dashed, "\Theta"] & \\
R & R \oplus S \arrow[l,"\pi_1"] \arrow[r, "\pi_2"'] & S
\end{tikzcd}
\end{center}
That is, there is a unique \(\Theta\) such that \(\pi_1 \circ \Theta = \varphi\) and \(\pi_2 \circ \Theta = \psi\).
\end{exercise}

Before we give the proof of this result, a note on its importance. The punchline of \eref{exerc:up-direct-sum} asserts the existence of a unique homomorphism \(\Theta\) through which any pair \((\varphi, \psi)\) must ``factor through''. This is the first of several results about rings known as \emph{universal properties}. The precise meaning of the term ``universal property'' is not important for us at the moment; it is enough to think of them as giving us ``unique homomorphisms for free'', in this case, homomorphisms into a direct sum. They usually also involve a kind of unique factorization of homomorphisms as we have here.

\begin{exercise}
When is the induced map in UPDP unital?
\end{exercise}

\begin{exercise}
Let \(R\), \(S\), and \(T\) be rings with homomorphisms \(\varphi : T \rightarrow R\) and \(\psi : T \rightarrow S\). Show that if \(\varphi\) and \(\psi\) are both injective, then the induced map \(\Theta : T \rightarrow R \oplus S\) is injective.
\end{exercise}

\begin{exercise}
Let \(\varphi : R_1 \rightarrow R_2\) and \(\psi : S_1 \rightarrow S_2\) be ring homomorphisms.
\begin{proplist}
\item Show that the mapping \(\Theta : R_1 \oplus S_1 \rightarrow R_2 \oplus S_2\) given by \(\Theta(r,s) = (\varphi(r), \psi(s))\) is a homomorphism. We refer to this mapping as \(\varphi \oplus \psi\).
\item (@@@) surjective
\item (@@@) injective
\item (@@@) unital
\end{proplist}
\end{exercise}

\begin{exercise}[A Zero-Square Ring.]
The componentwise operations are not the only way to turn the setwise cartesian product of rings into a ring. Here is another. Let \(S\) be a commutative ring and let \(R = S \times S \times S\). Define addition on \(R\) componentwise, and define multiplication by \[ (a_1,a_2,a_3) \cdot (b_1,b_2,b_3) = (0,0,a_1b_2 - a_2b_1). \] Show that \(R\) is a ring, and that \(x^2 = 0\) for all \(x \in R\).
\end{exercise}

\begin{exercise}
(@@@) move to subrings \textbf{Direct Products.} Let \(\Lambda\) be a nonempty set and let \(\mathcal{R} = \{ R_i \mid i \in \Lambda \}\) be a family of rings indexed by \(\Lambda\). We denote by \[ \prod_{i \in \Lambda} R_i \] the set of all mappings \(r : \Lambda \rightarrow \bigcup_{i \in \Lambda} R_i\) such that \(r(i) \in R_i\) for all \(i \in \Lambda\).
\begin{proplist}
\item Show that \(\prod_{i \in \Lambda} R_i\) is a ring with respect to pointwise plus and times.
\item Show that \(\prod_{i \in \Lambda} R_i\) is commutative if and only if \(R_i\) is commutative for each \(i \in \Lambda\).
\item Show that \(\prod_{i \in \Lambda} R_i\) is unital if and only if \(R_i\) is unital for each \(i \in \Lambda\).
\end{proplist}
\end{exercise}

\begin{exercise}
(@@@) move to subrings \textbf{Direct Sums.} Let \(\Lambda\) be a nonempty set and let \(\mathcal{R} = \{ R_i \mid i \in \Lambda \}\) be a family of rings indexed by \(\Lambda\). We say that an element \(\alpha \in \prod_{i \in \Lambda} R_i\) has \emph{finite support} if \(\alpha(i) = 0\) for all but finitely many \(i \in \Lambda\). We denote by \[ \bigoplus_{i \in \Lambda} R_i \] the set of all \(\alpha \in \prod_{i \in \Lambda} R_i\) of finite support.
\begin{proplist}
\item Show that \(\bigoplus_{i \in \Lambda} R_i\) is a subring of \(\prod_{i \in \Lambda} R_i\).
\item Show that \(\bigoplus_{i \in \Lambda} R_i\) is commutative if and only if \(R_i\) is commutative for each \(i \in \Lambda\).
\item Show that \(\bigoplus_{i \in \Lambda} R_i\) is unital if and only if \(R_i\) is unital for each \(i \in \Lambda\) and \(\Lambda\) is finite.
\end{proplist}
\end{exercise}

\begin{dfn}
Let \(R\) and \(S\) be rings, and let \(\theta : R \rightarrow \END{S}\) be a mapping. Note that \(\END{R}\) is not (generally) a ring under pointwise addition and composition, so \(\theta\) cannot possibly be a ring homomorphism. Instead, we say that \(\theta\) is a \emph{prehomomorphism} if \(\theta(a + b) = \theta(a) + \theta(b)\) and \(\theta(ab) = \theta(a)\theta(b)\) for all \(a,b \in R\). In this case, note that \(\IM{\theta}\) is a ring.

If \(\theta\) is a prehomomorphism such that \(\IM{theta}\) is commutative, we can define an alternative arithmetic on the cartesian product \(R \times S\) as follows. Addition is performed coordinatewise, while \[ (r_1, s_1) \cdot (r_2, s_2) = (r_1r_2, \theta(r_1)(s_2) + \theta(r_2)(s_1)). \] We call this the \emph{null extension} of \(S\) by \(R\) via \(\theta\), denoted \(\REXTEND{R}{\theta}{S}\)
\end{dfn}

\begin{exercise}
Show that \(\REXTEND{R}{\theta}{S}\) is a ring.
\end{exercise}

\begin{exercise}
Let \(S\) be a ring.
\begin{proplist}
\item Show that the map \(\theta : 0 \rightarrow \END{S}\) given by \(\theta(0)(s) = 0_S\) is a prehomomorphism, and that \(\IM{\theta}\) is commutative.
\item Show that \(\REXTEND{0}{\theta}{S}\) is a null ring, which has the same additive structure as \(S\). We call this the \emph{nullification} of \(S\), and denote it \(S^0\).
\end{proplist}
\end{exercise}

\begin{exercise}
Let \(R\) be a ring and \(N\) a null ring, and suppose we have a prehomomorphism \(\theta\) with \(\IM{\theta}\) commutative.
\end{exercise}

\begin{exercise}
Let \(R\) be a commutative ring. We define operations on the set \(\HAM{R} = R \times R \times R \times R\) by (@@@)
\begin{proplist}
\item These operations make \(\HAM{R}\) a noncommutative ring.
\item The map \(\iota : R \rightarrow \HAM{R}\) given by \(\iota(r) = (r,0,0,0)\) is an injective ring homomorphism.
\item If \(R\) is a field, then \(\HAM{R}\) is a skew field.
\end{proplist}
\end{exercise}
