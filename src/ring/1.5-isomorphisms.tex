Just for fun, let's consider the cayley tables of the ring \(\ZZ/(6)\) as shown in \autoref{fig:cayley-zz6}.
\begin{figure}[h!]
\begin{center}
\begin{tabular}{c|cccccc}
\(+\)
  & 0 & 1 & 2 & 3 & 4 & 5 \\ \hline
0 & 0 & 1 & 2 & 3 & 4 & 5 \\
1 & 1 & 2 & 3 & 4 & 5 & 0 \\
2 & 2 & 3 & 4 & 5 & 0 & 1 \\
3 & 3 & 4 & 5 & 0 & 1 & 2 \\
4 & 4 & 5 & 0 & 1 & 2 & 3 \\
5 & 5 & 0 & 1 & 2 & 3 & 4
\end{tabular}
\quad\quad
\begin{tabular}{c|cccccc}
\(\cdot\)
  & 0 & 1 & 2 & 3 & 4 & 5\\ \hline
0 & 0 & 0 & 0 & 0 & 0 & 0 \\
1 & 0 & 1 & 2 & 3 & 4 & 5 \\
2 & 0 & 2 & 4 & 0 & 2 & 4 \\
3 & 0 & 3 & 0 & 3 & 0 & 3 \\
4 & 0 & 4 & 2 & 0 & 4 & 2 \\
5 & 0 & 5 & 4 & 3 & 2 & 1 \\
\end{tabular}
\caption{\label{fig:cayley-zz6} Cayley tables of \(\ZZ/(6)\).}
\end{center}
\end{figure}
Let's also consider the cayley tables of the ring \(\ZZ/(2) \oplus \ZZ/(3)\), with the row and column headings written in a particular order (which will be explained in a moment) as shown in \autoref{fig:cayley-zz2-zz3}.
Drawing these tables is a little tedious, but straightforward.
\begin{figure}[h!]
\begin{center}
\begin{tabular}{c|cccccc}
\(+\) & (0,0) & (1,1) & (0,2) & (1,0) & (0,1) & (1,2) \\ \hline
(0,0) & (0,0) & (1,1) & (0,2) & (1,0) & (0,1) & (1,2) \\
(1,1) & (1,1) & (0,2) & (1,0) & (0,1) & (1,2) & (0,0) \\
(0,2) & (0,2) & (1,0) & (0,1) & (1,2) & (0,0) & (1,1) \\
(1,0) & (1,0) & (0,1) & (1,2) & (0,0) & (1,1) & (0,2) \\
(0,1) & (0,1) & (1,2) & (0,0) & (1,1) & (0,2) & (1,0) \\
(1,2) & (1,2) & (0,0) & (1,1) & (0,2) & (1,0) & (0,1)
\end{tabular}

\vspace{0.5cm}

\begin{tabular}{c|cccccc}
\(\cdot\) & (0,0) & (1,1) & (0,2) & (1,0) & (0,1) & (1,2) \\ \hline
(0,0)     & (0,0) & (0,0) & (0,0) & (0,0) & (0,0) & (0,0) \\
(1,1)     & (0,0) & (1,1) & (0,2) & (1,0) & (0,1) & (1,2) \\
(0,2)     & (0,0) & (0,2) & (0,1) & (0,0) & (0,2) & (0,1) \\
(1,0)     & (0,0) & (1,0) & (0,0) & (1,0) & (0,0) & (1,0) \\
(0,1)     & (0,0) & (0,1) & (0,2) & (0,0) & (0,1) & (0,2) \\
(1,2)     & (0,0) & (1,2) & (0,1) & (1,0) & (0,2) & (1,1) \\
\end{tabular}
\caption{\label{fig:cayley-zz2-zz3} Cayley tables of \(\ZZ/(2) \oplus \ZZ/(3)\).}
\end{center}
\end{figure}
It may be tricky to spot at first, but these two pairs of cayley tables are \emph{essentially the same}.
Why?
Note that the only real difference between the addition tables for \(\ZZ/(6)\) and \(\ZZ/(2) \oplus \ZZ/(3)\) is the \textbf{labels of the elements}.
Specifically, we can relabel them as follows.
\begin{center}
\begin{tabular}{ccc}
\(0 \leftrightarrow (0,0)\) & \(1 \leftrightarrow (1,1)\) & \(2 \leftrightarrow (0,2)\) \\
\(3 \leftrightarrow (1,0)\) & \(4 \leftrightarrow (0,1)\) & \(5 \leftrightarrow (1,2)\)
\end{tabular}
\end{center}
The same is true of the multiplication tables.
It may help to draw these cayley tables in color, so that corresponding elements are the same color.
Thinking of this correspondence between \(\ZZ/(6)\) and \(\ZZ/(2) \oplus \ZZ/(3)\) as a mapping, the sameness of the cayley tables gives us a little more: the correspondence is a \emph{homomorphism}.

Note that we can arrange the row and column labels of \autoref{fig:cayley-zz6} and \autoref{fig:cayley-zz2-zz3} however we like, but using this particular ordering makes it easier to see the correspondence.
Changing the order does not destroy the correspondence, it only makes it harder to see.

This relabeling raises an interesting question: in what sense are \(\ZZ/(6)\) and \(\ZZ/(2) \oplus \ZZ/(3)\) different rings?
Of course they are different as \emph{sets}.
But from a ring-theoretic perspective we only really care about \emph{structure} -- that is, the arithmetic, as embodied in the cayley tables.
These two rings have exactly the same structure, and we can translate between them via the relabeling without losing any information.
So in a sense \(\ZZ/(6)\) and \(\ZZ/(2) \oplus \ZZ/(3)\) are really two different manifestations of the same abstract ring structure, obtained by giving the elements different names.
When two rings are related in this way, we say they are \emph{isomorphic} to one another.

\begin{dfn} \label{dfn:isomorphism}
Let \(R\) and \(S\) be rings and \(\varphi : R \rightarrow S\) a ring homomorphism.
If \(\varphi\) is also bijective as a mapping, we say \(\varphi\) is an \emph{isomorphism}\index{isomorphism}.
In this case we say that \(R\) is \emph{isomorphic to} \(S\), denoted \(R \cong S\).
\end{dfn}

Remember: to show that one ring is isomorphic to another, we must find a bijective homomorphism from one to the other.
Conversely, if we know in advance that \(R \cong S\), then \emph{there exists} an isomorphism \(R \rightarrow S\); in general there may be many such mappings.
Isomorphisms formalize and generalize the ``structure preserving relabeling'' we saw between \(\ZZ/(6)\) and \(\ZZ/(2) \oplus \ZZ/(3)\).
In a very concrete sense, if \(R \cong S\), then \(R\) and \(S\) are really ``the same'' ring.
\emph{Isomorphism is structural equality.}
It is not too surprising, then, that ``is isomorphic to'' behaves very much like equality.

\begin{prop} \label{prop:iso-equiv}
For all rings \(R\), \(S\), and \(T\), the following hold.
\begin{proplist*}
\item \label{prop:iso-equiv:refl} \(R \cong R\).
\item \label{prop:iso-equiv:symm} If \(R \cong S\) then \(S \cong R\).
\item \label{prop:iso-equiv:trans} If \(R \cong S\) and \(S \cong T\) then \(R \cong T\).
\end{proplist*}
\end{prop}

A note before we prove this result: \ref{prop:iso-equiv} appears to be saying that \(\cong\) is an equivalence relation.
We have to be careful here, though, because equivalence relations only exist \emph{on sets}.
There is no set of all rings, for the same reason that there is no set of all sets.
And so we cannot say that isomorphism ``is'' an equivalence relation, even though it ``acts like'' and equivalence relation.

\begin{proof}
\begin{inlineproplist}
\item The identity mapping \(\ID : R \rightarrow R\) is a bijective homomorphism by \sref{prop:ring-cat}{id}.
\item Suppose we have an isomorphism \(\varphi : R \rightarrow S\).
Since \(\varphi\) is a bijection, it has an inverse \(\varphi^{-1} : S \rightarrow R\), with the property that \(\varphi \circ \varphi^{-1} = \ID[S]\) and \(\varphi \circ \varphi^{-1} = \ID[R]\) and which is also a bijection.
It thus suffices to show that \(\varphi^{-1}\) is also a ring homomorphism.
To this end, let \(x,y \in S\).
We then have the following.
\begin{eqnarray*}
\varphi^{-1}(x + y) & = & \varphi^{-1}\left(\ID[S](x) + \ID[S](y)\right) \\
 & = & \varphi^{-1}\left( (\varphi \circ \varphi^{-1})(x) + (\varphi \circ \varphi^{-1})(y) \right) \\
 & = & \varphi^{-1}\left( \varphi(\varphi^{-1}(x)) + \varphi(\varphi^{-1}(y)) \right) \\
 & = & \varphi^{-1}\left( \varphi\left( \varphi^{-1}(x) + \varphi^{-1}(y) \right) \right) \\
 & = & (\varphi^{-1} \circ \varphi)\left( \varphi^{-1}(x) + \varphi^{-1}(y) \right) \\
 & = & \ID[R]\left( \varphi^{-1}(x) + \varphi^{-1}(y) \right) \\
 & = & \varphi^{-1}(x) + \varphi^{-1}(y).
\end{eqnarray*}
Thus \(\varphi^{-1}\) preserves plus.
A similar argument shows that \(\varphi^{-1}\) preserves times, and so \(\varphi^{-1}\) is a ring homomorphism.
\item If \(R \cong S\) and \(S \cong T\), then we have bijective homomorphisms \(\varphi : R \rightarrow S\) and \(\psi : S \rightarrow T\).
Now \(\psi \circ \varphi : R \rightarrow T\) is a homomorphism by \sref{prop:ring-cat}{comp}, and since the composite of bijections is bijective, \(\psi \circ \varphi\) is bijective.
Thus \(R \cong T\).
\end{inlineproplist}
\end{proof}

Given two mathematical objects, one of the most basic questions we can ask about them is whether or not they are the same.
To emphasize the importance of this question, we'll give it a fancy name.

\begin{titlebox}{The Distinction Problem}
\begin{center}
Given rings \(R\) and \(S\), can we detect whether or not \(R \cong S\)?
\end{center}
\end{titlebox}

The Distinction Problem is one of the big questions of algebra.
Big enough, in fact, that we cannot hope for a complete answer; the best we can hope for is a sequence of partial answers of increasing power and sophistication.
To this end it is useful to have on hand several properties of rings which are \emph{preserved} by isomorphisms, so that two rings which differ must be nonisomorphic.
For instance, the property ``contains the number 3 as an element'' \emph{is not} preserved by isomorphisms: the elements of isomorphic rings may have nothing to do with each other.
However, the property ``consists of three elements'' \emph{is} preserved by isomorphisms.
Here are some others.

\begin{prop} \label{prop:iso-preserved-props}
Let \(R\) and \(S\) be rings.
If \(R \cong S\), then the following hold.
\begin{proplist*}
\item \(R\) and \(S\) have the same cardinality.
\item \(R\) is commutative if and only if \(S\) is commutative.
\item \(R\) is unital if and only if \(S\) is unital.
\item \(\CHAR{R} = \CHAR{S}\).
\item \(R\) is boolean if and only if \(S\) is boolean.
\end{proplist*}
\end{prop}

Just as important as determining when two rings are different is determining when two rings are the same, or determining what all the different rings are (up to isomorphism).

\begin{titlebox}{The Classification Problem}
\begin{center}
Given a property \(P\), what are the rings (up to isomorphism) with property \(P\)?
\end{center}
\end{titlebox}

For example, property \(P\) may be ``is commutative'', ``is finite'', ``is boolean'', or something more complicated.



%---------%
\Exercises%
%---------%

\begin{exercise}
Prove \autoref{prop:iso-preserved-props}.
\end{exercise}


\begin{exercise}
Are the rings \(\ZZ/(m)\) and \(\ZZ/(n)\) ever isomorphic?
Why or why not?
\end{exercise}


\begin{exercise}
Show that if \(R\) is a finite unital ring of prime order \(p\) then \(R \cong \ZZ/(p)\).
(cf. \eref{exerc:homs-from-zzn}.)
\end{exercise}

\begin{exercise}\label{exerc:direct-sum-basics}
Show that the following properties hold for all rings \(R\), \(S\), \(T\), and \(U\).
\begin{proplist*}
\item \label{exerc:direct-sum-basics:zero} \(R \oplus 0 \cong R\).
\item \label{exerc:direct-sum-basics:comm} \(R \oplus S \cong S \oplus R\).
\item \label{exerc:direct-sum-basics:assoc} \((R \oplus S) \oplus T \cong R \oplus (S \oplus T)\).
\item \label{exerc:direct-sum-well-defined} If \(R \cong T\) and \(S \cong U\), then \(R \oplus S \cong T \oplus U\).
\end{proplist*}
\end{exercise}


\begin{exercise}
Let \(R\) be a ring.
Show that \(\DIAGMAT{2}{R} \cong R \oplus R\).
(cf. \ref{dfn:triangular-diagonal-matrix}.)
\end{exercise}


\begin{exercise}
Let \(2 = \{ 0, 1 \}\) denote a set with two elements, and let \(R\) be a ring.
Show that \(R^2 \cong R \oplus R\), where \(R^2\) is the ring of functions \(2 \rightarrow R\).
\end{exercise}


\begin{exercise}
Let \(R\) be a ring.
Is the ring of sets \(R^\varnothing\) isomorphic to any other named rings?
\end{exercise}


\begin{exercise}
Show that \(\ZZ\) and \(\MAT{2}{\ZZ}\) are not isomorphic.
\end{exercise}


\begin{exercise}
Classify the rings of order 2 up to isomorphism.
\end{exercise}


\begin{exercise}[Rings of prime order.]
Let \(p \in \NN\) be a prime and let \(R\) be a finite ring of order \(p\).
\begin{proplist}
\item Let \(\alpha \in R\) be nonzero.
Show that every element of \(R\) is of the form \(k\alpha\) for some \(0 \leq k < p\).
\item Show that \(R\) is commutative.
\item Show that if \(R\) contains zero divisors, then \(R\) is null.
\end{proplist}
\end{exercise}
