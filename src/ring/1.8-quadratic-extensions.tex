In this section we will pause to construct an important family of example rings.

\begin{dfn}[Quadratic Extension]
Let \(R\) be a commutative ring, and let \(D \in R\).
We define the \emph{quadratic extension}\index{quadratic extension} of \(R\) by \(D\) to be \(R[\sqrt{D}] = R \times R\) as a set, and define operations on \(R[\sqrt{D}]\) as follows.
\begin{eqnarray*}
(r_1,r_2) + (s_1,s_2) & = & (r_1+s_1, r_2+s_2) \\
(r_1,r_2) \cdot (s_1,s_2) & = & (r_1s_1 + Dr_2s_2, r_1s_2 + r_2s_1)
\end{eqnarray*}
We will write the element \((a,b) \in R[\sqrt{D}]\) as \(a + b\sqrt{D}\).
\end{dfn}

The notation \(R[\sqrt{D}]\) is intended to suggest that this structure behaves much like \(R\), but now the element \(D\) has a ``square root''.
It's important to remember that the plus in \(a + b\sqrt{D}\) is a \emph{structural} symbol, not an operation symbol.

\begin{prop} \label{prop:quad-ext-ring}
Let \(R\) be a ring and \(D \in R\).
\begin{proplist*}
\item \(R[\sqrt{D}]\) is a commutative ring.
\item If \(R\) is unital, then \(R[\sqrt{D}]\) is unital.
\item The map \(\iota : R \rightarrow R[\sqrt{D}]\) given by \(\iota(r) = (r,0_R)\) is a ring homomorphism.
If \(R\) is unital, then \(\iota\) is unital.
\end{proplist*}
\end{prop}

The most important use of quadratic extensions is to construct new domains and fields out of old ones.

\begin{prop} \label{prop:quad-ext-dom}
Let \(R\) be a ring and \(D \in R\) nonzero.
\begin{proplist}
\item If \(R\) is a domain and the equation \(Dy^2 = x^2\) has no nonzero solutions in \(R\), then \(R[\sqrt{D}]\) is a domain.
\item If \(R\) is a field and the equation \(D = x^2\) has no nonzero solutions in \(R\), then \(R[\sqrt{D}]\) is a field.
\end{proplist}
\end{prop}

\begin{proof}
\begin{inlineproplist}
\item If \((x,y)\) is such a solution, then \((x,y) \cdot (x,-y) = 0\) and thus \(R\) is not a domain.
\item Follows because every field is a domain.
\end{inlineproplist}
\end{proof}

\begin{prop} \label{prop:sqfree-int}
Let \(D\) be an integer.
We say that \(D\) not equal to 0 or 1 is \emph{squarefree}\index{squarefree!integer} if \(D\) is not divisible by \(a^2\) for any \(a > 1\).
\begin{proplist}
\item \label{prop:sqfree-int:zz-dom} If \(D\) is squarefree, then the equation \(Dy^2 = x^2\) has no nonzero solutions in \(\ZZ\), and thus \(\ZZ[\sqrt{D}]\) is a domain.
\item \label{prop:sqfree-int:qq-field} If \(D\) is squarefree, then the equation \(D = x^2\) has no solutions in \(\QQ\), and thus \(\QQ[\sqrt{D}]\) is a field.
\end{proplist}
\end{prop}

\begin{proof}
\begin{inlineproplist}
\item We consider two cases: if \(|D| \leq 1\) and if \(|D| > 1\).
In the first case, \(D = -1\).
Now if \(Dy^2 = x^2\), then we have \(-y^2 = x^2\); by sign considerations we have \(x = y = 0\).
In the second case, \(D\) is divisible by a prime, say \(p\).
If \(x = 0\), then \(y = 0\), and likewise if \(y = 0\) then \(x = 0\), so we can assume \(x\) and \(y\) are both nonzero.
By the fundamental theorem of arithmetic, \(p\) appears the same number of times in the factorizations of \(Dy^2\) and \(x^2\).
But note that \(p\) must appear an even number of times in \(x^2\), but an odd number of times in \(Dy^2\), a contradiction.
Thus in either case the equation \(Dy^2 = x^2\) has no nonzero solutions.
\item If \(D = x^2\) has a rational solution \(x = p/q\), then \(Dq^2 = p^2\), which contradicts \ref{prop:sqfree-int:zz-dom}.
\end{inlineproplist}
\end{proof}

There are plenty of squarefree integers; in fact these are precisely the integers with no repeated prime factors.
Each such integer yields a different domain \(\ZZ[\sqrt{D}]\) and field \(\QQ[\sqrt{D}]\).
These rings are particularly nice for a few reasons.
They are easy to compute in, and while similar to \(\ZZ\) and \(\QQ\), can exhibit wildly different behavior, as we will see in the next chapter.
For now, these will serve as a large class of new examples of domains and fields.



%---------%
\Exercises%
%---------%

\begin{exercise}
Prove Proposition \ref{prop:quad-ext-ring}.
\end{exercise}

\begin{exercise}
Let \(R = \ZZ/(3)\).
\begin{proplist}
\item Show that \(2x^2 = y^2\) has no nonzero solutions in \(R\).
(Hint: When in doubt, use brute force.)
\item Conclude that \(R[\sqrt{2}]\) is a field with 9 elements.
Draw the cayley tables of this field.
\end{proplist}
\end{exercise}

\begin{exercise}
Elements of \(\ZZ[\sqrt{-1}]\) are sometimes called \emph{gaussian integers}.
\begin{proplist}
\item Determine which elements of \(\ZZ[\sqrt{-1}]\) are units.
\end{proplist}
\end{exercise}

\begin{exercise}
Show that \(\CHAR{R[\sqrt{D}]} = \CHAR{R}\).
\end{exercise}

\begin{exercise}
Let \(R\) be a ring and \(S \subseteq R\) a subring, with \(D \in S\).
Show that \(S[\sqrt{D}]\) is a subring of \(R[\sqrt{D}]\).
\end{exercise}

\begin{exercise}
Let \(\varphi : R \rightarrow S\) be a ring homomorphism with \(D \in R\) and \(E = \varphi(D)\).
Show that there is a unique ring homomorphism \(\Phi : R[\sqrt{D}] \rightarrow S[\sqrt{E}]\) such that the following diagram commutes.
\begin{center}
\begin{tikzcd}
R \arrow[r,"\varphi"] \arrow[d, "\iota"']    & S \arrow[d, "\iota"] \\
R[\sqrt{D}] \arrow[r, "\Phi"'] & S[\sqrt{E}]
\end{tikzcd}
\end{center}
\end{exercise}

\begin{exercise}
Let \(R\) be a ring and \(D,E \in R\).
Show that \( R[\sqrt{DE^2}] \cong R[\sqrt{D}] \).
\end{exercise}
