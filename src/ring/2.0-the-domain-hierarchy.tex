In \autoref{chap:rings} we defined a class of structures, called \emph{rings}, which generalize the basic arithmetic on \(\ZZ\) and \(\ZZ/(n)\). The ring of integers comes equipped with some extra technology which turns out to be useful: specifically, the \textbf{division algorithm}, \textbf{unique factorization}, and the existence of \textbf{greatest common divisors}. In this chapter we will see how this technology generalizes (or not!) to other rings -- to integral domains, in particular. At the risk of giving away the punchline, we will see that the technology on \(\ZZ\), when reimagined in arbitrary domains, gives rise to a \textbf{hierarchy} of families of domains of increasing algorithmic power.
