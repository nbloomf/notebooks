In a general ring with 1, or even a general domain, elements typically do not have multiplicative inverses. Those which do are called units and are very special. In this section we will see how a domain can be ``extended'' to a larger ring so that any given element can be made into a unit. First we need a definition.

\begin{dfn}[Multiplicative Subset] \label{dfn:multiplicative-subset}
Let \(R\) be a domain and \(S \subseteq R\). We say that \(S\) is a \emph{multiplicative subset} (or a set of \emph{denominators}) of \(R\) if \(0 \notin R\) and if \(S\) is closed under multiplication. \index{multiplicative subset}
\end{dfn}

Domains have plenty of multiplicative sets. For instance, the set of all nonzero elements is multiplicative. If \(a \in R\) is not zero, then the set \(S = \{ 1, a, a^2, a^3, \ldots \}\) of powers of \(a\) is multiplicative.

Here is the punch line of this section.

\begin{framed}
If \(S \subseteq R\) is a multiplicative subset, then we can construct a new ring, \(T\), which contains \(R\) as a subset, but in which the elements of \(S\) are units.
\end{framed}

\begin{prop}
Let \(R\) be a domain and \(S \subseteq R\) a multiplicative subset. We define a relation \(\Phi\) on the cartesian product \(S \times R\) as follows: \[ (s_1, r_1) \Phi (s_2, r_2) \quad \mathrm{iff} \quad r_1s_2 = r_2s_1. \]
This relation \(\Phi\) is an equivalence.
\end{prop}

\begin{proof} \mbox{}
\begin{itemize}
\item \(rs = rs\) for all \(r \in R\) and \(s \in S\), so that \((s,r) \Phi (s,r)\).
\item Suppose \((s_1,r_1) \Phi (s_2,r_2)\). Then \(r_1s_2 = r_2s_1\), so that \(r_2s_1 = r_1s_2\), and thus \((s_2,r_1) \Phi (s_1,r_1)\).
\item Suppose \((s_1, r_1) \Phi (s_2, r_2)\) and \((s_2, r_2) \Phi (s_3, r_3)\). Now \(r_1s_2 = r_2s_1\) and \(r_2s_3 = r_3s_2\). We then have \(r_1s_2r_2s_3 = r_2s_1r_3s_2\); rearranging (since \(R\) is commutative) and using cancellation, we have \(r_1s_3 = r_3s_1\). So \((s_1, r_1) \Phi (s_3, r_3)\) as needed. \qedhere
\end{itemize}
\end{proof}

Since \(\Phi\) is an equivalence, it induces a partition on the set \(S \times R\). We will denote this quotient set \(\LOCALIZE{R}{S} = (S \times R)/\Phi\), and denote the equivalence class of \((s,r)\) by \(\frac{r}{s}\).

\begin{prop}
Let \(R\) be a domain with \(S \subseteq R\) a multiplicative subset. Define operations \(+\) and \(\cdot\) on \(\LOCALIZE{R}{S}\) as follows. \[ \frac{r_1}{s_1} + \frac{r_2}{s_2} = \frac{r_1s_2 + r_2s_1}{s_1s_2} \quad \mathrm{and} \quad \frac{r_1}{s_1} \cdot \frac{r_2}{s_2} = \frac{r_1r_2}{s_1s_2}. \] Then we have the following.
\begin{proplist}
\item \(+\) and \(\cdot\) are well-defined.
\item \(\LOCALIZE{R}{S}\), with these operations, is an integral domain, which we call the \emph{localization} of \(R\) \emph{at} \(S\).
\item If \(t \in S\), then the mapping \(\iota : R \rightarrow S^{-1}R\) given by \(\iota(r) = \frac{rt}{t}\) is an injective ring homomorphism, and \(\iota(t)\) is a unit in \(S^{-1}R\).
\end{proplist}
\end{prop}

\begin{proof}
(super tedious)
\end{proof}

So \(S^{-1}R\) is a new ring which contains a ``copy'' (homomorphic image) of \(R\), within which the elements of \(S\) become units.

\begin{dfn}
Let \(R\) be a domain and let \(D = \{ x \in R \mid x \neq 0 \}\) be the multiplicative subset of all nonzero elements of \(R\). Then the localization \(D^{-1}R\) is a field, called the \emph{field of fractions} of \(R\).
\end{dfn}

For example, \(\QQ\) is properly defined as the field of fractions of \(\ZZ\).

\subsection*{Special things}

\begin{prop}
If \(R\) is a UFD and \(S \subseteq R\) any multiplicative set, then \(S^{-1}R\) is also a UFD.
\end{prop}

\begin{proof}
(type this)
\end{proof}

\begin{prop}
If \(R\) is a Euclidean domain and \(S \subseteq R\) any multiplicative set, then \(S^{-1}R\) is a Euclidean domain.
\end{prop}

\begin{proof}
(type this)
\end{proof}

%---------%
\Exercises%
%---------%

\begin{exercise}
Let \(R\) be a domain and \(S,T \subseteq R\) multiplicative sets. Show that \(ST = \{ st \mid s \in S, t \in T \}\) is multiplicative.
\end{exercise}

\begin{exercise}
Let \(R\) be a GCD domain, with \(F\) its field of fractions. An element \(\frac{a}{b} \in F\) is said to be \emph{reduced} (or in \emph{lowest terms}) if \(\GCD{a}{b} = 1\).
\begin{proplist}
\item Show that every element of \(F\) has a reduced representative.
\item Show that reduced fractions are unique in the following sense: If \(\frac{a}{b} = \frac{c}{d}\) are both reduced, then \(c = au\) and \(d = bu\) for some unit \(u\).
\end{proplist}
\end{exercise}

\begin{enumerate}
\item (do stuff in \(\ZZ[\frac{1}{2}]\))

\item \textbf{Universal Property of Localization.} Let \(R\) be a ring and let \(D \subseteq R\) be a multiplicative subset. Suppose \(\varphi : R \rightarrow S\) is a ring homomorphism (with \(S\) unital) such that \(\varphi(d)\) is a unit in \(S\) for all \(d \in D\). Then there is a unique ring homomorphism (@@@)
\end{enumerate}
