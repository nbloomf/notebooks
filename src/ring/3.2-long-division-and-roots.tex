As we will see, there are some fundamental similarities between polynomials and integers.
The first important result in this direction is the division algorithm for polynomials.

\begin{prop} \label{prop:poly-long-div}
Let \(R\) be a commutative unital ring, and let \(a(x), b(x) \in R[x]\) be polynomials such that \(b(x) \neq 0\) and the leading coefficient of \(b\) is a unit in \(R\).
Then there exist polynomials \(q(x), r(x) \in R[x]\) such that \(a(x) = q(x)b(x) + r(x)\) and either \(r(x) = 0\) or \(\deg{r} < \deg{b}\).
\end{prop}

\begin{proof}
If \(a(x) = 0\), set \(q(x) = r(x) = 0\).
Suppose now that \(a(x) \neq 0\); we proceed by strong induction on \(\deg{a}\).
\begin{itemize}
\item \textbf{Base case.} If \(\deg{a} = 0\), then \(a(x) = a_0\) is a constant.
If \(\deg{b} = 0\), then \(b(x) = b_0\) is also a constant, and in this case \(b_0\) is the leading coefficient of \(b\), hence a unit.
Let \(q(x) = a_0b_0^{-1}\) and \(r(x) = 0\).
If \(\deg{b} > 0\), let \(q(x) = 0\) and \(r(x) = a_0\).
Then \(a(x) = q(x)b(x) + r(x)\) and we have \(\deg{b} \geq 1 > 0 = \deg{r}\).
\item \textbf{Inductive Step.} Suppose the result holds for all polynomials \(\overline{a}(x)\) of degree strictly less than \(n\), where \(n > 0\), and suppose that \(a(x)\) has degree \(n\).
If \(\deg{a} < \deg{b}\), let \(q(x) = 0\) and \(r(x) = a(x)\).
Now suppose instead that \(\deg{a} \geq \deg{b}\).
Let \(m = \deg{b}\) and let \(a_n\) be the leading coefficient of \(a(x)\) and \(b_m\) the leading coefficient of \(b(x)\) (which is a unit).
Define \(\overline{a}(x) = a(x) - a_nb_m^{-1}x^{n-m}b(x)\).
Note that \(\deg{\overline{a}} < \deg{a}\).
By the inductive hypothesis, we have \(\overline{q}(x), r(x) \in R[x]\) such that \(\overline{a}(x) = \overline{q}(x)b(x) + r(x)\) and either \(r(x) = 0\) or \(\deg{r} < \deg{b}\).
Define \(q(x) = \overline{q}(x) + a_nb_m^{-1}x^{n-m}\).
Now 
\begin{eqnarray*}
a(x) - q(x)b(x) & = & a(x) - \overline{q}(x)b(x) - a_nb_m^{-1}x^{n-m}b(x) \\
 & = & \overline{a}(x) - \overline{q}(x)b(x) \\
 & = & r(x)
\end{eqnarray*}
as needed.
\end{itemize}
By induction, the result holds for all \(n\).
\end{proof}

Note that this result is very general; the coefficient ring \(R\) is only required to be commutative and unital.
In fact, with slight adjustments to the punchline, even the C and U conditions on \(R\) can be dropped (see the exercises).
For us, for now, the most important situation will be when \(R\) is a field, because in this case \(R[x]\) is a Euclidean domain.

\begin{cor}\label{cor:polys-over-a-field}
Suppose \(F\) is a field.
\begin{proplist}
\item \(F[x]\) is a Euclidean domain with norm \(N(a) = 2^{\deg{a}}\).
In particular, \(F[x]\) is also a UFD and a GCD domain.
\item \(p(x) \in F[x]\) is irreducible iff \(p(x)\) cannot be factored as a product of nonconstants.
\item If \(p(x)\) has degree 1, then \(p(x)\) is irreducible in \(R[x]\).
\end{proplist}
\end{cor}

In particular, if \(R\) is a domain, then \(R[x]\) is a subring of a Euclidean domain: namely, \(F[x]\), where \(F\) is the field of fractions of \(R\).
This means that \(R[x]\) immediately has a much stronger structure than the average integral domain.
For example: carrying out long division over \(F\) and then clearing denominators gives the following result.

\begin{cor}\label{cor:poly-long-div-over-domain}
If \(R\) is a domain and \(a,b \in R[x]\) with \(b(x) \neq 0\), then there exist \(q,r \in R[x]\) and \(k \in R\) such that \(ka(x) = q(x)b(x) + r(x)\) and either \(r = 0\) or \(r \neq 0\) and \(\deg{r} < \deg{b}\).
\end{cor}

So far, we've been thinking of polynomials as objects in their own right.
But we can also treat them like functions in the usual sense by ``plugging in'' ring elements for the variable.

\begin{dfn} \label{dfn:evaluation-map}
Given a polynomial \(p(x) = \sum_{i=0}^n a_nx^n\) in \(R[x]\), \(R\) a commutative unital ring, we define the \emph{evaluation map}\index{evaluation map} \(\POLYEVAL{p} : R \rightarrow R\) by \(\POLYEVAL{p}(r) = \sum_{i=0}^n a_i r^i\).
We say that an element \(r \in R\) is a \emph{root}\index{root} of \(p(x)\) if \(\POLYEVAL{p}(r) = 0_R\).
\end{dfn}

There is an important distinction between the polynomial \(p\) and its corresponding evaluation map \(\POLYEVAL{p}\), although in practice this distinction is easy to blur.

\begin{prop}
Let \(R\) be a commutative unital ring.
Then \(\POLYEVAL{p+q}(r) = \POLYEVAL{p}(r) + \POLYEVAL{q}(r)\) and \(\POLYEVAL{pq}(r) = \POLYEVAL{p}(r) \POLYEVAL{q}(r)\).
\end{prop}

\begin{prop}[Factor Theorem]\label{prop:factor-theorem}\index{Factor Theorem}
Let \(R\) be a commutative unital ring, with \(p(x) \in R[x]\) and \(a \in R\).
Then \(a\) is a root of \(p(x)\) if and only if \(x-a\) divides \(p(x)\) in \(R[x]\).
\end{prop}

\begin{proof}
Certainly if \(x-a\) divides \(p(x)\) then \(a\) is a root of \(p\).
Conversely, suppose \(a\) is a root of \(p(x)\).
Now \(b(x) = x - a\) is monic, so by the polynomial long division algorithm we have \(q(x), r(x) \in R[x]\) such that \(p(x) = q(x)(x-a) + r(x)\) and either \(r(x) = 0\) or \(\deg{r} < 1\).
If \(r(x) \neq 0\), then \(r(x) = r_0\) is a constant.
Evaluating at \(a\) we have \(p(a) = r_0\), a contradiction.
So \(r(x) = 0\) and \(x-a\) divides \(p(x)\).
\end{proof}

\begin{cor}
Let \(R\) be a domain with \(p(x) \in R[x]\) a nonzero polynomial of degree \(d\).
Then \(p(x)\) has at most \(d\) roots in \(R\), counting multiplicity.
\end{cor}

\begin{cor}
Let \(R\) be a domain with \(p(x), q(x) \in R[x]\) polynomials of degree at most \(d\).
If there exist \(d+1\) distinct elements \(a_i \in R\) such that \(p(a_i) = q(a_i)\), then \(p(x) = q(x)\).
\end{cor}

\begin{proof}
Let \(s(x) = p(x) - q(x)\); note that \(s\) has degree at most \(d\).
Now the \(a_i\) are \(d+1\) distinct roots of \(s\).
If \(s(x) \neq 0\), this contradicts the previous corollary.
\end{proof}

\begin{prop}
Let \(R\) be a domain and \(p(x) \in R[x]\) a polynomial of degree 2 or 3.
Then \(p(x)\) cannot be written as a product of nonconstants in \(R[x]\) if and only if \(p(x)\) does not have a root in \(R\).
\end{prop}

\begin{proof}
Note that if \(p(x) = a(x)b(x)\), then \(\deg{a} + \deg{b}\) is either 2 or 3.
Thus \(p(x)\) is a product of nonconstants iff it has a factor of degree 1.
But \(p(x)\) has a factor of degree 1 iff it has a root in \(R\).
\end{proof}

\begin{examples}
\item The degree 2 polynomial \(p(x) = x^2 + 1\) is irreducible over \(\ZZ/(3)\) since it has no roots.
\end{examples}



\begin{prop}
Let \(R\) be a domain and \(q(x) = \sum_{i=0}^n \in R[x]\).
Suppose \(p \in R\) is prime such that \(p \not| a_n\), \(p|a_i\) for each \(0 \leq i < n\), and \(p^2 \not| a_0\).
Then \(q(x)\) cannot be factored as a product of nonconstants.
\end{prop}

\begin{proof}
Suppose we have \[ q(x) = b(x)c(x) = \left(\sum_{i}b_ix^i\right)\left(\sum_{j}c_jx^j\right) = \sum_{k}\left( \sum_{i+j = k}b_ic_j \right)x^k. \]
Note that \(q_0 = b_0c_0\).
Since \(p | b_0c_0\) and \(p^2 \not| b_0c_0\), \(p\) divides exactly one of \(b_0\) and \(c_0\); suppose WLOG that \(p|b_0\), so \(p \not| c_0\).
Letting \(n = \deg{q}\), \(h = \deg{b}\), and \(k = \deg{c}\), we have \(q_n = b_hc_k\), and since \(p \not| q_n\), \(p \not| b_h\).
Let \(i\) be minimal such that \(p\not|b_i\).
(Note that \(0 < i \leq \deg{b} \leq n\).)

We now have \[ q_i = b_ic_0 + b_{i-1}c_1 + \cdots + b_{i-t}c_t \] for some \(t\).
If \(i < n\), then \(p|q_i\), and by construction, \(p|b_j\) for \(j < i\).
Thus \(p|b_ic_0\), and since \(p \not| c_0\) we have \(p|b_i\) -- a contradiction.
So \(i = n\) and thus \(\deg{b} = n = \deg{q}\).
But \(\deg{q} = \deg{b} + \deg{c}\), so that \(\deg{c} = 0\); hence \(c\) is a constant.
\end{proof}



%---------%
\Exercises%
%---------%

\begin{exercise}[Hensel's Lemma]
(@@@)
\end{exercise}
