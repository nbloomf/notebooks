In this section we establish some important results about irreducibility and factorization for polynomials over a GCD domain.

\begin{prop}[Gauss' Lemma -- Part II]
Let \(R\) be a GCD domain with field of fractions \(F\), and let \(p(x) \in R[x]\) have positive degree.
Then \(p(x)\) is irreducible in \(R[x]\) if and only if \(p(x)\) is irreducible in \(F[x]\) and primitive in \(R[x]\).
\end{prop}

\begin{proof}
(type this)
\end{proof}

Combined with Eisenstein's criterion, Gauss' lemma provides an easy-to-apply irreducibility criterion.

\begin{cor}
If \(p(x) \in R[x]\) (\(R\) a GCD domain) is Eisenstein and primitive, then \(p(x)\) is irreducible in \(R[x]\).
\end{cor}

\begin{proof}
Suppose \(p(x) = a(x)b(x)\) with \(a,b \in R[x]\).
Since \(p\) is Eisenstein, WLOG \(a(x)\) is a constant; say \(a(x) = a_0\).
Now \(a_0|p\) in \(R[x]\), so that \(a_0|\CONTENT{p}\) in \(R\).
Since \(p(x)\) is primitive, \(a\) is a unit in \(R\), hence a unit in \(R[x]\).
So \(p(x)\) is irreducible in \(R[x]\).
\end{proof}

This criterion can be used to quickly verify that a given polynomial is irreducible -- when it applies.
Unfortunately there are plenty of irreducible polynomials to which this criterion does not apply.
For example, \(p(x) = x^2 + 1\) is primitive in \(\ZZ[x]\), and in fact is irreducible.
But it is not Eisenstein at any prime.

\begin{prop}[Rational Root Theorem]
Let \(R\) be a GCD domain with fraction field \(F\).
Suppose \(p(x) \in R[x]\).
Let \(\frac{u}{v} \in F\) be a fraction in lowest terms; that is, \(\GCD{u}{v} = 1\) in \(R\).
If \(\frac{u}{v}\) is a root of \(p(x)\), then \(u\) divides the constant coefficient of \(p\), and \(v\) divides the leading coefficient of \(p\).
\end{prop}

The Rational Root Theorem allows us to restrict the possible ``rational roots'' (that is, those in \(F\), or equivalently factors over \(R\) of the form \(ax-b\)) to a finite list of possibilities.
For example, applying this theorem to \(p(x) = x^2 + 1\) we see that the only possible rational roots of \(p(x)\) are \(\pm 1\), and it is easily seen that neither of these is a root.
So by (???) this \(p\) is irreducible in \(\ZZ[x]\).

%---------%
\Exercises%
%---------%

\begin{exercise}
Let \(R\) be a GCD domain with \(p(x), q(x) \in R[x]\) so that \(q\) is irreducible (hence prime), and let \(k\) be a natural number.
Show that \(q^{k+1}\) divides \(p\) in \(R[x]\) iff \(q|p\) and \(q^k|p'\) in \(R[x]\).
In particular, show that \(p\) is squarefree iff \(\GCD{p}{p'} = 1\).
\end{exercise}
