The ideals of a ring enjoy an arithmetic of their own.

\begin{dfn}
If \(R\) is a ring, \(\IDEALS{R}\) denotes the set of all ideals of \(R\).
\end{dfn}

\begin{prop}
Let \(R\) be a ring and \(I, J \subseteq R\) ideals.
We can construct two new subsets \(I+J\) and \(IJ\) of \(R\) as follows.
\begin{eqnarray*}
I + J & = & \{ a+b \mid a \in I, b \in J \},\ \mathrm{and} \\
I   J & = & \left\{ \sum_{i=0}^n a_i b_i \mid n \in \NN, a_i \in I, b_i \in J \right\}.
\end{eqnarray*}
Moreover, \(I+J\) and \(IJ\) are both ideals of \(R\), called an \emph{ideal sum}\index{sum!of ideals} and \emph{ideal product}\index{product!of ideals}, respectively.
\end{prop}

\begin{prop}
Let \(R\) be a ring and \(I,J,K \subseteq R\) ideals.
Then we have the following.
\begin{proplist}
\item \(IJ \subseteq I \cap J\) and \(I,J \subseteq I+J\).
\item \(I+(J+K) = (I+J)+K\).
\item \(I+0 = 0+I = I\).
\item \(I+R = R+I = R\).
\item \(I(JK) = (IJ)K\).
\item \(I0 = 0I = 0\).
\item \(IR = RI = I\).
\item \(I(J+K) = IJ+IK\) and \((I+J)K = IK+JK\).
\end{proplist}
\end{prop}



%---------%
\Exercises%
%---------%

