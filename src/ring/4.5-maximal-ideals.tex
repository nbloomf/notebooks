\begin{dfn}
Let \(R\) be a ring.
An ideal \(I \subseteq R\) is called \emph{maximal}\index{maximal!ideal} if there does not exist an ideal \(J \subseteq R\) such that \(I \subsetneq J \subsetneq R\).
\end{dfn}

\begin{prop}
A ring \(R\) is a field if and only if the zero ideal is maximal.
\end{prop}

\begin{proof}
First let \(R\) be a field, and suppose \(I \subsetneq R\) is a nonzero ideal; say \(x \in I\) is not zero.
Now \(x\) is a unit in \(R\), so that \(I = R\).
Conversely, suppose the zero ideal is maximal in \(R\).
(@@@)
\end{proof}

\begin{prop}
An ideal \(I\) in a ring \(R\) is maximal if and only if \(R/I\) is a field.
\end{prop}

\begin{prop}[Krull]
Every unital ring has at least one maximal ideal.
\end{prop}

\begin{proof}
(@@@)
\end{proof}
