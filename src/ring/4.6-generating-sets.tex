\begin{prop}
Let \(R\) be a ring, and let \(\mathcal{I}\) be a collection of ideals of \(R\).
Then \(\bigcap \mathcal{I}\) is an ideal of \(R\).
\end{prop}

Not every subset of \(R\) is an ideal; in fact most aren't.
However, \emph{every subset of \(R\) is contained in a unique smallest ideal}.

\begin{prop}[Generated Ideals]
Let \(R\) be a ring and \(A \subseteq R\) a subset.
We define the set \((A)\) by \[ (A) = \bigcap \left\{ I \mid I \subseteq R\ \mathrm{is\ an\ ideal\ and}\ A \subseteq I \right\}. \]
Then the following hold.
\begin{proplist}
\item \((A)\) is an ideal of \(R\).
\item If \(A\) is an ideal of \(R\), then \((A) = A\).
\item \(A \subseteq (A)\).
\item If \(I\) is an ideal of \(R\) and \(A \subseteq R\), then \((A) \subseteq I\).
\end{proplist}
We call \((A)\) the ideal of \(R\) \emph{generated by} \(A\).
If \(I\) is an ideal and \(I = (A)\), we say that \(A\) is a \emph{generating set}\index{generating set!of an ideal} for \(I\).
\end{prop}

\begin{prop}
Let \(R\) be a ring and \(A\) and \(B\) be subsets of \(R\).
Then the following hold.
\begin{proplist}
\item \((A) + (B) = (A \cup B)\)
\item \((A)(B) = (ab \mid a \in A, b \in B)\)
\end{proplist}
\end{prop}

\begin{dfn}
Let \(I\) be an ideal of a ring.
\begin{enumerate}
\item We say \(I\) is \emph{finitely generated} if there is a finite set \(A\) such that \(I = (A)\).
\item We say a subset \(A\) is a \emph{minimal generating set} of \(I\) if \(I = (A)\) and whenever \(B \subsetneq A\) is a proper subset, \((B) \subsetneq (A)\) is also a proper subset.
\end{enumerate}
\end{dfn}

Important note: ``minimal'' here does not mean \emph{smallest size}, but rather \emph{contains no redundant elements}.
In general an ideal will have many minimal generating sets, and these may have very different cardinalities.
For example, the ideal \((3) \subseteq \ZZ\) is minimally generated by the set \(\{3\}\), but also by the set \(\{6,15\}\).

\begin{prop}
Let \(R\) be a commutative unital ring and \(A\) a nonempty subset of \(R\).
Then \[ (A) = \left\{ \sum_{i=1}^n r_ia_i \mid n \in \NN, r_i \in R, a_i \in A \right\}. \]
That is, in a commutative unital ring, the ideal generated by \(A\) consists of all finite \(R\)-linear combinations of elements of \(A\).
\end{prop}

\begin{proof}
Let \(I\) denote the set of all finite \(R\)-linear combinations of elements of \(A\).
Note that if \(a \in A\), then because \(R\) is unital we have \(a = 1_R \cdot a \in I\).
In particular, \(A \subseteq I\).
Because \(A\) is not empty, \(I\) is also not empty.
We can see that \(I\) is closed under subtraction, and because \(R\) is commutative, \(I\) also absorbs elements of \(R\) under multiplication from either side.
So \(I\) is an ideal of \(R\) by the Ideal Criterion, and since \(A \subseteq I\) we have \((A) \subseteq I\) by definition.
Conversely, we can see that \(I \subseteq (A)\) because \((A)\) is an ideal containing \(A\).
\end{proof}



%---------%
\Exercises%
%---------%

\begin{exercise}
Show that for any natural number \(k \geq 1\), the ideal \((3)\) in \(\ZZ\) has a minimal generating set containing \(k\) elements.
\end{exercise}
