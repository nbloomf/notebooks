\begin{prop}
Let \(R\) be a ring, and let \(\mathcal{I}\) be a collection of ideals of \(R\). Then \(\bigcap \mathcal{I}\) is an ideal of \(R\).
\end{prop}

Not every subset of \(R\) is an ideal; in fact most aren't. However, \emph{every subset of \(R\) is contained in a unique smallest ideal}.

\begin{prop}[Generated Ideals]
Let \(R\) be a ring and \(A \subseteq R\) a subset. We define the set \((A)\) by \[ (A) = \bigcap \{ I \mid I \subseteq R\ \mathrm{is\ an\ ideal\ and}\ A \subseteq I \}. \]
\begin{enumerate}
\item \((A)\) is an ideal of \(R\).
\item If \(A\) is an ideal of \(R\), then \((A) = A\).
\item \(A \subseteq (A)\).
\item If \(I\) is an ideal of \(R\) and \(A \subseteq R\), then \((A) \subseteq I\).
\end{enumerate}
We call \((A)\) the ideal of \(R\) \emph{generated by} \(A\). If \(I\) is an ideal and \(I = (A)\), we say that \(A\) is a \emph{generating set} for \(I\).
\end{prop}

\begin{prop} \mbox{}
\begin{enumerate}
\item \((A) + (B) = (A \cup B)\)
\item \((A)(B) = (ab \mid a \in A, b \in B)\)
\end{enumerate}
\end{prop}

\begin{dfn}
Let \(I\) be an ideal of a ring.
\begin{enumerate}
\item We say \(I\) is \emph{finitely generated} if there is a finite set \(A\) such that \(I = (A)\).
\item We say a subset \(A\) is a \emph{minimal generating set} of \(I\) if \(I = (A)\) and whenever \(B \subsetneq A\) is a proper subset, \((B) \subsetneq (A)\) is also a proper subset.
\end{enumerate}
\end{dfn}

Important note: ``minimal'' here does not mean ``smallest size'', but rather ``contains no redundant elements''. In general, an ideal has many minimal generating sets, and these may have very different cardinalities. For example, the ideal \((3) \subseteq \ZZ\) is minimally generated by the set \(\{3\}\), but also by the set \(\{6,15\}\).

\begin{prop}
Let \(R\) be a commutative unital ring and \(A\) a subset of \(R\). Then \[ (A) = \{ \sum_{i=0}^n r_ia_i \mid n \in \NN, r_i \in R, a_i \in A \}. \] That is, in a commutative unital ring, the ideal generated by \(A\) consists of all finite \(R\)-linear combinations of elements of \(A\).
\end{prop}

\begin{proof}
(type this)
\end{proof}



%---------%
\Exercises%
%---------%

\begin{exercise}
(@@@) Find an example of an ideal with minimal generating sets of arbitrarily large cardinality.
\end{exercise}
