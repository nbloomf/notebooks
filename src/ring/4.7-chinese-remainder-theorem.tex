\begin{dfn}[Comaximal Ideals]
Let \(R\) be a ring.
We say that two ideals \(I\) and \(J\) in \(R\) are \emph{comaximal}\index{comaximal!ideal} if \(I+J = R\).
\end{dfn}

\begin{prop}[A comaximal condition for unital rings]
Let \(R\) be a unital ring, with ideals \(I\) and \(J\).
Then \(I\) and \(J\) are comaximal if and only if there exist \(a \in I\) and \(b \in J\) such that \(a + b = 1_R\).
\end{prop}

\begin{proof}
(@@@)
\end{proof}

\begin{prop}
Let \(R\) be a unital ring, and suppose we have (two-sided) ideals \(I_1, \ldots, I_k, J\) in \(R\) such that \(I_i\) and \(J\) are comaximal for each \(i\).
Then \(\bigcap_{i=1}^k I_i\) and \(J\) are comaximal.
\end{prop}

\begin{proof}
We proceed by induction on \(k\).
For the base case \(k = 1\), the result is clear.
Suppose the result holds for all families of \(k\) ideals, and let \(I_1, \ldots, I_{k+1}\) be a family of \(k+1\) ideals such that the hypothesis holds.
Now since \(I_1\) and \(J\) are comaximal and \(R\) is unital, we have \(a_1 + b_1 = 1\) for some \(a_1 \in I_1\) and \(b_1 \in J\).
Similarly, \(a_2 + b_2 = 1\) for some \(a_2 \in I_2\) and \(b_2 \in J\).
Note then that \[ 1 = (a_1 + b_1)(a_2 + b_2) = a_1a_2 + a_1b_2 + b_1a_2 + b_1b_2. \]
(\(R\) is not necessarily commutative.)
In particular, we have \(a_1a_2 \in I_1 \cap I_2\), and \(a_1b_2 + b_1a_2 + b_1b_2 \in J\).
Thus \(1 \in (I_1 \cap I_2) + J\), so that \(I_1 \cap I_2\) and \(J\) are comaximal.
Now \(I_1 \cap I_2, I_3, \ldots, I_{k+1}\) is a family of \(k\) ideals which are each comaximal to \(J\), and by the induction hypothesis the result holds.
\end{proof}

\begin{prop}[Chinese Remainder Theorem]
Let \(R\) be a unital ring and let \(I_1, \ldots, I_k\) be a family of pairwise comaximal ideals in \(R\).
Then \[ R/(I_1 \cap \cdots \cap I_k) \cong R/I_1 \oplus \cdots \oplus R/I_k. \]
\end{prop}

\begin{proof}
For each \(t\), we have the natural projection \(\pi_t : R \rightarrow R/(I_t)\) as well as the coordinate projection \(\kappa_t : R/(I_t) \rightarrow \bigoplus_t R/I_t\).
By the Universal Property of direct sums, we have a unique homomorphism \(\Theta : R \rightarrow \bigoplus_t R/(I_t)\) such that \(\kappa_t \circ \Theta = \pi_t\).

We claim that this \(\Theta\) is surjective.
To this end, let \((a_t + I_t)_{t=1}^k \in \bigoplus_t R/(I_t)\).
Note that for each \(t\), the ideals \(I_t\) and \(\bigcap_{i=1, i \neq t}^k\) are comaximal by the lemma.
Say \(1 = N_t + M_t\), where \(N_t \in I_t\) and \(M_t \in \bigcap_{i=1,i \neq t}^k\).
Now let \(x = \sum_{t=1}^k a_tM_t \in R\).
Evidently, then, we have \(\Theta(x) = (a_t + I_t)_{t=1}^k\) as needed.

Next we claim that \(\KER{\Theta} = \bigcap_{t=1}^k I_t\).
To this end, (@@@).
By the First Isomorphism Theorem, then, there is a unique map \(\Omega\) such that the following diagram commutes.
\begin{center}
\begin{tikzcd}
 & R/I_t \\
R \arrow[r,"\Theta"'] \arrow[d, "\pi"'] \arrow[ur,"\pi_t"] & \bigoplus_t R/I_t \arrow[u,"\kappa_t"] \\
R/\left( \bigcap_t I_t \right) \arrow[ur, "\Omega"'] &
\end{tikzcd}
\end{center}
\end{proof}

\begin{cor}
Let \(n_1, \ldots, n_k\) be pairwise coprime integers, and let \(a_t\) be a residue mod \(n_t\) for each \(t\).
Then the system of congruences \[ \left\{ \begin{array}{rcl} x & \equiv & a_1 \pmod{n_1} \\ x & \equiv & a_2 \pmod{n_2} \\ & \vdots & \\ x & \equiv & a_k \pmod{n_k} \end{array} \right. \] has a solution in \(\ZZ\), and in fact this solution is unique mod \(\prod n_t\).
\end{cor}
