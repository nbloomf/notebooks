\begin{dfn}[Comaximal Ideals]
Let \(R\) be a ring.
We say that two ideals \(I\) and \(J\) in \(R\) are \emph{comaximal}\index{comaximal!ideal} if \(I+J = R\).
\end{dfn}

\begin{prop}[A comaximal condition for unital rings]
Let \(R\) be a unital ring, with ideals \(I\) and \(J\).
Then \(I\) and \(J\) are comaximal if and only if there exist \(a \in I\) and \(b \in J\) such that \(a + b = 1_R\).
\end{prop}

\begin{proof}
(@@@)
\end{proof}

\begin{prop}
Let \(R\) be a unital ring, and suppose we have (two-sided) ideals \(I_1, \ldots, I_k, J\) in \(R\) such that \(I_i\) and \(J\) are comaximal for each \(i\).
Then \(\bigcap_{i=1}^k I_i\) and \(J\) are comaximal.
\end{prop}

\begin{proof}
We proceed by induction on \(k\).
For the base case \(k = 1\), the result is clear.
Suppose the result holds for all families of \(k\) ideals, and let \(I_1, \ldots, I_{k+1}\) be a family of \(k+1\) ideals such that the hypothesis holds.
Now since \(I_1\) and \(J\) are comaximal and \(R\) is unital, we have \(a_1 + b_1 = 1\) for some \(a_1 \in I_1\) and \(b_1 \in J\).
Similarly, \(a_2 + b_2 = 1\) for some \(a_2 \in I_2\) and \(b_2 \in J\).
Note then that \[ 1 = (a_1 + b_1)(a_2 + b_2) = a_1a_2 + a_1b_2 + b_1a_2 + b_1b_2. \]
(\(R\) is not necessarily commutative.)
In particular, we have \(a_1a_2 \in I_1 \cap I_2\), and \(a_1b_2 + b_1a_2 + b_1b_2 \in J\).
Thus \(1 \in (I_1 \cap I_2) + J\), so that \(I_1 \cap I_2\) and \(J\) are comaximal.
Now \(I_1 \cap I_2, I_3, \ldots, I_{k+1}\) is a family of \(k\) ideals which are each comaximal to \(J\), and by the induction hypothesis the result holds.
\end{proof}
