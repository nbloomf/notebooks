\begin{dfn}[Divides]
Given integers \(a\) and \(b\), we say that \(a\) \emph{divides} \(b\), written \(a|b\), if there is an integer \(c\) such that \(ac = b\).
In this case we say that \(a\) is a \emph{divisor} of \(b\).
\end{dfn}

\begin{prop}
The following hold for all integers \(a\) and \(b\).
\begin{itemize}
\item \(a|0\).
\item \(1|a\).
\item \(a|a\).
\item If \(a|b\), then \((-a)|b\) and \(a|(-b)\).
\item If \(a|b\) and \(b \neq 0\), then \(0 < |a| \leq |b|\).
\end{itemize}
\end{prop}

\begin{dfn}
Let \(a\) and \(b\) be integers.
\begin{itemize}
\item We say that an integer \(c\) is a \emph{common divisor} of \(a\) and \(b\) if \(c|a\) and \(c|b\).
\item We say that an integer \(d\) is a \emph{greatest common divisor} of \(a\) and \(b\) if \(d\) is a common divisor, and if \(c\) is another common divisor, then \(c \leq d\).
\end{itemize}
\end{dfn}

\begin{prop}
Any two integers (not both zero) have a unique greatest common divisor, which we denote \(\GCD{a}{b}\). We also define \(\GCD{0}{0} = 0\) as a special case.
\end{prop}

\begin{prop} \mbox{}
\begin{itemize}
\item \(\GCD{a}{b} = \GCD{b}{a}\) for all integers \(a\) and \(b\).
\item \(\GCD{a}{a} = |a|\) for all integers \(a\).
\item If \(a\) and \(b\) are integers with \(b|a\), then \(\GCD{a}{b} = |b|\).
\item \(\GCD{a}{1} = 1\) for all integers \(a\).
\item \(\GCD{a}{0} = |a|\) for all integers \(a\).
\end{itemize}
\end{prop}

\begin{prop}[Euclidean Algorithm for \(\ZZ\)]
If \(a\) and \(b\) are integers with \(b > 0\), and if \(a = qb + r\) where \(0 \leq r < b\), then \(\GCD{a}{b} = \GCD{b}{r}\).
\end{prop}

\begin{proof}
Let \(d = \GCD{a}{b}\) and \(e = \GCD{b}{r}\). We need to show that \(d = e\); to do this, we will show that \(d \leq e\) and \(e \leq d\).
\begin{itemize}
\item By definition we have \(d|a\) and \(d|b\); that is, \(a = da'\) and \(b = db'\) for some integers \(a'\) and \(b'\). Now \[ r = a - qb = da' - qdb' = d(a' - qb'), \] so that \(d|r\). In particular, \(d\) is a common divisor of \(b\) and \(r\), and so \(d \leq e\).
\item Similarly, we have \(e|b\) and \(e|r\), so that \(e|a\), and thus \(e \leq d\). \qedhere
\end{itemize}
\end{proof}

The Euclidean Algorithm gives us a way to explicitly compute the GCD of two integers \emph{as long as} we can compute quotients and remainders as in the Division Algorithm; in fact, it is quite fast. Note that since \(r\) is strictly less than \(b\), this recursion must eventually terminate with a statement of the form \(\GCD{a}{0}\).


\begin{thm}[Bezout's Identity]
If \(a\) and \(b\) are integers, then there exist integers \(u\) and \(v\) such that \(\GCD{a}{b} = ua + vb\).
\end{thm}

\begin{proof}
We start with the case \(b \geq 0\), proceeding by strong induction.
\begin{itemize}
\item \textbf{Base Case} (\(b = 0\)): Note that \(\GCD{a}{0} = a = a \cdot 1 + 0 \cdot 0\) as needed. That is, the result holds with \(u = 1\) and \(v = 0\).
\item \textbf{Base Case} (\(b = 1\)): Note that \(\GCD{a}{1} = 1 = a \cdot 0 + 1 \cdot 1\) as needed. That is, the result holds with \(u = 0\) and \(v = 1\).
\item \textbf{Inductive Step}: Suppose the result holds for all integers \(b'\) with \(0 \leq b' < b\), where \(b > 1\). That is, for all such \(b'\) and all integers \(a\) there exist integers \(u\) and \(v\) such that \(\GCD{a}{b'} = au + b'v\). Now consider \(b\). By the division algorithm we have integers \(q\) and \(r\) such that \(a = qb + r\) and \(0 \leq r < b\). We have two possibilities to consider.
\begin{itemize}
\item If \(r = 0\), then in fact \(b|a\), since \(a = qb\). So \(\GCD{a}{b} = b = a \cdot 0 + q \cdot b\). That is, the result holds with \(u = 0\) and \(v = 1\).
\item If \(r > 0\), then by the induction hypothesis there exist integers \(u'\) and \(v'\) such that \(\GCD{b}{r} = bu' + rv'\). By the euclidean algorithm, we have
\begin{eqnarray*}
\GCD{a}{b} & = & \GCD{b}{r} \\
 & = & bu' + rv' \\
 & = & bu' + (a - qb)v' \\
 & = & av' + b(u' - qv').
\end{eqnarray*}
That is, the result holds with \(u = v'\) and \(v = u' - qv'\).
\end{itemize}
\end{itemize}
By Strong Induction, for all \(b \geq 0\) and all integers \(a\) there exist integers \(u\) and \(v\) such that \(\GCD{a}{b} = au + bv\).

Now suppose \(b < 0\), so that \(-b > 0\).
By the previous discussion, there exist integers \(u'\) and \(v'\) such that \(\GCD{a}{-b} = au' + (-b)v'\).
Now \[ \GCD{a}{b} = \GCD{a}{-b} = au' + (-b)v' = au' + b(-v'). \]
That is, the result holds with \(u = u'\) and \(v = -v'\).  
\end{proof}

Similar to the Euclidean Algorithm, this proof of Bezout's Identity provides us with a strategy for actually finding the coefficients \(u\) and \(v\) recursively.

\begin{dfn}[Relatively Prime]
We say that integers \(a\) and \(b\) are \emph{relatively prime} if \(\GCD{a}{b} = 1\).
\end{dfn}

\begin{thm}[Euclid's Lemma]
If \(a\) and \(b\) are relatively prime integers and \(c\) an integer such that \(a | bc\), then \(a|c\).
\end{thm}

\begin{proof}
By Bezout's Identity, we have \(1 = au + bv\) for some integers \(u\) and \(v\); so \(c = auc + bvc\).
Since \(a|bc\), we have \(bc = at\) for some integer \(t\).
Thus \[ c = auc + bvc = auc + atv = a(uc + tv), \] and so \(a|c\) as claimed.
\end{proof}

